\chapter*{Abstract}


In source-channel communication with a fidelity criterion and a transmission
cost constraint, the region of achievable cost and fidelity pairs is completely
characterized by Shannon's separation theorem. However, this is true only if
coding of arbitrary complexity and delay is admitted. If the delay is
constrained, the separation theorem only provides an outer bound to the
achievable cost/distortion region, and its exact shape is in general not known.

The first part of this thesis studies delay-limited communication of
continuous-valued sources over the additive white Gaussian noise channel, with a
squared-error distortion measure and a channel input power constraint, when a
\emph{single} source symbol is to be encoded at a time. In particular, the case
is considered where several uses of the channel can be made for each source
symbol. Inspired by communication with feedback, a simple communication strategy
based on quantization and uncoded communication is derived.  It is analyzed how
the smallest mean squared error achievable using this strategy scales as a
function of the signal-to-noise ratio; it turns out that the asymptotic
performance is strictly suboptimal, in the sense that the gap (in dB) between
the achievable performance and the best performance achievable without a delay
limit grows with increasing SNR. Several extensions of this communication
strategy are studied and shown to suffer from the same performance
limitations. On the other hand, there is no known communication strategy that
encodes a single source symbol at a time yet achieves a mean squared error that
scales better. 

The second part of the thesis studies source-channel communication when neither
a required fidelity nor a maximum cost are specified, but when the goal is to
maximize the \emph{ratio} between fidelity and cost. It is shown how the maximal
ratio relates to existing quantities such as the capacity per unit cost, and
necessary and sufficient conditions are derived to test whether a given system
operates at this maximal ratio. 

The thesis turns to a more practical subject in its last part. The case is made
why object-oriented programming is particularly suited to implementing
simulations. As a proof of concept, a complete implementation of an
object-oriented simulator for source-channel coding is presented that allows for
rapid development and analysis of arbitrary communication strategies.

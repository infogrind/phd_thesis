\chapter*{Abstract}
\nocite{KleinerR2009a}



In point-to-point source-channel communication with a fidelity criterion and a transmission cost constraint, the region of achievable cost and fidelity pairs is completely characterized by Shannon's separation theorem. However, this is in general only true if coding of arbitrary complexity and delay is admitted. If the delay is constrained, the separation theorem only provides an outer bound to the achievable cost\slash distortion region, and the exact shape of this region is in general not known.

The first part of this thesis studies communication of continuous-valued sources
over the additive white Gaussian noise channel when only a \emph{single} source
symbol is to be encoded at a time. In particular, the case is considered where
several uses of the channel can be made for each source symbol. Inspired by
communication with feedback, a simple communication strategy based on
quantization and uncoded transmission is derived and analyzed. It is shown that
this strategy achieves a mean squared error that performs as well as any known
communication strategy that encodes a single source symbol at a time. On the
other hand, its asymptotic performance is strictly suboptimal, in the sense that
the gap (in dB) between the achievable signal-to-distortion ratio (SDR) and the
best SDR achievable without a delay limit grows with increasing signal-to-noise
ratio. An extension of this scheme to larger block lengths is shown to suffer
from the same performance limitations. 

The second part of the thesis studies source-channel communication when neither
a required average fidelity nor a maximum cost are specified, but when the goal
is to maximize the \emph{ratio} of fidelity to cost. It is shown how the
maximal ratio relates to existing quantities such as the capacity per unit cost.
Finally, necessary and sufficient conditions are derived to test whether a given
system operates at this maximal ratio and when this is possible using a
single-letter code.

The thesis turns to a more practical subject in its last part. The case is made why object-oriented programming is particularly suited to implementing simulations. As a proof of concept, a complete implementation of an object-oriented simulator for source-channel coding is presented that allows for rapid development and analysis of arbitrary communication strategies.

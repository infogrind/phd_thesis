\chapter[A Hybrid Communication Strategy for Gaussian Channels]{Analysis of a
Hybrid Communication Strategy for the Gaussian Channel} \label{ch:mindelbwex}


%%%%%%%%%%%%%%%%%%%%%%%%%%%%%%%%%%%%%%%%%%%%%%%%%%%%%%%%%%%%%%%%%%%%%%%%%%%%%%%%
% SECTION: SCALAR QUANTIZER
%%%%%%%%%%%%%%%%%%%%%%%%%%%%%%%%%%%%%%%%%%%%%%%%%%%%%%%%%%%%%%%%%%%%%%%%%%%%%%%%

Based on the ideas outlined in the previous chapter, this section introduces a
simple communication strategy. The strategy consists of splitting a single
source symbol into $n-1$ discrete parts and a continous part (hence the term
``hybrid''), to be transmitted, respectively, in the first $n-1$ and in the last
channel use. While this strategy emerged as a natural way to take advantage
of the lessons learned from the feedback case, it turns out that the method of
combining quantization and uncoded transmission as done here is not new. It
was first suggested by McRae~\cite{McRae1971} 1971; for more detailed historical
references see \secref{mindelbwexhist} at the end of the chapter. \footnote{The
main results of this chapter have been published in~\cite{KleinerR2009b}
and~\cite{KleinerR2010}.}


\section{Transmission Strategy}\label{sec:commscheme}

The communication strategy described hereafter is displayed schematically in
\figref{1nencoding}.
\begin{figure}
  \begin{center}
    \input{figures/1nencoding.tex_t}
  \end{center}
  \caption{Schematic display of the encoder described in \secref{commscheme} for
  $n = 4$. The triangles represent the scaling operations of~(\ref{eq:QEdefB})
  and~(\ref{eq:Xdef}).}
  \label{fig:1nencoding}
\end{figure}
To encode a single source letter $S$ into $n$~channel input symbols $X_1$,
\dots, $X_n$, it proceeds as follows. Define $E_0 = S$ and recursively compute
the pairs $(Q_i, E_i)$ as
\begin{subequations}\label{eq:QEdef}
\begin{align}
  Q_i &= \frac{1}{\beta}\Int(\beta E_{i-1}) \label{eq:QEdefA} \\
  E_i &= \beta (E_{i-1} - Q_i) \label{eq:QEdefB}
\end{align}
\end{subequations}
for $i = 1$, \dots, $n-1$ where $\Int(x)$ is the unique integer~$i$ satisfying
\begin{equation*}
  x \in \left[i - \frac12, i +\frac12\right).
\end{equation*}

The equations~\ref{eq:QEdef} define a \emph{hierarchical quantization} of the
source. $Q_1$ is the quantized source symbol and $E_1$ the associated
quantization error. $Q_2$ is the quantized version of the previous quantization
error~$E_1$, and so on. $E_{n-1}$ is the remaining quantization error after
$n-1$ steps.

The parameter $\beta \in \NN$ determines the quantization
resolution; the larger~$\beta$, the finer the quantization.
Equation~\ref{eq:QEdef} implies a partition of the source space into intervals
of length $1/\beta^{n-1}$, as illustrated in \figvref{sourcepartition}. The
$Q_i$ determine the interval that contains~$S$, and $E_{n-1}$ determines the
position of~$S$ within an interval. 

\begin{figure}
  \begin{center}
    \input{figures/sourcepartition.tex_t}
  \end{center}
  \caption{Example partition of the source space $[-1/2,1/2]$ for $\beta = 3$
  and $n = 4$. }
  \label{fig:sourcepartition}
\end{figure}


The following result will be useful in the sequel.

\begin{lemma}
  \label{lem:Qvarbound}
  For all $i = 1$, \dots, $n-1$, the variance of $Q_i$ satisfies
  \begin{equation*}
    \E[Q_i^2] \le \E[E_{i-1}^2] + \frac{\sqrt{\E[E_{i-1}^2]}}{\beta} +
    \frac{1}{4\beta^2}.
  \end{equation*}
\end{lemma}

\begin{proof}
  See \appref{qvarboundproof}.
\end{proof}

\begin{proposition}
  \label{prop:qeproperties}
  The $Q_i$ and $E_i$ satisfy the following properties:
\begin{enumerate}
  \item The map $S \mapsto (Q_1, \dots, Q_{n-1}, E_{n-1})$ is one-to-one, with
    the inverse given by
    \begin{equation}
      \label{eq:unwraprec}
      S = \sum_{i=1}^{n-1} \frac{1}{\beta^{i-1}} Q_i + \frac{1}{\beta^{n-1}}
      E_{n-1}.
    \end{equation}

  \item There exists a constant~$\gamma > 0$ such that $\Var Q_i \le \gamma^2$
    and $\Var E_i \le \gamma^2$ for all~$i$, regardless of the value of~$\beta$.
\end{enumerate}
\end{proposition}

\begin{proof}
  \begin{enumerate}
    \item From the definition~\eqref{eq:QEdefB}, with $E_0 = S$,
    \begin{equation}
      \label{eq:reverserec}
      E_{i-1} = \frac{1}{\beta} E_i + Q_i.
    \end{equation}
    Repeated use of this relationship leads to the given expression for~$S$. 

  \item First, $\Var E_0 = \Var S = \ssq$, which doesn't depend
    on~$\beta$. For $i = 1$, \ldots, $n-1$, $E_i \in [-1/2, 1/2)$ and so its
    variance is upper bounded as well. Finally, since $\beta \ge 1$ and by
    \lemref{Qvarbound}, 
    \begin{align*}
      \Var Q_i \le \E[Q_i^2] \le \E[E_{i-1}^2] + \sqrt{\E[E_{i-1}^2]} +
    \frac{1}{4}.
    \end{align*}
    which completes the proof.
  \end{enumerate}
\end{proof}

To complete the description of the transmission strategy, the $Q_i$ and
$E_{n-1}$ are scaled to satisfy the power constraint, resulting in the channel
inputs
\begin{align}
  X_i &= (\sqrt{P}/\gamma) Q_i \quad
  \text{for $i = 1$, \dots, $n-1$ and} \nonumber\\
  X_n &= (\sqrt{P}/\gamma) E_{n-1}.
  \label{eq:Xdef}
\end{align}
Following Proposition~\ref{prop:qeproperties}, this ensures that $\E[X_i^2] \le
P$ for all~$i$. 

\begin{remark}
  \label{rem:sqgeometry}
  The signal locus resulting from this encoding scheme is precisely the one
  illustrated in \figref{hybridlocus}. Each of the segments of length
  $\beta^{-(n-1)}$ of the source (cf.~\figref{sourcepartition}) is mapped into a
  line segment of length $\sqrt{P}/\gamma$; the stretch of the signal curve, as
  defined in \secref{geomviewpoint}, is therefore $\beta^{n-1}/\gamma$.
\end{remark}

% [Marius] - Removed this since the bound SDR <= 2^{2 I(X;Y)} is no longer
% explicitly mentioned. Moreover, the argument is only valid for sources with
% bounded support. 
%\begin{remark}
%  \label{rem:betagrow}
%  If the quantization resolution $\beta$ is upper bounded then the SDR scales at
%  best as $\snr$, as a simple argument shows. If $\beta \le \beta_{\max}$, the
%  entropy of the $Q_i$ is upper bounded, and so is the entropy of the $X_i$ for
%  $i = 1$, \ldots, $n$. As $\snr \goesto \infty$, the entropy of the $Y_i$
%  converges to that of the $X_i$, and so for all $i = 1$, \ldots, $n-1$, the
%  mutual information $I(X_i; Y_i)$ is upper bounded by a constant independent of
%  the SNR. Since the SDR scales at best as $2^{2I(X^n;Y^n)}$ (see
%  Chapter~\ref{ch:fundamentals}), this limits the SDR scaling to be at best linear in
%  the SNR.
%\end{remark}


\section{Lower Bound on the Mean Squared Error}\label{sec:scalarlowerbound}

When an analog source is transmitted across a Gaussian channel using the scheme
just presented, two types of decoding errors can occur. Either the decoded point
is on the same line segment as the transmitted point (\ie, all the $Q_i$ are
decoded correctly) or it is on a different segment. Which one of these errors
dominates the overall error behavior depends on how the parameter~$\beta$ is
chosen as a function of the SNR: if $\beta$ grows fast with the SNR the
quantization resolution quickly becomes very fine, and the line segments of the
constellation move close together fast, so the probability of decoding the wrong
segment is high. On the other hand, since the stretch of the signal curve is
proportional to~$\beta^{n-1}$ (\remref{sqgeometry}), the error decreases fast in
case the correct segment is decoded. Conversely, if $\beta$ grows only slowly
with the SNR then decoding the wrong segment is unlikely. But because the
stretch also only grows slowly, the error when the correct segment is decoded
decreases only slowly.

The following results make these reflections precise. The results apply to any
source; provided that there is an interval on which the source distribution
admits a density. Based on these results one can choose $\beta$ as a function of
the SNR in order to optimize the MSE scaling. 

\begin{remark}
  \label{rem:betaepswlog}
  Throughout this section it is assumed that $\beta = \lceil
  \snr^{(1-\e)/2}\rceil$, where $\e = \e(\snr)$ is a positive function
  of~$\snr$. This results in no loss of generality, since for an arbitrary
  positive function~$f$ one can set $\e(\snr) = 1-2 \log(f(\snr))/\log\snr$ to
  get $\beta(\snr) = \lceil f(\snr) \rceil$.  Writing $\beta$ in this form will
  slightly simplify the mathematical derivations to follow. Note that one can
  bound $\beta$ by $\snr^{(1-\e)/2} \le \beta \le \snr^{(1-\e)/2} + 1$, which
  implies $\beta \in \Theta(\snr^{(1-\e)/2})$ (the $\Theta$-notation is defined
  in \appref{asymptotic}). Choosing an optimal $\beta(\snr)$ is therefore
  equivalent to choosing an optimal~$\e(\snr)$.
\end{remark}

\begin{remark}
  \label{rem:functionnotation}
  Note that by~\eqref{eq:QEdef} the $Q_i$ are completely determined by~$S$.
  With a slight abuse of notation, $Q_i(s)$ is therefore used in the sequel to
  denote the value of~$Q_i$ when $S = s$. $E_i(s)$ and $X_i(s)$ are defined in
  an analogous manner. Furthermore, $\X(s) \deq (X_1(s), \dots, X_n(s))$.
\end{remark}


To obtain a lower bound on the MSE that holds for all possible decoders, the
obvious thing to do is to assume a minimum mean squared error (MMSE) decoder. As
mentioned in this chapter's introduction, though, the MMSE decoder is in general
hard to evaluate mathematically. This is no different for the scheme at hand.
Fortunately the following lemma, due to Ziv~\cite{Ziv1970}, presents a way
around the MMSE decoder. Moreover, unlike the MMSE decoder, it does not depend
on the source distribution.


\begin{lemma}
  \label{lem:zivbound}
  Consider a communication system where a con\-tin\-u\-ous-valued source~$S$ is
  encoded into an $n$-dimensional vector $\X(S)$, sent across $n$~independent
  parallel AWGN channels with noise variance~$\szq$, and decoded at the receiver
  to produce an estimate~$\Sh$.  If the density $p_S$ of the source is such that
  there exists an interval $[A,B]$ and a number $p_{\min} > 0$ such that $p_S(s)
  \ge p_{\min}$ whenever $s \in [A,B]$, then for any $\Delta \in [0,B-A)$ the
  mean squared error incurred by the communication system satisfies
  \begin{equation}
    \label{eq:zivbound}
    \E[(\Sh - S)^2] \ge p_{\min} \left(\frac{\Delta}{2} \right)^2 
    \int_A^{B-\Delta} Q(d(s, \Delta) / 2 \sz) ds,
  \end{equation}
  where $d(s, \Delta) \deq \|\vect{X}(s) - \vect{X}(s+\Delta)\|$ and 
  \[Q(x) = \int_x^{\infty} (1/\sqrt{2\pi}) \exp\{-\xi^2/2\} d\xi.\]
\end{lemma}

\begin{proof}
  See Appendix~\ref{app:zivboundproof}.
\end{proof}

The next two lemmas provide two different asymptotic lower bounds on the mean
squared error of the hybrid transmission strategy considered. They hold
regardless of the decoder used.  (The $\Omega$-notation is defined in
Appendix~\ref{app:asymptotic}.)

\begin{lemma}
  \label{lem:lowerbound1}
  Given a hybrid transmission strategy characterized by $\e(\snr) \ge 0$, the
  mean squared error satisfies
  \begin{equation*} \mse \in \Omega(\snr^{-n +
    (n-1)\e}).  \end{equation*}
\end{lemma}

\begin{lemma}
  \label{lem:lowerbound2}
  Given a hybrid transmission strategy characterized by $\e(\snr) \ge 0$, the
  mean squared error satisfies
  \begin{equation*}
    \mse \in \Omega(\snr^{-1+\e/2} \exp\{-c\snr^\e\}),
  \end{equation*}
  where $c>0$ does not depend on~$\snr$.
\end{lemma}

\emph{Discussion:} An immediate consequence of the lemmas is that the
scaling $\snr^{-n}$ is not achievable with the given
encoding strategy: by Lemma~\ref{lem:lowerbound1} this would require $\e = 0$,
but following Lemma~\ref{lem:lowerbound2} the scaling is at best $\snr^{-1}$ if
$\e = 0 $.  More generally, which one of the two lower bounds decays more slowly
and is therefore tighter depends on the scaling of~$\e(\snr)$. How to
choose~$\e(\snr)$ optimally will be the subject of Theorem~\ref{thm:scalinglb}.

\begin{proof}[Proof of Lemma~\ref{lem:lowerbound1}]
  Assume $\Delta \in [0, \beta^{-(n-1)})$ and define for $j \in \Z$
  \[ \I_j^\Delta = \left[ (j - \frac12 )\beta^{-(n-1)}, 
    (j + \frac12 ) \beta^{-(n-1)} - \Delta \right).\]
  These intervals will be used to partition the source space, as illustrated in
  \figref{lowerbound1intervals}.

  It can be verified from~\eqref{eq:QEdef} that if $s \in \I_j^\Delta$ for
  some~$j$, the following properties hold: 1) $Q_i(s) = Q_i(s+\Delta)$ for
  $i=1$, \dots, $n-1$, and 2) $E_{n-1}(s+\Delta) - E_{n-1}(s) =
  \beta^{n-1}\Delta$. Geometrically this means that $s$ and~$s + \Delta$ are
  mapped to the same straight line segment; see \figref{sdrubproofa}.
  From~\eqref{eq:Xdef} it follows that $s \in \I_j^\Delta$ implies
  \begin{equation*}
    d(s, \Delta) = \|\X(s) - \X(s+\Delta)\| = \sqrt{P/\gamma^2}
    \beta^{n-1}\Delta.
  \end{equation*}

  Now apply Lemma~\ref{lem:zivbound} and restrict the integral to the
  set~$\psi(\Delta) \deq [A,B-\Delta) \cap \bigcup_{j\in\Z} \I_j^\Delta$. The
  lower bound is then relaxed to give
  \begin{equation*}
    \mse \ge \frac{\pmin}{4} \Delta^2 Q(\sqrt{\snr/\gamma^2} \beta^{n-1}
    \Delta/2) \int_{\psi(\Delta)} ds.
  \end{equation*}
  Letting $\Delta = 1/(\sqrt{\snr}\beta^{n-1})$ and using $\beta^2 \in
  \Theta(\snr^{1-\e})$ (see \remref{betaepswlog}) yields (for sufficiently
  large~$\snr$)
  \begin{equation*}
    \mse \ge c \snr^{-n+(n-1)\e} Q\left(1/2\gamma\right)
    \int_{\psi(\Delta)} ds.
  \end{equation*}

  The proof is almost complete; it only remains to show that
  $\int_{\psi(\Delta)}ds$ can be lower bounded by a constant for large SNR. The
  length of a single interval~$\I_j^\Delta$ is $\beta^{-(n-1)} - \Delta$. Within
  $[A,B-\Delta)$ there are $(B-A-\Delta)\beta^{n-1}$ such intervals (see
  \figref{lowerbound1intervals}). The total
  length of all intervals~$\I_j^\Delta$ in $[A, B-\Delta)$ is therefore
  \[ \int_{\psi(\Delta)} ds = (B-A-\Delta)
  (1 - \beta^{n-1}\Delta), \]
  which, for the given values of~$\beta$ and~$\Delta$, 
  converges to $B-A$ for $\snr \ra \infty$ and thus can be lower bounded by a
  constant for $\snr$ greater than some $\snr_0$. With this, the proof is
  complete.
\end{proof}

\begin{figure}
  \begin{center}
    \input{figures/lowerbound1intervals.tex_t}
  \end{center}
  \caption{Illustration of the intervals $\I_j^\Delta$ in the proof of
  \lemref{lowerbound1}. The horizontal line represents a subset of the source
  space. The length of each $\I_j^\Delta$ is $\beta^{-(n-1)} -
  \Delta$, and they are placed such that within an interval of length~$\ell$
  there are approximately $\ell \beta^{n-1}$ of them.}
  \label{fig:lowerbound1intervals}
\end{figure}

\begin{proof}[Proof of Lemma~\ref{lem:lowerbound2}]
  Observe first that~\eqref{eq:QEdef} implies $Q_1(s + \beta^{-1}) = Q_1(s) +
  \beta^{-1}$ and $E_1(s + \beta^{-1}) = E_1(s)$. Since all $Q_i$ and $E_i$ for
  $i \ge 2$ are by recursion a function of $E_1$ only, $Q_i(s) = Q_i(s +
  \beta^{-1})$ for $i = 2$, \dots, $n-1$, and $E_{n-1}(s) = E_{n-1}(s +
  \beta^{-1})$. Consequently,  $X_i(s) = X_i(s + \beta^{-1})$ for all $i =
  2$, \dots, $n$. Geometrically speaking, $s$ and $s + \beta^{-1}$ are mapped to
  the same position on two adjacent segments of the signal locus; see
  \figref{sdrubproofb}. By~\eqref{eq:Xdef} and the above, the Euclidean distance
  between $\X(s)$ and~$\X(s+\beta^{-1})$ is therefore
  \begin{equation}
    \label{eq:xbetadist}
    d(s, \beta^{-1}) = \frac{\sqrt P}{\gamma} |Q_1(s) - Q_1(s+\beta^{-1})| 
    = \frac{\sqrt P}{\gamma\beta}.
  \end{equation}

  Apply now \lemref{zivbound} with $\Delta = \beta^{-1}$. The
  parameter $\beta$ will be chosen to increase with the SNR, therefore $\Delta
  \in [0, B-A)$ holds for sufficiently large values of~$\snr$.
  Using~\eqref{eq:xbetadist}, the resulting bound on the mean squared error is
  \begin{equation*}
    \mse \ge \frac{\pmin}{4} \beta^{-2}
    Q\left (\frac{\sqrt{\snr}}{2\gamma\beta}  \right) (B-A-\beta^{-1}
    ).
  \end{equation*}
  Because $\beta^2 \in \Theta(\snr^{1-\e})$ (see \remref{betaepswlog}),
  $\sqrt{\snr}/\beta \in \Theta(\snr^{\e/2})$. If $\e(\snr)$ is such that
  $\limtoinf{\snr} \snr^{\e(\snr)} = \infty$, use the fact that $Q(x)$~converges
  to $e^{-x^2/2}/\sqrt{2\pi}x$ (cf.~\cite[\Spg 26.2.12]{AbramowitzS1964}).
  Otherwise $\snr^{\e(\snr)}$ is upper bounded by a constant, in which case
  $Q(x)$ is bounded above and below by constants for $x = \snr^{\e(\snr)}$, and
  $e^{-x^2/2}/x$ is equally upper and lower bounded.  In any case, for
  sufficiently large values of~$\snr$,
  \begin{equation*}
    \mse \ge c_1 \snr^{-1 + \e/2} \exp\{-c_2\snr^\e\},
  \end{equation*}
  with $c_1$ and $c_2$ positive constants that do not depend on~$\snr$, thus
  proving the lemma.
\end{proof}

\begin{figure}
  \centerline{%
  \subfloat[$\Delta \in [0, \beta^{-(n-1)})$ and $s \in \I_j^\Delta$]%
  {\label{fig:sdrubproofa}\input{figures/sdrubproofa.tex_t}}%
  \hfil
  \subfloat[$\Delta =
  \beta^{-1}$]{\label{fig:sdrubproofb}\input{figures/sdrubproofb.tex_t}}
  } % end centerline
  \caption{Geometric view of the proofs of \lemref{lowerbound1} (left) and
  \lemref{lowerbound2} (right) for $n = 2$. In the former case, $s$ and $s +
  \Delta$ are mapped to the same segment of the signal locus, whereas in the
  latter case they are mapped to adjacent segments but at the same ``height''.}
  \label{fig:sdrubproof}
\end{figure}

\begin{remark}
  \label{rem:geomlb}
  \figref{sdrubproof} illustrates the connection to the geometric
  argument given in \secref{geomviewpoint}. The MSE is lower bounded by
  \lemref{lowerbound1} in case the correct fold of the signal curve is decoded
  and by \lemref{lowerbound2} in case the wrong fold is decoded. The
  parameter~$\e$ represents the tradeoff between the two kinds of errors.
\end{remark}

The following lemma will be used to prove Theorem~\ref{thm:scalinglb}, the main
result of this section.

\begin{lemma}
  \label{lem:epssolution}
   For $\snr > 1$ and arbitrary real constants $a$, $b>0$, and $c > 0$, it holds
   that
  \begin{equation}
    \label{eq:epsequation}
    \snr^{a+b\e} = \exp\{-c\snr^\e\},
  \end{equation}
  if and only if
  \begin{equation}
    \label{eq:epssolution}
    \snr^\e = (b/c) W(c\snr^{-a/b} / b),
  \end{equation}
  where $W(x)$ is the function that satisfies $W(x)e^{W(x)} = x$ for $x > 0$.
  This function is well defined and is sometimes called the \emph{Lambert
  $W$-function}~\textnormal{\cite{CorlessGHJK1996}}.
\end{lemma}

\begin{proof}
  Let $\snr>1$. Since $\snr^{a+b\e}$ is strictly increasing and
  $\exp\{-c \snr^\e\}$ is strictly decreasing in~$\snr^\e$, there is at most one
  solution to~\eqref{eq:epsequation} in~$\snr^\e$.  Assume now $\snr^\e$ is as
  in~\eqref{eq:epssolution}. Then
  \begin{equation*}
    \exp\{-c \snr^\e\} = \exp\{-b W(c \snr^{-a/b}/b)\}.
  \end{equation*}
  On the other hand,
  \begin{align*}
    \snr^{a+b\e} &= \snr^a \left( (b/c) W(c\snr^{-a/b}/b) \right)^b \\
    &= \left( W(c\snr^{-a/b}/b) / (c\snr^{-a/b}/b) \right)^b.
  \end{align*}
  By definition, $W(x)/x = e^{-W(x)}$, so the above is equal to
  \begin{equation*}
    \snr^{a+b\e} = \exp\{-bW(c \snr^{-a/b}/b)\},
  \end{equation*}
  which proves the claim.
\end{proof}

Lemmas~\ref{lem:lowerbound1} and~\ref{lem:lowerbound2} provide conflicting
objectives for the choice of~$\e$: according to
\lemref{lowerbound1}, $\e$ should be small for the MSE to decay fast, but
according to \lemref{lowerbound2} it should be large to have a fast exponential
decay of the error. 
The following theorem, which is the main result of this section, is obtained by
finding the $\e$ for which the two lower bounds scale the same.
\begin{theorem}
  \label{thm:scalinglb}
  For any parameter~$\beta$ and for any decoder, the mean squared error of the
  hybrid transmission strategy described in this section satisfies
  \begin{equation*}
    \mse \in \Omega(\snr^{-n}(\log\snr)^{n-1}).
  \end{equation*}
\end{theorem}

% [Marius] Removed the following paragraph, since in this chapter the
% achievability result comes _after_ the bound.
%\emph{Discussion:} The asymptotic lower bound on the mean squared error given
%by the theorem coincides with the asymptotic performance achieved by the
%suboptimal decoder in Section~\ref{sec:achievable}; the bound is therefore
%asymptotically tight. 

\begin{proof}
  For notational simplicity define
  \begin{align*}
    l_1(\snr, \e) &= \snr^{-n+(n-1)\e} \quad\text{and} \\
    l_2(\snr,\e) &= \snr^{-1+\e/2} \exp\{-c\snr^\e\}.
  \end{align*}
  By Lemmas~\ref{lem:lowerbound1} and~\ref{lem:lowerbound2},
  \begin{equation*}
    \mse \in \Omega\big(\max \left( l_1(\snr,\e), l_2(\snr,\e) \right) \big).
  \end{equation*}
  The optimal parameter $\e(\snr)$ is therefore such that for
  any~$\snr$
  \begin{equation}
    \label{eq:epsmax}
    \max\left( l_1(\snr,\e), l_2(\snr,\e) \right)
  \end{equation}
  is minimized. Now for any fixed~$\snr$, $l_1(\snr,\e)$ is increasing in~$\e$
  and $l_2(\snr,\e)$ is increasing in~$\e$ for $0 \le \e < \xi =
  \log(1/2c)/\log\snr$ and decreasing in~$\e$ for $\e \ge \xi$.  The maximum
  in~\eqref{eq:epsmax} is therefore minimized either for $\e = 0$ or for $\e \ge
  \xi$ such that $l_1(\e) = l_2(\e)$. As remarked before, $\e = 0$ leads to
  $\mse \in \Omega(\snr^{-1})$. In the following, let thus $\e(\snr)$ be such
  that $l_1(\snr,\e) = l_2(\snr, \e)$, to see whether this gives a better lower
  bound.  Inserting the definitions of $l_1$ and $l_2$ and rearranging the terms
  yields
  \begin{equation*}
    \snr^{-(n-1) + (n-3/2)\e} = \exp\{-c\snr^\e\},
  \end{equation*}
  which is of the form~\eqref{eq:epsequation} with $a = -(n-1)$ and $b = n-3/2$.
  By Lemma~\ref{lem:epssolution}, for $\snr > 1$,
  \begin{equation*}
    \snr^\e = \frac{n-3/2}{c}
    W\left( \frac{c\snr^{\frac{2(n-1)}{2n-3}}}{n-3/2} \right).
  \end{equation*}
  Using L'H\^opital's rule and because the derivative of
  $W(x)$ is $W(x)/[x(1 + W(x))]$ (cf.~\cite{CorlessGHJK1996}), it is
  straightforward to check that $W(x)/\log x$ converges to~$1$ for $x \ra
  \infty$.  For sufficiently large $\snr$, therefore, there exists a constant
  $c_1 > 0$ such that
  \begin{equation*}
    \snr^\e \ge c_1 \frac{n-3/2}{c} \left[ \frac{2(n-1)}{2n-3}\log\snr -
    \log\left(\frac{n-3/2}{c}\right)
    \right],
  \end{equation*}
  and so $\snr^\e \in \Omega(\log\snr)$. Plugging this into the bound of
  Lemma~\ref{lem:lowerbound1} finally results in\footnote{If $a(x) \in
  \Omega(f(x))$ and $b(x) \in \Omega(g(x))$, then $a(x)b(x)^m \in
  \Omega(f(x)g(x)^m)$.}
  \begin{equation*}
    \mse \in \Omega(\snr^{-n}(\log\snr)^{n-1}),
  \end{equation*}
  and no choice of $\e(\snr)$ can improve this bound.
\end{proof}


\section{Asymptotic Achievability of Lower Bound}
\label{sec:achievable}

The previous section showed that the hybrid transmission strategy 
achieves an SDR scaling of at best $\snr^n/(\log\snr)^{n-1}$. Using a simple
decoder this scaling is in fact achievable, as will now be shown.

\subsection{A Suboptimal Decoder}

The $X_i$ are transmitted across the channel, producing at the channel output
the symbols
\begin{equation*}
  Y_i = X_i + Z_i, \quad i = 1, \dots, n,
\end{equation*}
where the $Z_i$ are iid Gaussian random variables of variance~$\szq$. 
To estimate $S$ from  $Y_1$, \dots, $Y_n$, the decoder first
computes separate estimates $\Qh_1$, \dots, $\Qh_{n-1}$ and $\Eh_{n-1}$, and
then combines them to obtain the final estimate~$\Sh$.  While this strategy is
suboptimal in terms of achieving a small MSE, it will turn out to be good enough
to achieve the desired SDR scaling.

The $Q_i$ are estimated using a maximum likelihood (ML) decoder, which yields
the minimum distance estimate
\begin{equation}
  \label{eq:mldecoder}
  \Qh_i = \frac{1}{\beta} \arg \min_{j\in \Z} \left| \frac{j\sqrt{P}}
  {\gamma\beta} - Y_i \right|.
\end{equation}
To estimate $E_{n-1}$, a linear minimum mean-square error (LMMSE)
estimator is used (see \eg~\cite[Section~8.3]{Scharf1990}), which computes
\begin{equation}
  \label{eq:lmmse}
  \Eh_{n-1} = \frac{\E[E_{n-1} Y_n]}{\E[Y_n^2]} Y_n.
\end{equation}
Finally, using~\eqref{eq:unwraprec}, the estimate of~$S$ is computed as
\begin{equation}
  \label{eq:unwrapestim}
  \Sh = \sum_{i=1}^{n-1} \frac{1}{\beta^{i-1}} \Qh_i + \frac{1}{\beta^{n-1}}
  \Eh_{n-1}.
\end{equation}


\subsection{Error Analysis}

The overall MSE $\mse$ can be broken up into contributions due to the
errors in decoding $Q_i$ and $E_{n-1}$ as follows. From~\eqref{eq:unwraprec}
and~\eqref{eq:unwrapestim}, the difference between $\Sh$ and $S$ is
\begin{equation*}
  \Sh - S = \sum_{i=1}^{n-1} \frac1{\beta^{i-1}} (\Qh_i - Q_i) + \frac1{\beta^{n-1}}
  (\Eh_{n-1} - E_{n-1}).
\end{equation*}
The error terms $\Qh_i - Q_i$ depend only on the noise of the respective channel
uses and are therefore independent of each other and of $\Eh_{n-1} - E_{n-1}$,
so the error variance can be written componentwise as
\begin{equation}
  \label{eq:totalerror}
  \mse = \sum_{i=1}^{n-1} \frac{1}{\beta^{2(i-1)}} \Eqi +
  \frac{1}{\beta^{2(n-1)}} \Ee, 
\end{equation}
where $\Eqi \deq \E[(\Qh_i - Q_i)^2]$ and $\Ee \deq \E[(\Eh_{n-1} -
E_{n-1})^2]$.

\begin{lemma}
  \label{lem:eqbound}
  For each $i = 1$, \dots, $n-1$, 
  \begin{equation}
    \label{eq:eqidecay}
    \Eqi \in O\left(\exp\{-c \snr/\beta^2\}\right),
  \end{equation}
  where $c > 0$~is a constant.
(The $O$-notation is defined in Appendix~\ref{app:asymptotic}.)
\end{lemma}

\begin{proof}
  Define the interval
  \begin{equation*}
    \I_j = \left[ \frac{(j - \frac12) \sqrt{P}}{\gamma\beta},
    \frac{(j + \frac12) \sqrt{P}}{\gamma\beta } \right).
  \end{equation*}
  According to the minimum distance decoder~\eqref{eq:mldecoder}, $\Qh_i - Q_i
  = j/\beta$ whenever $Z_i \in \I_j$.  The error $\Eqi$ satisfies thus
  \begin{align}
    \E[(\Qh_i - Q_i)^2] &= \frac{1}{\beta^2} \sum_{j \in \Z} j^2 \Pr[Z_i \in
    \I_j]  \nonumber \\
    &= \frac{2}{\beta^2} \sum_{j = 1}^\infty j^2 \Pr[Z_i \in \I_j],
    \label{eq:eqexact}
  \end{align}
  where the second equality follows from the symmetry of the distribution
  of~$Z_i$. Now,
  \begin{equation*}
    \Pr[Z_i \in \I_j] = Q\left( \frac{(j - \frac12) \sqrt{\snr}}{\gamma\beta}
    \right) - Q\left( \frac{(j + \frac12) \sqrt{\snr}}{\gamma\beta } \right),
  \end{equation*}
  where
  \begin{equation*}
    Q(x) = \frac{1}{\sqrt{2\pi}} \int_x^\infty e^{-\xi^2/2} d\xi,
  \end{equation*}
  which can be bounded from above for $x \ge 0$ as
  \begin{equation*}
    Q(x) \le \frac12 e^{-x^2/2}.
  \end{equation*}
  Since $\beta \ge 1$, \eqref{eq:eqexact} is then upper bounded by
  \begin{equation*}
    \Eqi \le \sum_{j=1}^\infty j^2 \exp\left\{ - \frac{(j - 1/2)^2
    \snr}{2\gamma^2\beta^2} \right\}.
  \end{equation*}
  Note that for $j \ge 2$, $(j - 1/2)^2 > j$.  Thus
  \begin{eqnarray}
    \Eqi &\le & \exp \left \{ - \frac{\snr}{8 \gamma^2 \beta^2} \right\}
    \nonumber \\
    & & \mbox{} + 
    \sum_{j = 2}^\infty j^2 \exp \left \{ - \frac{j \snr}{2\gamma^2 \beta^2}
    \right\}. \label{eq:eqibound}
  \end{eqnarray}
  To bound the infinite sum, use 
  \begin{equation}
    \label{eq:geomsum}
    \sum_{j=2}^\infty j^2 p^j \le \sum_{j=1}^\infty j^2 p^j = 
    \frac{p^2+p}{(1-p)^3}
  \end{equation}
  with $p = \exp\{-\snr/2 \gamma^2 \beta^2\}$. The first term
  of~\eqref{eq:eqibound} thus dominates for large values of
  $\snr/\beta^2$ and
  \begin{equation*}
    \Eqi \le c_1\exp\left\{ - \frac{\snr}{c_2 \beta^2} \right\}
  \end{equation*}
  for some~$c_1 > 0$ and $c_2 = 8 \gamma^2$, which completes the proof. 
\end{proof}

\begin{lemma}
  \label{lem:eedecay}
  $\Ee \in O(\snr^{-1})$. 
\end{lemma}
\begin{proof}
  The mean-squared error that results from the LMMSE estimation~\eqref{eq:lmmse}
  is
  \begin{equation}
    \label{eq:lmmse-error}
    \Ee = \seq - \frac{(\E[E_{n-1}
    Y_n])^2}{\E[Y_n^2]}. 
  \end{equation}
  Since
  \begin{equation*}
    Y_n = X_n + Z_n = \frac{\sqrt{P}}{\gamma} E_{n-1} + Z_n,
  \end{equation*}
  $\E[E_{n-1}Y_n] = \sqrt{P}\seq/\gamma$. Moreover, $\E[Y_n^2] = \E[X^2]
  + \E[Z^2] = P\seq / \gamma^2 +\szq$.  Inserting this
  into~\eqref{eq:lmmse-error} yields
  \begin{align*}
    \Ee &= \seq - \frac{P \se^4/\gamma^2}{P\seq/\gamma^2 + \szq} \\
    &= \seq \left( 1 - \frac{P\seq/\gamma^2}{P\seq/\gamma^2 + \szq} \right) \\
    &= \frac{\seq}{1 + \snr\seq/\gamma^2} \\
    & < \frac{\gamma^2}{\snr}.
  \end{align*}
  Since $\gamma$ is independent of the SNR (cf.\
  Proposition~\ref{prop:qeproperties}), $\Ee \in O(\snr^{-1})$ as claimed.
\end{proof}


\subsection{Optimizing the Quantization Resolution}

Recall the formula for the overall error
\begin{equation*}
  \E[(S-\Sh)^2] = \sum_{i=1}^{n-1} \frac{1}{\beta^{2(i-1)}} \Eqi +
  \frac{1}{\beta^{2(n-1)}} \Ee.
\end{equation*}
According to Lemma~\ref{lem:eqbound}, $\Eqi$ decreases exponentially
when $\snr/\beta^2$ goes to infinity. This happens for increasing SNR if
$\beta$~is set \eg\ to
\begin{equation*}
  \beta = \lceil \snr^{(1-\e)/2} \rceil
\end{equation*}
for some $\e > 0$, in which case $\Eqi \in O\left(\exp(-c \snr^\e) \right)$.
From this and Lemma~\ref{lem:eedecay}, the overall error satisfies
\begin{equation}
  \label{eq:overallO}
  \E[(S-\Sh)^2] \in O(\snr^{-(n - \e')}),
\end{equation}
where $\e' = (n-1)\e$ can be made as small as desired.
%The scaling exponent for
%a fixed $\e$ satisfies therefore
%\begin{equation}
%  \label{eq:sdrepsilon}
%  \limtoinf{\snr} \frac{\log\sdr}{\log\snr} \ge
%  \limtoinf{\snr} \frac{\log \ssq + (n - \e') \log\snr}{\log \snr} = n - \e'. 
%\end{equation}

As already mentioned in \remref{geomlb}, the choice of $\e$ represents a
tradeoff: for small $\e$ the error due to the ``discrete'' part vanishes only
slowly, but the scaling exponent in the limit is larger. For larger $\e$, $\Eq$
vanishes quickly but the resulting exponent is smaller. This is illustrated by
the simulation results in \figref{scalarcomparison}.  The remainder of this
section shows how to choose $\e$ as a function of~$\snr$ in order to achieve the
SDR scaling upper bound of \thmref{scalinglb}.

\begin{figure}
  \begin{center}
    \input{figures/matlab/fig_epsilon_envelope.tex_t}
  \end{center}
  \caption{Simulation results illustrating the tradeoff represented by the
  choice of~$\e$ for $n = 3$. For larger~$\e$, the error from decoding the
  discrete signal part decays quickly but the final scaling is worse, while
  for smaller~$\e$ the opposite holds. If $\e$~is chosen optimally as a function
  of~{\normalfont$\snr$}, the resulting performance is the convex hull of the
  collection of all curves.}
  \label{fig:scalarcomparison}
\end{figure}

%\begin{figure}
%  \begin{center}
%    \input{figures/matlab/fig_scalar_comparison.tex_t}
%  \end{center}
%  \caption{Simulation results illustrating the tradeoff represented by the
%  choice of~$\e$ for $n = 3$. For larger~$\e$, the error from decoding the
%  discrete signal part decays quickly but the final scaling is worse, while
%  for smaller~$\e$ the opposite holds. The delay unconstrained curve
%  corresponds to {\normalfont$\sdr = (1 + \snr)^n$}.}
%  \label{fig:scalarcomparison}
%\end{figure}

Let
\begin{equation}
  \label{eq:esnrdecay}
  \e = \e(\snr) = \frac{\log(n \log\snr / c)}{\log\snr},
\end{equation}
where $c$~is the constant indicating the decay of $\Eqi$ in~\eqref{eq:eqidecay}.
With this choice of $\e$,
\begin{align*}
  \Eqi &\in O\left( \exp\left( - c \snr^\e \right) \right) \\
 % &= O\left( \exp\left( - \exp\left( \e \log \snr \right) \right) \right) \\
 &= O(\snr^{-n}),
\end{align*}
hence the overall error is still dominated as in~\eqref{eq:overallO}.
Inserting~\eqref{eq:esnrdecay} in~\eqref{eq:overallO} leads to the following
achievability result, which coincides with the converse result of
\thmref{scalinglb}, asserting that separately decoding the $Q_i$ and $E_{n-1}$
is asymptotically optimal.
\begin{theorem}
  \label{thm:scalarachievability}
  Setting $\beta = \lceil \snr^{(1-\e)/2} \rceil$ with $\e$~as
  in~\eqref{eq:esnrdecay}, the decoder described by
  \eqref{eq:mldecoder}--\eqref{eq:unwrapestim} achieves a mean squared error
  that scales as
  \begin{equation*}
    \mse \in O(\snr^{-n} (\log\snr)^{n-1}).
  \end{equation*}
\end{theorem}




%%%%%%%%%%%%%%%%%%%%%%%%%%%%%%%%%%%%%%%%%%%%%%%%%%%%%%%%%%%%%%%%%%%%%%%%%%%%%%%%
% SECTION: LATTICE QUANTIZERS
%%%%%%%%%%%%%%%%%%%%%%%%%%%%%%%%%%%%%%%%%%%%%%%%%%%%%%%%%%%%%%%%%%%%%%%%%%%%%%%%


\section{Encoding Blocks of Source Symbols using Lattices}
\label{sec:latticequant}

The scalar quantizer scheme described in the previous section can be extended
quite easily to treat blocks of $m$ source symbols using lattices for the
hierarchical quantization. For the reader unfamiliar with lattices or in need of
a refresher, a concise presentation of the necessary concepts and results is
provided in \appref{latticebasics}.

To anticipate the conclusion of this section, it turns out that the SDR scaling
achieved for $m=1$ cannot be improved upon by choosing a larger~$m$. Choosing a
larger~$m$ and thus a lattice of larger dimension can however result in faster
convergence to the asymptotic scaling and thus increase the SDR for low SNR
values (see \figref{sim_latticeperf} for a preview).


\subsection{Transmission Strategy}

The procedure to encode a vector~$\S$ of $m$~source symbols into $mn$~channel
input vectors $\X_1$, \ldots, $\X_n$ using an $m$-dimensional lattice for
quantization is analog to that in \secref{commscheme}, except that now all
involved quantities are $m$-dimensional vectors.

Let $\Lambda$ be some fixed lattice of dimension~$m$ and define $\Evc_0 = \S$.
For $i = 1$, \ldots, $n-1$ define
\begin{align}
  \Q_i &= \QLb(\Evc_{i-1})  \nonumber \\
  \Evc_i &= \beta ( \Evc_{i-1} - \Q_i), \label{eq:latticeQE}
\end{align}
where $\beta \in \NN$ and where $\QLb(\cdot)$ denotes quantization with respect
to the lattice $\Lambda/\beta$. According to \lemref{latquantvar} (in the
appendix), $\E[\|\Q_i\|^2]$~is upper bounded by a constant independent
of~$\beta$.  Moreover, $\Evc_i$ is contained within the Voronoi region
of~$\Lambda$ around the origin, so by \lemref{voronoivarbound}
$\E[\|\Evc_i\|^2]$ is also upper bounded independent of~$\beta$. Let
$\gamma$~denote the common upper bound on~$\E[\|\Q_i\|^2]/m$
and~$\E[\|\Evc_i\|^2]/m$.  The channel inputs are then computed as
\begin{align*}
  \X_i &= \frac{\sqrt{P}}{\gamma} \Q_i \quad \text{for $i = 1$, \dots, $n-1$
  and} \\
  \X_n &= \frac{\sqrt{P}}{\gamma} \Evc_{n-1}.
\end{align*}
By the above argument, this ensures that $\E[\|\X_i\|^2]/m \le P$ for all~$i$.



\subsection{Error Lower Bound}

Ziv's lower bound on the mean squared error introduced in
\secref{scalarlowerbound} allows a straightforward extension to vector sources.
With this, essentially the same argument sequence as in
\secref{scalarlowerbound} can be used to lower bound the mean squared error,
independent of the particular decoder used.  (Again, the quantization resolution
in terms of~$\snr$ is assumed to be $\beta = \lceil \snr^{(1-\e)/2} \rceil$
without loss of generality (cf.~\remref{betaepswlog}).)

\begin{lemma}[Extension of \lemref{zivbound} to vector sources]
  \label{lem:zivboundvec}
  Consider a communication system where a continuous-valued source vector~$\S$
  is encoded into a vector~$\X(\S)$, sent across independent parallel scalar
  AWGN channels with noise variance~$\szq$, and decoded at the receiver to
  produce an estimate~$\Shv$. If the density $f_{\S}(\s)$ of the source is such
  that there exists a set~$\Xi$ and a number $\pmin > 0$ such that $f_{\S}(\s)
  \ge \pmin$ whenever $s \in \Xi$, then for any vector~$\Dv$ the mean squared
  error incurred by the communication system satisfies
  \begin{equation*}
    \msev \ge \pmin  \left( \frac{\|\Dv\|}{2}\right)^2 \int_{\Xi \cap (\Xi -
    \Dv)} Q(d(\s, \Dv)/2 \sz) d\s,
  \end{equation*}
  where $d(\s,\Dv) = \|\X(\s) - \X(\s + \Dv)\|$. (The addition of a set $A$ and
  a vector~$\x$ is defined as $A + \x = \{\vect{a} + \x : \vect{a} \in A\}$.)
\end{lemma}

\begin{remark}
  \label{rem:zivboundvec}
  The set $\Xi \cap (\Xi - \Dv)$ is the equivalent of the interval $[A,
  B-\Delta]$ in the scalar case of \lemref{zivbound}. It has the property that
  for every $\s \in \Xi \cap (\Xi - \Dv)$, $\s + \Dv \in \Xi$. To get a
  meaningful lower bound, $\Dv$~should of course be chosen such that $\Xi \cap
  (\Xi - \Dv)$ is nonempty.  Note also that when $\|\Dv\| \goesto 0$, the volume
  (or area) of $\Xi \cap (\Xi - \Dv)$ converges to the volume of~$\Xi$.
\end{remark}

\begin{proof}[Proof of \lemref{zivboundvec}]
  The proof is essentially the same as that for the scalar case
  (\lemref{zivbound}). The only difference is that the
  integrals~$\int_A^{B-\Delta}$ and~$\int_{A+\Delta}^B$ are replaced,
  respectively, with~$\int_{\Xi \cap (\Xi - \Dv)}$ and~$\int_{\Xi \cap (\Xi +
  \Dv)}$.
\end{proof}

Using \lemref{zivboundvec}, Lemmas~\ref{lem:lowerbound1}
and~\ref{lem:lowerbound2} can be rederived for the case of lattices; the
statements are in fact identical to the scalar case.

\begin{lemma}
  \label{lem:lowerbound1vec}
  For an arbitrary function~$\e(\snr) \ge 0$, the mean squared error of the
  lattice communication scheme characterized by this function satisfies
  \begin{equation*}
    \msev \in \Omega(\snr^{-n + (n-1)\e}).
  \end{equation*}
\end{lemma}

\begin{proof}
  See \appref{lbvecproofs}.
\end{proof}

\begin{lemma}
  \label{lem:lowerbound2vec}
  For an arbitrary function $\e(\snr) \ge 0$, the mean squared error of the
  lattice communication scheme characterized by this function satisfies
  \begin{equation*}
    \msev \in \Omega(\snr^{-1 + \e/2} \exp\{- c \snr^\e \}),
  \end{equation*}
  where $c > 0$ does not depend on~$\snr$.
\end{lemma}

\begin{proof}
  See \appref{lbvecproofs}.
\end{proof}

Since Lemmas~\ref{lem:lowerbound1vec} and~\ref{lem:lowerbound2vec} are
identical to Lemmas~\ref{lem:lowerbound1} and~\ref{lem:lowerbound2}, it is a
direct consequence that \thmref{scalinglb} also applies to lattice quantizers.
It is restated here for completeness.

\begin{theorem}
  \label{thm:scalinglbvec}
  For any choice of the parameter~$\beta$ (as a function of the SNR) and for any
  decoder, the mean squared error of the lattice quantizer transmission strategy
  of \secref{latticequant} satisfies
  \begin{equation*}
    \msev \in \Omega(\snr^{-n} (\log\snr)^{n-1}).
  \end{equation*}
\end{theorem}

\begin{proof}
  The proof is identical to that of \thmref{scalinglb}.
\end{proof}


\subsection{Achievability}

The MSE scaling lower bound of \thmref{scalinglbvec} is trivially achievable
with the integer lattice~$\Z$: quantization with this lattice can be performed
independently in each dimension and is equivalent to repeated application of the
scalar scheme of \secref{commscheme}. Hence \thmref{scalarachievability}
applies to lattice quantizers as well. 

\figref{sim_latticeperf} compares SDR curves achieved with a scalar quantizer
with those achieved using the 24-dimensional Leech lattice for quantization.
While the scaling at high SNR is indeed the same, confirming the results of this
section, the error due to decoding the discrete part of the signal decreases
faster if the quantizing lattice has higher dimension. The following
calculations can be used to quantify the performance.

\begin{figure}
  \begin{center}
    \input{figures/scalarvsleech.tex_t}
  \end{center}
  \caption{If the 24-dimensional Leech lattice is used for quantization, the
  scaling at large SNR remains the same but the MSE decreases quicker. This is
  useful for applications at low SNR. (Here $n = 3$.)}
  \label{fig:sim_latticeperf}
\end{figure}

Like in \secref{achievable}, a combination of ML decoder (for the lattice
points) and LMMSE decoder (for the quantization error) may be used to
estimate~$\S$.  Consider first the ML decoder. Since $\Q_i \in \Lambda/\beta$
and $\X_i = \sqrt{P/\gamma^2} \Q_i$, the ML decoder divides $\Y_i$ by
$\sqrt{P/\gamma^2}$ and then sets $\Qhv_i$ to be the closest point in
$\Lambda/\beta$. The resulting error therefore satisfies $\Qhv_i - \Q_i = \p \in
\Lambda/\beta$ if $\sqrt{\gamma^2/P} \Zv_i \in V(\p)$, where $V(\p)$ is the
Voronoi region of $\p$ (with respect to $\Lambda/\beta$). Averaging over the
noise, the average squared error is then
\begin{align*}
  \mseqiv &= \sum_{\p \in (\Lambda/\beta) \setminus \{\vz\}} \| \p \|^2
  \int_{V(\p)} \xi(\z) d\z,
\end{align*}
where $\xi(\z)$ is the pdf of $\sqrt{\gamma^2/P} \Zv_i$.
Since $\z \in V(\p)$ implies $\|\p\| \le \|\z\| + R/\beta$, where $R/\beta$ is
the covering radius of $\Lambda/\beta$, $\mseqiv$ can be upper bounded by
\begin{align*}
  \mseqiv &\le \sum_{\p \in (\Lambda/\beta) \setminus \{\vz\}} \int_{V(\p)}
  (\|\z\| + R/\beta)^2 \xi(\z) d\z \\
  &\le \int_{\R^m \setminus B(0,\rho/\beta)} (\|\z\| + R/\beta)^2 \xi(\z) d\z,
\end{align*}
where $B(0,r)$ denotes a ball of radius~$r$ around the origin and $\rho/\beta$
is the packing radius of~$\Lambda/\beta$. 

Transforming the above integral to spherical coordinates, one obtains $\|\z\| =
r$ and $d\z = r^{m-1} \Sm dr$, where $\Sm$~is the surface of a unit sphere
in~$\R^m$.  With this, 
\begin{equation}
  \label{eq:mseqivbound}
  \mseqiv \le \Sm \int_{\rho/\beta}^\infty (r + R/\beta)^2 r^{m-1} \phi(r) dr,
\end{equation}
where $\phi(r)$ is the pdf of a real Gaussian random variable of variance
$(\gamma^2/P) \szq = \gamma^2/\snr$. To get explicit bounds for finite~SNR that
depend on $R$ and $\rho$ (and thus on the particular lattice used), this
integral can be evaluated using the formula
\begin{equation}
  \label{eq:integralgamma}
  \int_\mu^\infty r^n e^{-c r^2} dr = \frac12 c^{-(n+1)/2} \Gamma\left(
  \frac{1+n}{2}, c \mu^2 \right),
\end{equation}
where $\Gamma(s, x) = \int_x^\infty t^{s-1} e^{-t} dt$ is the upper incomplete
gamma function.\footnote{This can be shown using the formula $\alpha_n(z)  =
z^{-(n+1)} \Gamma(n+1, z)$, where $\alpha_n(z) = \int_1^\infty t^n e^{-zt} dt$
\cite[\Spg5.1.5, \Spg5.1.56]{AbramowitzS1964} and using variable substitution
to find that $\int_\mu^\infty r^m e^{-cr^2} dr = \frac12 \mu^{m+1}
\alpha_{(m-1)/2} (c\mu^2)$.}

As $\snr \goesto \infty$, the bound~\eqref{eq:mseqivbound} behaves
as\footnote{An exact derivation of this statement fell victim to the delay
constraints of thesis writing. For now, the author puts his trust in
Mathematica; a rigorous proof is left to the investigative reader (Chapter~5
of~\cite{AbramowitzS1964} may be of help).}
\begin{align*}
  \mseqiv &\in O\left( \frac{1}{\beta^2} \int_{\rho/\beta}^\infty \phi(r) dr 
  \right) \\
  &= O\left( \frac{1}{\beta^2} Q\left( \frac{\sqrt{\snr}\rho}{\beta\gamma}
  \right) \right) \\
  &= O\left( \frac{1}{\beta\sqrt{\snr}} \exp\left\{-\frac{\snr \rho^2}{2 \beta^2
  \gamma^2} \right\} \right),
\end{align*}
where the last equality is because of the approximation $Q(x) \approx
\exp\{-x^2/2\}/2\pi x$ (see~\cite[\Spg 26.2.12]{AbramowitzS1964}). Letting
$\beta = \lceil \snr^{(1-\e)/2} \rceil$ yields
\begin{equation*}
  \mseqiv \in O(\snr^{-1 + \e/2} \exp\{-c \snr^\e\}),
\end{equation*}
which is the same scaling as the lower bound of~\lemref{lowerbound2vec}.

As for the LMMSE decoder, it is a straightforward consequence
of~\eqref{eq:latticeQE} that $\E[\|\hat{\vect{E}}_{n-1} - \Evc_{n-1} \|] /
\beta^{2(n-1)}$ scales as $\snr^{-n+(n-1)\e}$. Together, these two results again
confirm that \thmref{scalinglb} applies to lattices as well.


%%%%%%%%%%%%%%%%%%%%%%%%%%%%%%%%%%%%%%%%%%%%%%%%%%%%%%%%%%%%%%%%%%%%%%%%%%%%%%%%
% SECTION: GENERAL BANDWIDTH EXPANSION
%%%%%%%%%%%%%%%%%%%%%%%%%%%%%%%%%%%%%%%%%%%%%%%%%%%%%%%%%%%%%%%%%%%%%%%%%%%%%%%%

\section{General Bandwidth Expansion}\label{sec:genbwexp}

If the source~$S$ has bounded support, the transmission strategy of
\secref{commscheme} can easily be adapted to encode not one but $k$~source
symbols into $n$~channel inputs (where $k < n$). For simplicity it is first
assumed that the source support is contained in the interval $[-1/2, 1/2]$; a
generalization follows at the end of this section.


\subsection{Transmission Strategy}

The strategy used to encode the source symbols $S_1$, \dots, $S_k$ into the
channel inputs $X_1$, \dots, $X_n$ is displayed schematically on
\figvref{knencoding} for~$k=3$ and~$n=5$. 
\begin{figure}
  \begin{center}
    \input{figures/knencoding.tex_t}
  \end{center}
  \caption{Schematic display of the transmission strategy of \secref{genbwexp}
  for~$k = 3$ and~$n=5$. The boxes labeled~\textsf{Q} are uniform scalar
  quantizers with resolution~$\beta$ as described in~(\ref{eq:QEdefkn}). The
  boxes marked ``combine'' implement the operation~(\ref{eq:Qtot}). The
  part marked in thick lines corresponds exactly to the encoding scheme
  described in \secref{commscheme} to encode one source symbol into three
  channel uses; compare this with \figref{1nencoding}.}
  \label{fig:knencoding}
\end{figure}
The encoder consists of $k$~parallel scalar encoders that encode each~$S_i$ into
$n-k$~quantizer outputs $Q_{i,1}$, \dots, $Q_{i,n-k}$ and a quantization
error~$E_{i,n-k}$, just like for the case~$k=1$. The quantizer outputs at each
level are then combined into a single value $Q_{\tot,j}$
(cf.~\figref{knencoding}). More precisely, letting $i \in \{1, \dots, k\}$ be
the index of the source symbol, the encoder defines $E_{i,0} = S_i$ for
each~$i$ and computes
\begin{align}
  Q_{i,j} &= \frac{1}{\beta} \Int(\beta E_{i,j-1}) \quad\text{and} \nonumber \\
  E_{i,j} &= \beta(E_{i,j-1} - Q_{i,j}) \label{eq:QEdefkn}
\end{align}
for $j = 1$, \dots, $n-k$.
(This is exactly the same as~\eqref{eq:QEdef}, except for the addition of the
subscript~$i$.) For each quantization level~$j$, the
$k$~parallel quantizer outputs are combined into
\begin{equation}
  \label{eq:Qtot}
  Q_{\tot,j} = \sum_{i=1}^k \frac{1}{\beta^{i-1}} Q_{i,j}.
\end{equation}
Since each $Q_{i,j}$ is a multiple of $1/\beta$ and is assumed to lie in $[-1/2,
1/2]$, the mapping from the $Q_{i,j}$ into~$Q_{\tot,j}$ is invertible, and the
inverse can be recursively computed as
\begin{align}
  Q_{k,j} &= \beta^{k-1} Q_{\tot,j} \bmod 1 \nonumber \\
  Q_{k-i,j} &= \beta^{k-1-i} Q_{\tot,j} - \sum_{l = 1}^i \beta^{-l} Q_{k - i +
  l} \bmod 1, \quad\text{$i = 1$, \dots, $k-1$,} 
  \label{eq:Qtotinv}
\end{align}
where $x \bmod 1 \deq x - \Int(x)$.

Point~2 of \propref{qeproperties} from \secref{commscheme} applies here as
well, so there exists a constant~$\gamma^2$ that upper bounds the variances of
all $Q_{i,j}$ and~$E_{i,j}$ for all~$\beta$. The channel inputs are thus
\begin{align*}
  X_j &= (\sqrt{P}/\gamma) Q_{\tot,j} &&\quad\text{for $j = 1$, \dots, $n-k$ and}
  \\
  X_j &= (\sqrt{P}/\gamma) E_{j-n+k,n-k} &&\quad\text{for $j = n-k+1$, \dots,
  $n$.}
\end{align*}


\subsection{Decoder}

Just like when $k=1$, the decoder first computes separate estimates~$\Qh_{i,j}$
and~$E_{i,n-k}$ for $i = 1$, \dots, $k$ and then combines them into the estimates~$\Sh_i$. The
$\Qh_{i,j}$ are obtained using a maximum likelihood (minimum distance) estimate
of~$Q_{\tot,j}$ and then by breaking it down according to~\eqref{eq:Qtotinv}.
The estimate of $Q_{\tot,j}$ is
\begin{equation*}
  \Qh_{\tot,j} = \frac{1}{\beta^k} \arg\min_{l\in\Z}
  \left| \frac{l\sqrt P}{\gamma \beta^k} - Y_j \right|,
\end{equation*}
and the quantization errors~$E_{i,n-k}$ are estimated as
\begin{equation*}
  E_{i,n-k} = \frac{\E[E_{i,n-k} Y_{n-k+i}]}{\E[Y_{n-k+i}^2]}.
\end{equation*}
Using~\eqref{eq:unwraprec} and~\eqref{eq:Qtotinv}, the final estimate for $i =
1$, \dots, $k$ is
\begin{equation*}
  \Sh_i = \sum_{j=1}^{n-k} \frac{1}{\beta^{j-1}} \Qh_{i,j} +
  \frac{1}{\beta^{n-k}} \Eh_{i,n-k}.
\end{equation*}


\subsection{Error Analysis}

As in the case~$k=1$, the overall mean squared error can be written as
\begin{equation*}
  \mse = \sum_{j=1}^{n-k} \frac{1}{\beta^{2(j-1)}} \Err_{Q,i,j} + 
  \frac{1}{\beta^{2(n-k)}} \Err_{E,i}
\end{equation*}
where $\Err_{Q,i,j} \deq \E[(\Qh_{i,j} - Q_{i,j})^2]$ and $\Err_{E,i} \deq
\E[(\Eh_{i,n-k} - E_{i,n-k})^2]$. The behavior of the~$\Err_{E,i}$ is exactly
the same as when~$k=1$; the following lemma is therefore given without proof.

\begin{lemma}
  \label{lem:Eedecayk}
  The estimation error of the $E_{i,n-k}$ satisfies
  \begin{equation*}
    \Err_{E,i} \in O(\snr^{-1})
  \end{equation*}
  for all $i = 1$, \dots, $k$. 
\end{lemma}

The main difference to the case~$k=1$ concerns the behavior of~$\Err_{Q,i,j}$.
Because $k$~quantizer outputs are packed into a single channel input as
described by~\eqref{eq:Qtot}, $\beta^2$ is raised to the exponent~$k$ in the
following lemma (compare with \lemref{eqbound}).

\begin{lemma}
  \label{lem:eqboundk}
  For each $i = 1$, \dots, $k$ and for each $j = 1$, \dots, $n-k$,
  \begin{equation*}
    \Err_{Q,i,j} \in O(\exp\{-c \snr/\beta^{2k} \})
  \end{equation*}
  where~$c > 0$ is a constant.
\end{lemma}

\begin{proof}
  For any $i$ and~$j$, $| \Qh_{i,j} - Q_{i,j}| \ne 0$ only if $|Z_j| \ge
  \sqrt{P} / 2 \gamma \beta^k$. Furthermore, since $\Qh_{i,j}, Q_{i,j} \in
  [-1/2, 1/2)$, $|\Qh_{i,j} - Q_{i,j}| \le 1$. The error~$\Err_{Q,i,j}$ is
  therefore upper bounded by
  \begin{align*}
    \E[(\Qh_{i,j} - Q_{i,j})^2] &\le \Pr\left[|Z_j| \ge \frac{\sqrt{P}}{2 \gamma
    \beta^k} \right] \\
    &= 2 Q\left( \frac{\sqrt{\snr}}{2\gamma\beta^k} \right) \\
    &\le \exp\{-c \snr / \beta^{2k} \}
  \end{align*}
  with $c = 1/8\gamma$.
\end{proof}


Let now $\beta = \lceil \snr^{(1-\e)/2k} \rceil$. Then $\beta^{2k} \in
O(\snr^{1-\e})$, and the bound from \lemref{eqboundk} becomes
\begin{equation}
  \label{eq:qboundkepsilon}
  \Err_{Q,i,j} \in O(\exp\{-c \snr^\e\})
\end{equation}
(where $c > 0$ is not necessarily the same constant as in \lemref{eqboundk}).
Moreover, using \lemref{Eedecayk},
\begin{equation}
  \label{eq:Eeboundkepsilon}
  \frac{\Err_{E,i}}{\beta^{2(n-k)}} \in O(\snr^{\frac{n}{k} - \e
  \frac{n-k}{k}}).
\end{equation}

The final step is to choose $\e$ as a function of~$\snr$ (again just like
for~$k=1$). Let
\begin{equation*}
  \e(\snr) = \frac{\log(n \log\snr / c)}{\log\snr}.
\end{equation*}
Inserting this in~\eqref{eq:qboundkepsilon} and~\eqref{eq:Eeboundkepsilon}, 
\begin{align*}
  \Err_{Q,i,j} &\in O(\snr^{-n}) \quad\text{and} \\
  \Err_{E,i} &\in O(\snr^{-n/k} (\log\snr)^{(n-k)/k}).
\end{align*}
The overall MSE scales thus as
\begin{equation*}
  \mse \in O(\snr^{-n/k} (\log\snr)^{(n-k)/k}).
\end{equation*}


\subsection{Extension to General Sources}\label{sec:extgensources}

The assumption in \secref{genbwexp} has so far been that the support of the
source is limited to~$[-1/2,1/2]$. If a source~$S$ has support outside this
interval but its support still lies within a bounded set, just define $S' =
S/\alpha$, with $\alpha > 1$ such that $S' \in [-1/2, 1/2]$. Then use
the described scheme to transmit~$S'$ and let $\Sh = \alpha \Sh'$. The
incurred distortion is $\E[(S - \Sh)^2] = \alpha^2 \E[(S' - \Sh')^2]$; the SDR
therefore still scales in the same way as when~$S \in [-1/2, 1/2]$.

For sources with unbounded support, some form of compander must be used to bring
them into a bounded interval; this problem is left as future work. 




%%%%%%%%%%%%%%%%%%%%%%%%%%%%%%%%%%%%%%%%%%%%%%%%%%%%%%%%%%%%%%%%%%%%%%%%%%%%%%%%
% SECTION: TOWARDS A GENERAL SDR UPPER BOUND
%%%%%%%%%%%%%%%%%%%%%%%%%%%%%%%%%%%%%%%%%%%%%%%%%%%%%%%%%%%%%%%%%%%%%%%%%%%%%%%%



\section{Towards a General SDR Upper Bound}\label{sec:gensdrub}

None of the communication strategies studied in this chapter achieve the
SDR scaling of  $\snr^{n/k}$ that is achievable without a delay limit. Instead,
in each case the scaling is divided by $(\log\snr)^{(n-k)/k}$, \ie,
there is a ``penalty factor'' of $(\log\snr)^{1/k}$ for each of the $n-k$
channel inputs that carry quantized information about the source.

At first sight this may just be due to the particular nature of the
constellations used here. Because a significant fraction of channel inputs is
from discrete alphabets, a certain decrease from the theoretically optimal
performance is likely to result. On the other hand, the arguments brought
forward in \chapref{delaygauss} make the use of discrete channel inputs
plausible for a good minimal-delay code. What is more, there seems to be no
known minimal-delay code that has been shown to achieve a better scaling than
the one found here. The obvious question to ask, then, is whether the upper
bounds found in this chapter do not only apply to the type of schemes studied,
but may be of a more general nature. 


\subsubsection{Lower Bounding the Mean Squared Error for More General
Constellations}

To evaluate the best achievable SDR for a given transmission strategy, one has
to assume that the best possible decoder is used. The decoder that minimizes the
mean squared error is the posterior mean decoder, which computes $\Sh^k = \E[S^k
| Y^n]$. As mentioned earlier in this chapter, the error resulting from this
decoder is in general hard to evaluate mathematically. 

This chapter uses a trick by Ziv to lower bound the MSE by the error probability
of binary decoding, regardless of the particular decoder used, thus avoiding the
need to analyze the posterior mean decoder. However, Ziv's bound can only be
applied to a signal constellation that has some regularity: there must be a
$\Delta$ such that for any source value $s$ (in an interval with strictly
positive probability density), the source points $s$ and $s + \Delta$ are mapped
to two channel inputs whose Euclidean distance is upper bounded. For example, in
the proof of \lemref{lowerbound2} we found that when $\Delta  = \beta^{-1}$,
the distance between $X(s)$ and $X(s + \Delta)$ is always
$\sqrt{P}/\gamma\beta$.  If there does not exist such a~$\Delta$, then the MSE
cannot be bounded as in Equation~\ref{eq:avglbd} in the proof of Ziv's lemma
(see page~\pageref{eq:avglbd}). 

This does not mean that less regular constellations necessarily perform better.
It only means that Ziv's bound cannot be readily applied to constellations that
do not have the same regularity properties. One direction for future work is
thus to extend Ziv's result such that the MSE resulting from more general
constellations can be upper bounded. 


\subsubsection{Properties of a Scheme With Optimal SDR Scaling}

Assuming that there exists a minimal-delay code that achieves the optimal SDR
scaling $\snr^{n/k}$, such a code should have good minimal distance properties
in the following intuitive sense. In conventional channel coding, a code has
good minimum distance properties if every codeword is not too close to another
codeword. In joint source-channel coding with a mean squared error criterion for
a Gaussian channel, not all constellation points must be far from each other. In
fact, two constellation points may be close to each other provided that the
corresponding points in the source are close (except possibly for a set of
source points of vanishing probability). Conversely, the farther two points of
the source are from each other, the farther away the corresponding constellation
points should be. This ensures that decoding errors resulting in a large
squared error occur less frequently than those resulting in a smaller squared
error. 

In order to find bounds on the performance of minimal-delay source-channel
codes, one could thus try to characterize the ``best'' minimum distance behavior
that an arbitrary map from $\R^k$ to $\R^n$ (under a power constraint on its
image) can have. 

Essentially, a map $f: \R^k \rightarrow \R^n$ is good for joint source-channel
coding if its inverse $f^{-1}$ is ``almost surely'' \emph{continuous}. A map
from $\R^n$ to $\R^k$ is continuous, roughly speaking, if any two points close
in $\R^n$ are mapped to points close in $\R^k$. Here we say ``almost surely''
continuous, meaning that there may be pairs of points in $\R^n$ that violate the
continuity property, as long as a decoding error between these points occurs
almost never.

The preceding reflections are admittedly quite vague. The problem is that as
long as it is not clear how one can bound the MSE for an arbitrary map
(cf.~above), it is also not clear how to formally specify the properties that
such a map should have. In any case, how to formulate an exact problem and
come up with good answers is definitely a topic that should be investigated
further.

\subsubsection{Piecewise Continuous Maps from $\R$ to $\R^n$}

Of particular interest are maps that are (at least piecewise) continuous, since
they can be implemented more easily than less structured maps. We can make a few
general observations about such maps. First note that no loss of performance is
incurred by using a constant stretch. The argumentation is similar to the
``minimax considerations'' by Wozencraft and
Jacobs~\cite[p.~620]{WozencraftJ1965}: using Ziv's bound, we can obtain a lower
bound on the MSE by choosing the interval $[A,B]$ such that it contains the
source section corresponding to the smallest stretch. It is therefore the
\emph{minimum} stretch that determines the MSE scaling, and by making the
stretch constant (while preserving the shape of the constellation) this minimum
can only be increased, leading to a better MSE scaling.

On the other hand, by a similar argument as that used in the proof of
\lemref{lowerbound1}, if the stretch is constant and has value~$\ell$, then the
SDR scales at most as $\ell^2 \snr$. Thus, if the squared stretch is less than
$\snr^{n-1}$, the optimal scaling cannot be achieved. 

This implies that to achieve the optimal SDR scaling, a piecewise continuous map
must have a constant squared stretch of at least $\snr^{n-1}$. For the example
of a uniform source with support~$[-1/2, 1/2]$, the problem of designing a good
encoder then becomes the problem of how to arrange segments of a line of total
length $\snr^{(n-1)/2}$ in~$\R^n$ such that the resulting signal locus has good
minimum distance properties (as explained before) while simultaneously
satisfying the power constraint. The maps presented in this chapter are one such
way; they arrange the line as parallel segments. Another possibility is \eg\ the
Archimedes spiral~\cite{Ramstad2002}.


\subsubsection{Summary}

Summarizing the above considerations, the problem of upper bounding the
performance of minimal-delay codes can be broken down into two aspects. On one
hand, it is about finding maps with good minimum distance properties, \ie, maps
whose inverse is ``almost surely'' continuous (see above). On the other hand, a way must be
found to lower bound the MSE for general maps. As we have seen, Ziv's bound only
works if the constellation fulfills certain regularity properties. For more
general bounds, either Ziv's bound must be extended or a wholly different
bounding method must be found. 


%%%%%%%%%%%%%%%%%%%%%%%%%%%%%%%%%%%%%%%%%%%%%%%%%%%%%%%%%%%%%%%%%%%%%%%%%%%%%%%%
% SECTION: HISTORICAL REMARKS
%%%%%%%%%%%%%%%%%%%%%%%%%%%%%%%%%%%%%%%%%%%%%%%%%%%%%%%%%%%%%%%%%%%%%%%%%%%%%%%%

\section{Historical Remarks}\label{sec:mindelbwexhist}

Schemes similar to the ones proposed here have been considered before. Indeed,
one of the first schemes to transmit an analog source across two uses of a
Gaussian channel was suggested by Shannon~\cite{Shannon1949}. In fact, such
joint source-channel mappings are sometimes called Shannon mappings  or
Shannon-Kotel'nikov mappings after Shannon and V.~E.~Kotel'nikov, who studied
the problem independently of Shannon in his 1947 doctoral
dissertation~\cite{Kotelnikov1960}.  Notice the resemblance of the constellation
resulting from the communication scheme of \secref{commscheme} to Shannon's
original suggestion, both shown in Figure~\ref{fig:shannoncomparison}. The
contribution of this thesis is to specify exactly how the distance between the
segments must behave as the SNR increases in order to optimize the SDR scaling.

\begin{figure}
  \centerline{\subfloat[Shannon's original proposition.]{\input{figures/shannonline.tex_t}}
  \hfil
  \subfloat[The mapping proposed in \secref{commscheme}
  (for~$n=2$).]{\input{figures/ourconstellation.tex_t}} }% end centerline
  \caption{A minimum-delay source-channel code for $n=2$ can be visualized as a
  curve in $\R^2$ parametrized by the source. Here the mapping
  presented in \secref{commscheme} is compared to Shannon's original
  suggestion (left).} \label{fig:shannoncomparison}
\end{figure}

Wozencraft and Jacobs devoted a whole section of their 1965 textbook to
the study of minimal-delay source-channel codes as curves in $n$-dimensional
space~\cite[Section~8.2]{WozencraftJ1965}. Sakrison's
monograph~\cite{Sakrison1970} derived the optimal compander for sources with
unbounded support (such as Gaussian sources) under the low noise assumption. 

The particular communication strategy introduced in \secref{commscheme} was
first explicitly mentioned by McRae in 1971~\cite{McRae1971} as a modulation
technique for bandspreading communication.  Much of the more recent work on
minimal-delay joint source-channel codes is due to Ramstad and his coauthors
(see~\cite{Ramstad2002}, \cite{FloorR2006}, \cite{CowardR2000,CowardR2000a},
\cite{WernerssonSR2007}, \cite{HeklandFR2009}).  For $n=2$, the scheme of
\secref{commscheme} is almost identical to the HSQLC scheme by
Coward~\cite{Coward2001}, which uses a numerically optimized quantizer,
transmitter and receiver to minimize the mean-squared error (MSE) for finite
values of the SNR. Coward conjectured that the right strategy for $n > 2$ would
be to repeatedly quantize the quantization error from the previous step, which
is exactly what we do here.

Another closely related communication scheme is the \emph{shift-map} scheme due
to Chen and Wornell~\cite{ChenW1998}.  Vaishampayan and
Costa~\cite{VaishampayanC2003} showed in their analysis that it achieves an SDR
that scales as $\snr^{n-\e}$ for any $\e > 0$ if the relevant parameters are
chosen correctly as a function of the SNR. Up to rotation and a different
constellation shaping, the shift-map scheme is in fact virtually identical to
the one presented here, a fact that was pointed out recently by Taherzadeh and
Khandani~\cite{TaherzadehK2008}. In their own paper they develop a scheme that
achieves almost the same SDR scaling the scheme presented here and is in
addition robust to SNR estimation errors; their scheme, however, is based on
rearranging the digits of the binary expansion of the source and requires
greater implementation complexity.

Shamai, Verd\'u and Zamir~\cite{ShamaiVZ1998} used Wyner-Ziv coding to extend an
existing analog system with a digital code when additional bandwidth is
available. Mittal and Phamdo~\cite{MittalP2002} (see also the paper by Skoglund,
Phamdo and Alajaji~\cite{SkoglundPA2002}) split up the source into a quantized
part and a quantization error, much like we do here, but they use a
separation-based code (or ``tandem'' code) to transmit the quantization symbols.
Reznic et al.~\cite{ReznicFZ2006} use both quantization and Wyner-Ziv coding,
and their scheme includes Shamai et al.\ and Mittal \& Phamdo as extreme cases.
All three schemes, however, use long block codes for the digital phase and incur
correspondingly large delays, so they are not directly comparable with minimum
delay schemes.

The bound used to lower bound the MSE scaling in \secref{scalarlowerbound} first
occurred in a simpler form in a paper by Ziv and Zakai~\cite{ZivZ1969}. The
version used here is based on the version that Ziv developped for his 1970
paper~\cite{Ziv1970}. In that paper, Ziv found important
theoretical limitations of source-channel mappings if the encoder can depend
on the SNR only through a scaling factor.





%%%%%%%%%%%%%%%%%%%%%%%%%%%%%%%%%%%%%%%%%%%%%%%%%%%%%%%%%%%%%%%%%%%%%%%%%%%%%%%%
%                                                                              %
% CHAPTER APPENDICES                                                           %
%                                                                              %
%%%%%%%%%%%%%%%%%%%%%%%%%%%%%%%%%%%%%%%%%%%%%%%%%%%%%%%%%%%%%%%%%%%%%%%%%%%%%%%%


\begin{subappendices}



\section{Proof of \lemref{Qvarbound}}\label{app:qvarboundproof}

\begin{proof}
  Let $f(\xi)$ be the probability density function of~$E_{i-1}$. Then
  \begin{align*}
    \E[Q_i^2] &= \frac{1}{\beta^2} \int_{\R} \Int(\beta E_{i-1})^2 f(\xi) d\xi
    \\
    &= \frac{1}{\beta^2} \sum_{i\in\Z} i^2
    \int_{\frac{i-1/2}{\beta}}^{\frac{i+1/2}{\beta}} f(\xi) d\xi.
  \end{align*}
  Now, $|i| \le \beta|\xi| + 1/2$ whenever $\xi \in [(i-1/2)/\beta,
  (i+1/2)/\beta)$, so
  \begin{align*}
    \E[Q_i^2] &\le \frac{1}{\beta^2} \sum_{i\in \Z}
    \int_{\frac{i-1/2}{\beta}}^{\frac{i+1/2}{\beta}} (\beta|\xi| + 1/2)^2 
    f(\xi) d\xi \\
    &= \frac{1}{\beta^2} \int_{\R} (\beta|\xi| + 1/2)^2 f(\xi) d\xi \\
    &= \E[E_{i-1}^2] + \frac{1}{4\beta^2} + \frac{1}{\beta} \int_{\R} |\xi|
    f(\xi)d\xi.
  \end{align*}
  To bound the last integral, use the fact that $\E[|E_{i-1}|] \le \sqrt{
  \E[E_{i-1}^2]}$ to finally obtain
  \begin{equation*}
    \E[Q_i^2] \le \E[E_{i-1}^2] + \frac{\sqrt{\E[E_{i-1}^2]}}{\beta} +
    \frac{1}{4\beta^2}.
  \end{equation*}
\end{proof}
  

  \section{Proof of Ziv's Lower Bound (Lemma~\ref{lem:zivbound})}
  \label{app:zivboundproof}

  Conditioning the mean squared error on~$S$ and using the assumption on~$p_S$
  one obtains
  \begin{equation*}
    \mse \ge \pmin \int_A^B \msecond ds.
  \end{equation*}
  For $\Delta \in [0, B-A]$ one can further bound this in two ways:
  \begin{align*}
    \mse &\ge \pmin \intabd \msecond ds \\
    \mse &\ge \pmin \int_{A+\Delta}^B \msecond ds \\
    &= \pmin \intabd \msecondd ds.
  \end{align*}
  Averaging the two lower bounds yields
  \begin{equation}
    \label{eq:avglbd}
    \mse \ge \frac{\pmin}{2} \intabd \bigg( \msecond + \\
    \msecondd \bigg) ds,
  \end{equation}
  and applying Markov's inequality to the expectation terms leads to
  \begin{equation}
    \label{eq:markov1}
    \msecond \ge \dhsq \Pr[|\Sh - S| \ge \Delta/2 \mid s]
  \end{equation}
  and
  \begin{equation}
    \label{eq:markov2}
    \msecondd \ge \dhsq \Pr[|\Sh - S| \ge \Delta/2 \mid s+\Delta].
  \end{equation}

  Now suppose that the communication system in question is used for binary
  signaling. One wants to send either $s$ or $s+\Delta$; at the decoder the
  estimate~$\Sh$ is used to decide for~$s$ or $s + \Delta$ depending on which
  one $\Sh$ is closer to. When $s$ is sent, the decoder makes an error only if
  $|\Sh - s| \ge \Delta/2$; when $s + \Delta$ is sent, it makes an error only if
  $|\Sh - s - \Delta| \ge \Delta/2$. The conditional error probabilities
  therefore satisfy $\Pr[\text{error} | s] \le \Pr[|\Sh - S| \ge \Delta/2 \mid
  s]$ and $\Pr[\text{error} | s + \Delta] \le \Pr[|\Sh - S - \Delta| \ge
  \Delta/2 \mid s + \Delta]$. Applying this to~\eqref{eq:markov1}
  and~\eqref{eq:markov2} and inserting the result in~\eqref{eq:avglbd} yields
  \begin{equation}
    \label{eq:zivalmostproved}
    \mse \ge \pmin \dhsq \intabd \pe(s, \Delta) ds,
  \end{equation}
  where $\pe(s, \Delta) = \left(\Pr[\text{error}|s] + \Pr[\text{error}|s +
  \Delta] \right)/2$ is the average error probability.

  If $s$ and $s+\Delta$ are picked with equal probability and transmitted across
  $n$~parallel Gaussian channels as $\X(s)$ and $\X(s+\Delta)$, and if $d(s,
  \Delta) = \| \X(s) - \X(s + \Delta)\|$, then the error probability of the MAP
  decoder is $Q(d(s,\Delta) / 2 \sz)$, a standard result of communication theory
  (see \eg~\cite[Section~4.5]{WozencraftJ1965}). Because the MAP decoder minimizes
  the error probability, $Q(d(s,\Delta)/2\sz) \le \pe(s,\Delta)$, which, when
  inserted into~\eqref{eq:zivalmostproved}, completes the proof. \hfill\qed



  \section{Lattice Basics}\label{app:latticebasics}

  This appendix contains the very basics on lattices and lattice quantization
  needed in \secref{latticequant}. For a comprehensive treatment of
  lattices and/or quantization the reader is referred to the books by Conway and
  Sloane~\cite{ConwayS1988} and by Gersho and Gray~\cite{GershoG1992}.

  \begin{definition}
    An $n$-dimensional \emph{lattice} $\Lambda$ is a discrete subgroup of~$\R^n$
    that spans~$\R^n$. A sublattice of~$\Lambda$ is a subset $\Lambda' \subseteq
    \Lambda$ that is itself a lattice. 
  \end{definition}

  \begin{figure}
    \centerline{%
    \subfloat[Rectangular lattice.]{%
    \label{fig:rectlattice}\input{figures/rectlattice.tex_t}}%
    \hfil
    \subfloat[Hexagonal lattice.]{%
    \label{fig:hexlattice}\input{figures/hexlattice.tex_t}}%
    }
    \caption{Two lattices in~$\R^2$ and the corresponding partition into Voronoi
    regions.}
    \label{fig:r2lattices}
  \end{figure}

  \begin{example}
    \label{ex:scalarlattice}
    In~$\R$ there exists only a single lattice (up to scaling), the scalar
    lattice~$\Z$. Two examples of lattices in~$\R^2$ are displayed in
    \figref{r2lattices}.
  \end{example}

  \begin{proposition}
    \label{prop:intsublattice}
    If $\Lambda$ is a lattice and $\beta \in \NN$, then $\beta \Lambda$ is a
    sublattice of~$\Lambda$. (The set $\beta \Lambda$ is defined as $\{ \beta \x
    : \x \in \Lambda\}$.)
  \end{proposition}

  \begin{proof}
    By definition of a lattice, $\beta \Lambda \subseteq \Lambda$. Moreover, if
    $\x, \y \in \beta \Lambda$, then $\x = \beta \x'$ and $\y = \beta \y'$ with
    $\x', \y' \in \Lambda$. It follows that $\x + \y = \beta (\x' + \y') \in \beta
    \Lambda$, so $\beta\Lambda$ is itself a lattice.
  \end{proof}

  \begin{definition}
    The \emph{Voronoi region} $V(\p)$ of a lattice point $\p \in \Lambda$ is
    defined as \begin{equation*} V(\p) = \{\x \in \R^n : \|\x - \p\| \le \|\x -
      \q\|, \forall \q \in \Lambda\}, \end{equation*} \ie, $V(\p)$ is the set of
      points in $\R^n$ that at least as close to $\p$ as to any other lattice
      point.
  \end{definition}
  See \figref{r2lattices} for an illustration of the Voronoi region.

  \begin{definition}
    The \emph{packing radius} $\rho$ of a lattice is half the minimal distance
    between lattice points. Thus, $\rho$~is the largest radius of spheres that
    can be packed in~$\R^n$ by placing them at the lattice points.
  \end{definition}

  \begin{definition}
    The \emph{covering radius}~$R$ of a lattice~$\Lambda$ is the least upper
    bound for the distance from any point of~$\R^n$ to the closest point $\x \in
    \Lambda$. Thus, spheres of radius~$\rho$ around each lattice point will
    cover~$\R^n$, and no smaller radius will do.~\cite{ConwayS1988}
  \end{definition}

  The packing radius and the covering radius are illustrated on
  \figref{packingcoveringr}.
  \begin{figure}
    \begin{center}
      \input{figures/packingcoveringr.tex_t}
    \end{center}
    \caption{The packing radius~$\rho$ and the covering radius~$R$ for the
    rectangular lattice and the hexagonal lattice.}
    \label{fig:packingcoveringr}
  \end{figure}

  \begin{definition}
    \label{def:latticequant}
    A \emph{lattice quantizer} $\QL : \R^n \ra \Lambda$ maps each point
    of~$\R^n$ to the closest lattice point. Thus, for any $\x \in \R^n$, $\y \in
    \Lambda$,
    \begin{equation*}
      \| \x - \QL(\x) \| \le \|\x - \y\|.
    \end{equation*}
  \end{definition}

  \begin{remark}
    \label{rem:latticequant}
    \defref{latticequant} does not unambiguously specify $\QL(\x)$ if $\x$~lies
    on the boundary between the Voronoi regions of two adjacent lattice points.
    Since quantization is only applied to continuous-valued random variables in
    this chapter, however, the probability of this happening is zero, and this
    ambiguity can be left alone without causing any trouble.
  \end{remark}

  \begin{example}
    Let $\Lambda = \Z / \beta$ for some $\beta >0$. The associated
    quantizer~$\QL$ maps each real number to the closest multiple of $1/\beta$.
    This is exactly the quantizer used in \secref{commscheme}.
  \end{example}

  %The following simple property will be useful later.
  %\begin{proposition}
  %  Let $\Lambda = \Z / \beta$ with $\beta \in \NN$ and let $x \in [-1/2, 1/2]$.
  %  Then
  %  \begin{equation*}
  %    \QL(x) \in
  %    \begin{cases}
  %      \left\{ -\frac{\beta/2}{\beta}, -\frac{\beta/2 - 1}{\beta}, \dots, 
  %      -\frac{1}{\beta}, 0, \frac1\beta, \dots, \frac{\beta/2}{\beta} \right\}
  %      & \text{if $\beta$ is even} \\
  %      \left\{ -\frac{(\beta - 1)/2}{\beta}, -\frac{(\beta-1)/2 - 1}{\beta}, 
  %      \dots, -\frac1\beta, 0, \frac1\beta, \dots, \frac{(\beta-1)/2}{\beta}
  %      \right\}
  %      & \text{if $\beta$ is odd}.
  %    \end{cases}
  %  \end{equation*}
  %\end{proposition}
  %
  %\begin{proof}
  %  The proof is left as an exercise for the reviewers.
  %\end{proof}
  %
  The next two lemmas are useful to bound the transmit power when transmitting
  a quantized random vector.
  \begin{lemma}
    \label{lem:latquantvar}
    Let $\X$ be a random vector satisfying $\E[\|\X\|^2]
    = \sq < \infty$. Let $\Y = \QL(\X)$. Then $\E[\|\Y\|^2] \le \sq + 2R\sigma +
    R^2$, where $R$~is the covering radius of~$\Lambda$.
  \end{lemma}

  \begin{proof}
    The power of~$\Y$ is given by
    \begin{equation*}
      \E[\|\QL(\X)\|^2] = \int_{R^n} \|\QL(\x)\|^2 \fXv(\x) d\x
      = \sum_{\p \in \Lambda} \|\p\|^2 \int_{V(\p)} \fXv(\x) d\x.
    \end{equation*}
    By definition of the covering radius, $\|\p\| \le \|\x\| + R$ for all $\x
    \in V(\p)$. Thus,
    \begin{align*}
      \E[\|\QL(\X)\|^2] &\le \sum_{\p \in \Lambda} \int_{V(\p)} (\|\x\| + R)^2
      \fXv(\x) d\x \\
      &= \int_{R^n} (\|\x\| + R)^2 \fXv(\x) d\x.
    \end{align*}
    By assumption, $\int_{\R^n} \|\x\|^2 \fXv(\x) d\x = \sq$. Moreover, by the
    positivity of the variance, $\E[\xi] \le (\E[\xi^2])^{1/2}$, and so
    $\int_{\R^n} \|\x\| \fXv(\x) d\x \le \sigma$. Applying this to the above
    yields
    \begin{equation*}
      \E[\|\QL(\X)\|^2] \le \sq + 2R\sigma + R^2,
    \end{equation*}
    thus completing the proof.
  \end{proof}

  \begin{example}
    \lemref{latquantvar} can be used to derive \lemref{Qvarbound}. Indeed,
    let $X$ be a scalar zero-mean random variable of variance~$\sq$ and let
    $\Lambda = \Z / \beta$, for some $\beta > 0$. The covering radius of this
    lattice is $R = 1/2\beta$, so $\E[\QL(X)^2] \le \sq + \beta^{-1} \sigma +
    \beta^{-2}/4$. 
  \end{example}

  \begin{lemma}
    \label{lem:voronoivarbound}
    Let $\X$~be a random vector whose support is limited to the Voronoi
    region~$V(\vz)$ of a lattice~$\Lambda$. Then $\E[\|\X\|^2] \le R^2$, where
    $R$~is the covering radius of~$\Lambda$.
  \end{lemma}

  \begin{proof}
    \begin{align*}
      \E[\|\X\|^2] & = \int_{V(\vz)} \|x\|^2 \fXv(\x) d\x \\
      & \le R^2 \int_{V(\vz)} \fXv(\x) d\x = R^2.
    \end{align*}
  \end{proof}




  \section{Proof of Lemmas~\ref{lem:lowerbound1vec}
  and~\ref{lem:lowerbound2vec}}\label{app:lbvecproofs}

  The following auxiliary result and its corollary will be useful for the proofs
  to come. See \figref{voronoiintersect} for an illustration.

  \begin{lemma}
    \label{lem:subvoronoi}
    Let $\Lambda$ be an arbitrary $m$-dimensional lattice and let $\Lambda' =
    \Lambda/\beta$, where $\beta \in \NN$. (By \propref{intsublattice},
    $\Lambda$ is a sublattice of $\Lambda'$.) Then the fraction of Voronoi cells
    of $\Lambda$ that do not lie on the boundary between two Voronoi cells of
    $\Lambda'$ is bounded away from zero as $\beta$~grows large.
  \end{lemma}

  \begin{proof}
    Let $V(\Lambda)$ be a Voronoi region of $\Lambda$ and let $V(\Lambda')$ be a
    Voronoi region of~$\Lambda'$.  Consider a sphere of radius $\rho - 2R'$
    around the center of $V(\Lambda)$, where $\rho$ is the
    packing radius of $\Lambda$ and $R' = R/\beta$ is
    the covering radius of~$\Lambda'$. By definition of the packing radius, this
    sphere is completely contained within $V(\Lambda)$. Furthermore, the
    distance from the border of the sphere to the border of $V(\Lambda)$ is
    at least~$2R'$. Any Voronoi cell of $\Lambda'$ that intersects
    with the boundary of $V(\Lambda)$ lies therefore outside the sphere. 
    
    The fraction of Voronoi cells of $\Lambda'$ that do not lie on the boundary
    of $V(\Lambda)$ is therefore lower bounded by
    \begin{align*}
      \frac{(\rho - 2R')^m V^{(m)}(1)}{\Vol V(\Lambda)} &= \frac{(\rho -
      2R/\beta)^n V^{(m)}(1)}{\Vol V(\Lambda)},
    \end{align*}
    where $V^{(m)}(1)$ is the volume of the $m$-dimensional unit sphere. As beta
    grows large, this converges to $\rho V^{(n)}(1) / \Vol V(\Lambda) > 0$, and
    the proof is complete.
  \end{proof}

  \begin{corollary}
    Let $\Lambda$ be a fixed lattice and consider the sequence of lattices
    $\Lambda/\beta$, $\Lambda/\beta^2$, \dots, $\Lambda/\beta^{n-1}$, where
    $\beta \in \NN$. Then the fraction of Voronoi regions of
    $\Lambda/\beta^{n-1}$ that do not lie on the boundary between two Voronoi
    regions of any of its sublattices $\Lambda/\beta^i$ is bounded
    away from zero as $\beta$ grows large.
  \end{corollary}

  \begin{proof}
    Apply \lemref{subvoronoi} successively to the pairs $\Lambda/\beta^i$,
    $\Lambda/\beta^{i+1}$, for $i = 1$, \dots, $n-2$.
  \end{proof}

  \begin{figure}
    \centerline{
    \subfloat[$\beta = 3$]{\input{figures/hierhex3.tex_t}}
    \hfil
    \subfloat[$\beta = 4$]{\input{figures/hierhex4.tex_t}}
    }
    \caption{Illustration for \lemref{subvoronoi} and the corollary thereof.
    Note how in the right picture the fraction of small Voronoi cells that are
    not ``cut'' by a Voronoi boundary of the sublattice is larger.}
    \label{fig:voronoiintersect}
  \end{figure}

  \begin{proof}[Proof of \lemref{lowerbound1vec}]
    Consider the lattice $\Ln \deq \Lambda/\beta^{n-1}$. For $\p \in
    \Ln$, let $V(\p)$ be the Voronoi region of~$\p$.
    For some $\vxi \in V(\vz)$ define $\Dv = \Delta \vxi$ for some $\Delta \le
    1$ and define $\Vd(\p) = V(\p) \cap (V(\p) - \Dv)$. This set has the
    property that $\x + \Dv \in V(\p)$ whenever $\x \in \Vd(\p)$.\footnote{For
    reference, the sets $\Vd(\p)$ are the equivalent of the intervals
    $\I_j^{\Delta}$ in the proof of \lemref{lowerbound1}.}

    If $\p$ is such that $V(\p)$ does not lie on the boundary of the Voronoi
    region of a sublattice, then $\Vd(\p)$ defined this way has the property
    that if $\s \in \Vd(\p)$,
    \begin{align*}
      d(\s,\Dv) &= \frac{\sqrt{mP}}{\gamma} \| \Evc_{n-1}(\s) - \Evc_{n-1}(\s +
      \Dv) \| \\
      &= \frac{\sqrt{mP}}{\gamma} \| \vxi \| \beta^{n-1} \Delta.
    \end{align*}
    In other words, the lattice quantizers at each level map $\s$ and $\s +
    \Dv$ to the same lattice point.

    One can now apply \lemref{zivboundvec} and restrict the integral to the set
    \[ \Psi(\Dv) \deq \Xi \cap (\Xi - \Dv) \cap \bigcup_{\p \in \Ln'} \Vd(\p),
    \]
    where $\Ln'$ is defined to be the subset of those points in $\Ln$ whose
    Voronoi region does not lie on the boundary between Voronoi regions of a
    sublattice. By the corollary to \lemref{subvoronoi} this set has positive
    probability.

    The lower bound of \lemref{zivboundvec} is then relaxed to give
    \begin{equation*}
      \msev \ge c_1 \Delta^2 Q(c_2 \sqrt{\snr} \beta^{n-1} \Delta) 
      \int_{\Psi(\Dv)} d\s
    \end{equation*}
    for some positive constants~$c_1$ and~$c_2$. Letting
    $\Delta = 1/(\sqrt{\snr} \beta^{n-1})$ and $\beta^2 = \snr^{1-\e}$ yields
    \begin{equation*}
      \msev \ge c_3 \snr^{-n+(n-1)\e} \int_{\Psi(\Dv)} d\s,
    \end{equation*}
    with $c_3 > 0$ a constant independent of~$\snr$.

    It remains to prove the convergence of $\int_{\Psi(\Dv)} d\s$ to a constant.
    Since $\Delta$ goes to zero as $\snr \goesto \infty$, the set $\Psi(\Dv)$
    converges to the set $\Xi \cap \bigcup_{\p \in \Ln'} V(\p)$. Since both
    $\Xi$ and~$\bigcup_{\p \in \Ln'} V(\p)$ have positive volume and because the
    points of $\Ln'$ grow closer and closer as $\snr$ and thus $\beta$
    increases, the overall set also has positive volume.
  \end{proof}


  \begin{proof}[Proof of \lemref{lowerbound2vec}]
    Let $\Lambda$~be the normalized lattice used for quantization. Let $\vxi \in
    \Lambda$ be arbitrary but fixed. Let $\Dv = \beta^{-1} \vxi$. Using the same
    reasoning as in the proof of \lemref{lowerbound2}, it follows that $\Q_1(\s
    + \Dv) = \Q_1(\s) + \Dv$, and also $\Q_i(\s + \Dv) = \Q_i(\s)$ for $i = 2$,
    \ldots, $n-1$, and $\Evc_{n-1}(\s+\Dv) = \Evc_{n-1}(\s)$. Consequently,
    \begin{equation*}
      \| \X(\s) - \X(\s +\Dv) \| = \frac{\sqrt{mP}}{\gamma}\|\vxi\| \beta^{-1}.
    \end{equation*}
    According to \lemref{zivboundvec}, therefore,
    \begin{equation*}
      \msev \ge c_1 \beta^{-2} Q(c_2 \sqrt{\snr} \beta^{-1}) \int_{\Xi \cap (\Xi
      - \Dv)} d\s,
    \end{equation*}
    where $c_1$ and~$c_2$ are positive constants independent of~$\snr$.  The
    rest of the proof is essentially identical to that of
    \lemref{lowerbound2}. 
  \end{proof}

\end{subappendices}

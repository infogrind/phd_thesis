\chapter{Minimal-Delay Codes for Bandwidth Expansion}

To get a good $(1,n)$~bandwidth expansion code, a promising method is to turn
the channel into a reliable channel, thus emulating the effect of feedback. This
is the lesson from the previous chapter. Upon closer inspection, the encoder in
fact only needs to know the channel state after the first $n-1$ channel uses. If
the encoder knows the best estimate of the decoder after $n-1$ channel uses, it
can send the remaining error uncoded. 

Suppose that after $n-1$~channel uses the decoder makes a preliminary
estimate~$\Sh'$, and that the encoder knows this estimate. Let $E = \Sh' - S$ be
the corresponding estimation error and let $\seq$ be its variance. In the
$n\th$~channel use, the method of \exref{gausssingle} is used to transmit~$E$.

The decoder estimates~$\Eh$ and computes the final source estimate $\Sh = \Sh' -
\Eh$. The resulting squared error is
\begin{align*}
  \E[(\Sh - S)^2] &= \E[(S + E - \Eh - S)^2] \\
  &= \E[(\Eh - E)^2] \\
  &= \frac{\seq}{1 + \snr}.
\end{align*}
The $\sdr$ after the first $n-1$~rounds of transmission was $\sdr' = \ssq/\seq$.
The final $\sdr$ is thus
\begin{equation*}
  \sdr = \frac{\ssq}{\E[(\Sh - S)^2]} = \sdr' (1 + \snr).
\end{equation*}
Hence no matter what the strategy in the first $n-1$ channel uses was, we can
get full advantage of the last channel use using uncoded transmission.


\section{Hybrid Transmission Scheme}\label{sec:hybridscheme}

The hybrid transmission scheme at the heart of this chapter is illustrated in
\figref{qeschemegen}. It works as follows. The source~$S$ is quantized, and the
\begin{figure}[tbp]
  \begin{center}
    \input{figures/qeschemegen.tex_t}
  \end{center}
  \caption{General structure of the hybrid transmission scheme studied in this
  chapter.}
  \label{fig:qeschemegen}
\end{figure}
quantizer output $Q_1$ is scaled up to produce the first channel input~$X_1$.
Next, the resulting quantization error $E_1 = S - Q_1$ is scaled up and
quantized again. The new quantizer output is scaled to be come the next channel
input~$X_2$. This process is continued until after $n-1$~rounds the remaining
quantization error~$E_{n-1}$ is simply scaled up to produce the final channel
input~$X_n$. 

With respect to the goal of an \sdr\ that grows fast with the \snr, a first
comment can be made. If a quantizer uses a resolution that is upper
bounded with respect to the \snr, then as $\snr \goesto \infty$ the mutual
information of the corresponding channel approaches $H(Q_i)$, which is bounded.
The resulting overall mutual information is then
\begin{align*}
  I(X^n; Y^n) &\le H(Q_i) + \sum_{\substack{j = 1 \\ j \ne i}}^n I(X_i; Y_i) \\
  &\le H(Q_i) + \frac{n-1}{2} \log_2(1 + \snr).
\end{align*}
Following a similar line as in the proof of \thmref{sdrub}, it follows that the
\sdr\ can grow at best as $\snr^{n-1}$. Since the optimal scaling is $\snr^n$,
this means that one channel use is ``lost'' in terms of order.  If there is to
be any hope to get the optimal scaling, the resolution of all quantizers must
therefore grow with the \snr.

The simplest implementation of the hybrid scheme uses the same uniform quantizer
in every round. This setting is analyzed in detail in \secref{scalarquant},
where upper and lower bounds on the asymptotic performance of the scheme are
provided. 

It is straightforward to extend the scheme of \figref{qeschemegen} by
considering blocks of $m$~source symbols and using vector quantizers, in
particular lattices. This avenue is explored in \secref{latticequant}. In
particular it is shown that in the order sense, no finite dimensional lattice
can improve the asymptotic performance. 

\secref{genbwexp} discusses the case of general bandwidth expansion, where
$k$~source symbols are encoded into $n$~channel inputs with $1 < k < n$. 

\subsection{Historical Remarks}

Schemes similar to the one proposed here have been considered before. Indeed,
one of the first schemes to transmit an analog source across two uses of a
Gaussian channel was suggested by Shannon~\cite{Shannon1949}. Notice its
resemblance to the constellation studied here, shown in
Figure~\ref{fig:shannoncomparison}.
\begin{figure}
  \centerline{\subfloat[Shannon's original proposition.]{\input{figures/shannonline.tex_t}}
  \hfil
  \subfloat[Mapping proposed in this paper
  (for~$n=2$).]{\input{figures/ourconstellation.tex_t}} }% end centerline
  \caption{A minimum-delay source-channel code for $n=2$ can be visualized as a
  curve in $\R^2$ parametrized by the source. Here we compare the mapping
  presented in this chapter (right) to Shannon's original suggestion (left).}
  \label{fig:shannoncomparison}
\end{figure}

After Shannon, Wozencraft and Jacobs~\cite{WozencraftJ1965} were among the first
to study source-channel mappings as curves in $n$-dimensional space.
Ziv~\cite{Ziv1970} found important theoretical limitations of such mappings.
Much of the later work is due to Ramstad and his coauthors
(see~\cite{Ramstad2002}, \cite{FloorR2006}, \cite{CowardR2000,CowardR2000a},
\cite{WernerssonSR2007}, \cite{HeklandFR2009}). A proof that the performance of
minimal-delay codes is strictly smaller than that of codes with unrestricted
delay when $n>1$ was given in 2008 by Ingber et al.~\cite{IngberLZF2008}.

For $n=2$, the presented scheme is almost identical to the HSQLC scheme by
Coward~\cite{Coward2001}, which uses a numerically optimized quantizer,
transmitter and receiver to minimize the mean-squared error (MSE) for finite
values of the SNR. Coward correctly conjectured that the right strategy for $n >
2$ would be to repeatedly quantize the quantization error from the previous
step, which is exactly what we do here.

Another closely related communication scheme is the \emph{shift-map} scheme due
to Chen and Wornell~\cite{ChenW1998}.  Vaishampayan and
Costa~\cite{VaishampayanC2003} showed in their analysis that it achieves the
scaling exponent $n-\e$ for any $\e > 0$ if the relevant parameters are chosen
correctly as a function of the SNR. Up to rotation and a different constellation
shaping, the shift-map scheme is in fact virtually identical to the one
presented here, a fact that was pointed out recently by Taherzadeh and
Khandani~\cite{TaherzadehK2008}. In their own paper they develop a scheme that
achieves the optimal scaling exponent exactly and is in addition robust to SNR
estimation errors; their scheme, however, is based on rearranging the digits of
the binary expansion of the source and is thus quite different from the one
presented here.

Shamai, Verd\'u and Zamir~\cite{ShamaiVZ1998} used Wyner-Ziv coding to extend an
existing analog system with a digital code when additional bandwidth is
available. Mittal and Phamdo~\cite{MittalP2002} (see also the paper by Skoglund,
Phamdo and Alajaji~\cite{SkoglundPA2002}) split up the source into a quantized
part and a quantization error, much like we do here, but they use a
separation-based code (or ``tandem'' code) to transmit the quantization symbols.
Reznic et al.~\cite{ReznicFZ2006} use both quantization and Wyner-Ziv coding,
and their scheme includes Shamai et al.\ and Mittal \& Phamdo as extreme cases.
All three schemes, however, use long block codes for the digital phase and incur
correspondingly large delays, so they are not directly comparable with minimum
delay schemes.




\section{Lattice Quantization Basics}\label{sec:latquantbasics}

This section contains the very basics on lattices and lattice quantization
needed in the remainer of the chapter. For a comprehensive treatment of lattices
and/or quantization the reader is referred to the books by Conway and
Sloane~\cite{ConwayS1988} and by Gersho and Gray~\cite{GershoG1992}.

\begin{definition}
  An $n$-dimensional \emph{lattice} $\Lambda$ is a discrete set of vectors
  (points) in~$\R^n$ such that for any $\x$, $\y \in \Lambda$, $\x + \y \in
  \Lambda$. A sublattice of~$\Lambda$ is a subset $\Lambda' \subseteq \Lambda$
  that is itself a lattice. 
\end{definition}

\begin{example}
  \label{ex:scalarlattice}
  In~$\R$ there exists only a single lattice (up to scaling), the scalar
  lattice~$\Z$. Two examples of lattices in~$\R^2$ are displayed in
  \figref{r2lattices}.
\end{example}
\begin{figure}[tbp]
  \centerline{%
  \subfloat[Rectangular lattice.]{%
  \label{fig:rectlattice}\input{figures/rectlattice.tex_t}}%
  \hfil
  \subfloat[Hexagonal lattice.]{%
  \label{fig:hexlattice}\input{figures/hexlattice.tex_t}}%
  }
  \caption{Two lattices in~$\R^2$ and the corresponding partition into Voronoi
  regions. The Voronoi region of a particular point is shaded.}
  \label{fig:r2lattices}
\end{figure}
\begin{proposition}
  \label{prop:intsublattice}
  If $\Lambda$ is a lattice and $\beta \in \NN$, then $\beta \Lambda$ is a
  sublattice of~$\Lambda$. (The set $\beta \Lambda$ is defined as $\{ \beta \x :
  \x \in \Lambda\}$.)
\end{proposition}

\begin{proof}
  By definition of a lattice, $\beta \Lambda \subseteq \Lambda$. Moreover, if
  $\x, \y \in \beta \Lambda$, then $\x = \beta \x'$ and $\y = \beta \y'$ with
  $\x', \y' \in \Lambda$. It follows that $\x + \y = \beta (\x' + \y') \in \beta
  \Lambda$, so $\beta\Lambda$ is itself a lattice.
\end{proof}

\begin{definition}
  The \emph{Voronoi region} $V(\x)$ of a lattice point $\x \in \Lambda$ is
  defined as
  \begin{equation*}
    V(\x) = \{\y \in \R^n : \|\y - \x\| \le \|\y - \z\|, \forall \z \in \Lambda,
    \z \ne \x \},
  \end{equation*}
  \ie, $V(\x)$ is the set of points in $\R^n$ that are closer to $\x$ than to
  any other lattice point.
\end{definition}
See \figref{r2lattices} for an illustration of the Voronoi region.

\begin{definition}
  The \emph{packing radius} $\rho$ of a lattice is half the minimal distance
  between lattice points. Thus, $\rho$~is the largest radius of spheres that can
  be packed in~$\R^n$ by placing them at the lattice points.
\end{definition}

\begin{definition}
  The \emph{covering radius}~$R$ of a lattice~$\Lambda$ is the least upper
  bound for the distance from any point of~$\R^n$ to the closest point $\x \in
  \Lambda$. Thus, spheres of radius~$\rho$ around each lattice point will
  cover~$\R^n$, and no smaller radius will do.~\cite{ConwayS1988}
\end{definition}

The packing radius and the covering radius are illustrated on
\figref{packingcoveringr} for a hexagonal lattice.
\begin{figure}[tbp]
  \begin{center}
    \input{figures/packingcoveringr.tex_t}
  \end{center}
  \caption{The packing radius~$\rho$ and the covering radius~$R$ for the
  rectangular lattice and the hexagonal lattice.}
  \label{fig:packingcoveringr}
\end{figure}

\begin{definition}
  \label{def:latticequant}
  A \emph{lattice quantizer} $\QL : \R^n \ra \Lambda$ maps each point
  of~$\R^n$ to the closest lattice point. Thus, for any $\x \in \R^n$, $\y \in
  \Lambda$,
  \begin{equation*}
    \| \x - \QL(\x) \| \le \|\x - \y\|.
  \end{equation*}
\end{definition}

\begin{remark}
  \label{rem:latticequant}
  \defref{latticequant} does not unambiguously specify $\QL(\x)$ if $\x$~lies on
  the boundary between the Voronoi regions of two adjacent lattice points. Since
  quantization is only applied to continuous-valued random variables in this
  chapter, however, the probability of this happening is zero, and this
  ambiguity can be left alone without causing any trouble.
\end{remark}

\begin{example}
  Let $\Lambda = \Z / \beta$ for some $\beta \in \NN$. The associated
  quantizer~$\QL$ maps each real number to the closest multiple of $1/\beta$.
\end{example}

The following simple property will be useful later.
\begin{proposition}
  Let $\Lambda = \Z / \beta$ with $\beta \in \NN$ and let $x \in [-1/2, 1/2]$.
  Then
  \begin{equation*}
    \QL(x) \in
    \begin{cases}
      \left\{ -\frac{\beta/2}{\beta}, -\frac{\beta/2 - 1}{\beta}, \dots, 
      -\frac{1}{\beta}, 0, \frac1\beta, \dots, \frac{\beta/2}{\beta} \right\}
      & \text{if $\beta$ is even} \\
      \left\{ -\frac{(\beta - 1)/2}{\beta}, -\frac{(\beta-1)/2 - 1}{\beta}, 
      \dots, -\frac1\beta, 0, \frac1\beta, \dots, \frac{(\beta-1)/2}{\beta}
      \right\}
      & \text{if $\beta$ is odd}.
    \end{cases}
  \end{equation*}
\end{proposition}

\begin{proof}
  The proof is left as an exercise for the reviewers.
\end{proof}

The next lemma is useful to bound the transmit power when transmitting a
quantized random vector.
\begin{lemma}
  \label{lem:latquantvar}
  Let $\X$ be a random vector with probability density function $\fXv(\x)$,
  zero-mean, and covariance matrix $K_{\X}$ satisfying $\Tr K_{\X} = \sq <
  \infty$. Let $\Y = \QL(\X)$. Then $Tr K_{\Y} \le \sq + 2R\sigma + R^2$, 
  where $R$~is the covering radius of~$\Lambda$.
\end{lemma}

\begin{proof}
  First write $\Tr K_{\Y} = \E[\|\Y\|^2] = \E[\|\QL(\X)\|^2]$. Next,
  \begin{equation*}
    \E[\|\QL(\X)\|^2] = \int_{R^n} \|\QL(\x)\|^2 \fXv(\x) d\x
    = \sum_{\p \in \Lambda} \|\p\|^2 \int_{V(\p)} \fXv(\x) d\x.
  \end{equation*}
  By definition of the covering radius, $\|\p\| \le \|\x\| + R$ for all $\x \in
  V(\p)$. Thus,
  \begin{align*}
    \E[\|\QL(\X)\|^2] &\le \sum_{\p \in \Lambda} \int_{V(\p)} (\|\x\| + R)^2
    \fXv(\x) d\x \\
    &= \int_{R^n} (\|\x\| + R)^2 \fXv(\x) d\x.
  \end{align*}
  By assumption, $\int_{\R^n} \|\x\|^2 \fXv(\x) d\x = \sq$. Moreover, by the
  positivity of the variance, $\E[\xi] \le (\E[\xi^2])^{1/2}$, and so
  $\int_{\R^n} \|\x\| \fXv(\x) d\x \le \sigma$. Applying this to the above
  yields
  \begin{equation*}
    \E[\|\QL(\X)\|^2] \le \sq + 2R\sigma + R^2,
  \end{equation*}
  thus completing the proof.
\end{proof}

\begin{notebox}
  Todo: Add a general bound on the quantization error variance.
\end{notebox}

\begin{example}
  Let $X$ be a scalar zero-mean random variable of variance~$\sq$ and let
  $\Lambda = \Z / \beta$, for some $\beta > 0$. The covering radius of this
  lattice is $R = 1/\beta$, so $\E[\QL(X)^2] \le \sq + 2 \beta^{-1} \sigma +
  \beta^{-2}$.
\end{example}


\section{Scalar Quantizer}\label{sec:scalarquant}

The simplest version of the hybrid transmission scheme introduced in
\secref{hybridscheme} encodes one source symbol into $n$~channel inputs using a
scalar quantizer. This section derives upper and lower bounds on the \sdr\
scaling for this case. The sections that follow will progressively extend the
analysis to more general hybrid schemes.


\subsection{Transmission Strategy}\label{sec:commscheme}

To encode a single source letter $S$ into $n$~channel input symbols $X_1$,
\dots, $X_n$, we proceed as follows. Define $E_0 = S$ and recursively compute
the pairs $(Q_i, E_i)$ as
\begin{align}
  Q_i &= \QL(E_{i-1}) \nonumber \\
  E_i &= \beta (E_{i-1} - Q_i) \label{eq:QEdef}
\end{align}
for $i = 1$, \dots, $n-1$ where $\Lambda = \Z/\beta$. The parameter $\beta \in
\N$ determines the quantization resolution, the larger~$\beta$ the finer the
quantization. Following the argument at the beginning of this chapter,
$\beta$~is assumed to grow with the~\snr.

\begin{proposition}
  \label{prop:qeproperties}
  The $Q_i$ and $E_i$ satisfy the following properties:
\begin{enumerate}
  \item The map $S \mapsto (Q_1, \dots, Q_{n-1}, E_{n-1})$ is one-to-one and
    \begin{equation}
      \label{eq:unwraprec}
      S = \sum_{i=1}^{n-1} \frac{1}{\beta^{i-1}} Q_i + \frac{1}{\beta^{n-1}}
      E_{n-1}.
    \end{equation}
\end{enumerate}
\end{proposition}

\goodbreak
\begin{proof}
  \begin{enumerate}
    \item From the definition~\eqref{eq:QEdef} we have
    \begin{equation}
      \label{eq:reverserec}
      E_{i-1} = \frac{1}{\beta} E_i + Q_i.
    \end{equation}
    Repeated use of this relationship leads to the given expression for~$S$. 
  \end{enumerate}
\end{proof}

We determine the channel input symbols $X_i$ from the $Q_i$ and from $E_{n-1}$
according to 
\begin{align}
  X_i &= \sqrt{\frac{P}{\ssq + \delta}} Q_i \quad
  \text{for $i = 1$, \dots, $n-1$ and} \nonumber\\
  X_n &= \sqrt{\frac{P}{\seq}} E_{n-1},
  \label{eq:Xdef}
\end{align}
where $\seq = \var(E_{n-1})$.  Following Proposition~\ref{prop:qeproperties},
this ensures that $\E[X_i^2] \le P$ for all~$i$ and for $\beta >
\beta_0(\delta)$.  Since we are interested in the large SNR regime and since we
have defined $\beta$ to grow with~$P$, we can thus assume for the remainder that
the power constraint is satisfied. 


\subsection{Lower Bound on the Squared Error}\label{sec:scalarlowerbound}

The goal of this section is to lower bound the scaling of the mean squared error
of the transmission strategy described in Section~\ref{sec:commscheme}.

Throughout this section we assume $\beta^2 = \snr^{1-\e}$, where $\e = \e(\snr)$
is a positive function of~$\snr$. This results in no loss of generality, since
for an arbitrary positive function~$f$ we can set $\e(\snr) =
1-\log(f(\snr))/\log\snr$ to get $\beta^2(\snr) = f(\snr)$.

Note that by~\eqref{eq:QEdef}, the $Q_i$ are completely determined by~$S$.
In this section, with a slight abuse of notation, we therefore write $Q_i(s)$ to
denote the value of~$Q_i$ when $S = s$. We use $E_i(s)$ and $X_i(s)$ in a
similar manner. Furthermore, we define $\X(s) = (X_1(s), \dots, X_n(s))$.

The following result, adapted from Ziv~\cite{Ziv1970}, is a key ingredient in
the proofs of the lemmas that follow.

\begin{lemma}
  \label{lem:zivbound}
  \label{LEM:ZIVBOUND}
  Consider a communication system where a con\-tin\-u\-ous-valued source~$S$ is
  encoded into an $n$-dimensional vector $\X(S)$, sent across $n$~independent
  parallel AWGN channels with noise variance~$\szq$, and decoded at the receiver
  to produce an estimate~$\Sh$.  If the density $p_S$ of the source is such that
  there exists an interval $[A,B]$ and a number $p_{\min} > 0$ such that $p_S(s)
  \ge p_{\min}$ whenever $s \in [A,B]$, then for any $\Delta \in [0,B-A)$ the
  mean squared error incurred by the communication system satisfies
  \begin{equation}
    \label{eq:zivbound}
    \E[(\Sh - S)^2] \ge p_{\min} \left(\frac{\Delta}{2} \right)^2 
    \int_A^{B-\Delta} Q(d(s, \Delta) / 2 \sz) ds,
  \end{equation}
  where $d(s, \Delta) \deq \|\vect{X}(s) - \vect{X}(s+\Delta)\|$ and 
  \[Q(x) = \int_x^{\infty} (1/\sqrt{2\pi}) \exp\{-\xi^2/2\} d\xi.\]
\end{lemma}

\begin{proof}
  See Appendix~\ref{app:zivboundproof}.
\end{proof}

The next two lemmata provide two different asymptotic lower bounds on the
mean squared error of our transmission strategy, each of which is tighter for a
different class of~$\e$. They hold regardless of the decoder used.  (The
$\Omega$-notation is defined in Appendix~\ref{app:asymptotic}.)

\begin{lemma}
  \label{lem:lowerbound1}
  For an arbitrary function $\e(\snr) \ge 0$, the mean squared error satisfies
  \begin{equation*}
    \mse \in \Omega(\snr^{-n + (n-1)\e}).
  \end{equation*}
\end{lemma}

\begin{lemma}
  \label{lem:lowerbound2}
  For an arbitrary function $\e(\snr) \ge 0$, the mean squared error satisfies
  \begin{equation*}
    \mse \in \Omega(\snr^{-1+\e/2} \exp\{-\snr^\e/k\})
  \end{equation*}
  where $k > 0$ does not depend on~$\snr$.
\end{lemma}

\emph{Discussion:} An immediate consequence of the lemmata is that the
theoretically optimal scaling $\snr^{-n}$ is not achievable with the given
encoding strategy: by Lemma~\ref{lem:lowerbound1} this would require $\e = 0$,
but following Lemma~\ref{lem:lowerbound2} the scaling is at best $\snr^{-1}$ if
$\e = 0 $.  More generally, which one of the two lower bounds decays more slowly
and is therefore tighter depends on the scaling of~$\e(\snr)$. How to
choose~$\e(\snr)$ optimally will be the subject of Theorem~\ref{thm:scalinglb}.

\begin{proof}[Proof of Lemma~\ref{lem:lowerbound1}]
  Assume $\Delta \in [0, \beta^{-(n-1)})$ and define for $j \in \Z$
  \[ \I_j^\Delta = \left[ (j - \frac12 )\beta^{-(n-1)}, 
    (j + \frac12 ) \beta^{-(n-1)} - \Delta \right).\]
  It can be verified from~\eqref{eq:QEdef} that if $s \in \I_j^\Delta$ for
  some~$j$, the following properties hold: 1) $Q_i(s) = Q_i(s+\Delta)$ for
  $i=1$, \dots, $n-1$, and 2) $E_{n-1}(s+\Delta) - E_{n-1}(s) =
  \beta^{n-1}\Delta$.  From~\eqref{eq:Xdef} it follows that
  $s \in \I_j^\Delta$ implies $d(s, \Delta) = \sqrt{P/\seq} \beta^{n-1}\Delta$.

  We now apply Lemma~\ref{lem:zivbound} and restrict the integral to the
  set~$\psi(\Delta) \deq [A,B-\Delta) \cap \bigcup_{j\in\Z} \I_j^\Delta$. The
  lower bound is then relaxed to give
  \begin{equation*}
    \mse \ge \frac{\pmin}{4} \Delta^2 Q(\sqrt{\snr/\seq} \beta^{n-1} \Delta/2)
    \int_{\psi(\Delta)} ds.
  \end{equation*}
  Letting $\Delta = 1/(\sqrt{\snr}\beta^{n-1})$ and $\beta^2 = \snr^{1-\e}$
  yields
  \begin{equation*}
    \mse \ge \frac{\pmin}{4} \snr^{-n+(n-1)\e} Q\left(\frac{1}{2\se}\right)
    \int_{\psi(\Delta)} ds.
  \end{equation*}

  The proof is almost complete, but we still have to show that
  $\int_{\psi(\Delta)}ds$ can be bounded below by a constant for large SNR. The
  length of a single interval~$\I_j^\Delta$ is $\beta^{-(n-1)} - \Delta$. Within
  $[A,B-\Delta)$ there are $(B-A-\Delta)\beta^{n-1}$ such
  intervals. The total length of all intervals~$\I_j^\Delta$ in $[A, B-\Delta)$
  is therefore
  \[ \int_{\psi(\Delta)} ds = (B-A-\Delta)
  (1 - \beta^{n-1}\Delta), \]
  which, for the given values of~$\beta$ and~$\Delta$, 
  converges to $B-A$ for $\snr \ra \infty$ and thus can be lower bounded by a
  constant for $\snr$ greater than some $\snr_0$. With this, the proof is
  complete.
\end{proof}

\begin{proof}[Proof of Lemma~\ref{lem:lowerbound2}]
  Observe first that~\eqref{eq:QEdef} implies $Q_1(s + \beta^{-1}) = Q_1(s) +
  \beta^{-1}$ and $E_1(s + \beta^{-1}) = E_1(s)$. Since all $Q_i$ and $E_i$ for
  $i \ge 2$ are by recursion a function of $E_1$ only, $Q_i(s) = Q_i(s +
  \beta^{-1})$ for $i = 2$, \dots, $n-1$, and $E_{n-1}(s) = E_{n-1}(s +
  \beta^{-1})$. Consequently,  $X_i(s) = X_i(s + \beta^{-1})$ for all $i =
  2$, \dots, $n$. By~\eqref{eq:Xdef} and the above, the Euclidean distance
  between $\X(s)$ and~$\X(s+\beta^{-1})$ is therefore
  \begin{equation}
    \label{eq:xbetadist}
    \sqrt{\frac{P}{\ssq + \delta}} |Q_1(s) - Q_1(s+\beta^{-1})| 
    = \sqrt{\frac{P}{\ssq + \delta}} \beta^{-1}.
  \end{equation}

  We now apply Lemma~\ref{lem:zivbound} with $\Delta = \beta^{-1}$. The
  parameter $\beta$ will be chosen to increase with~$\snr$, therefore $\Delta
  \in [0, B-A)$ holds for sufficiently large~$\snr$.

  Using~\eqref{eq:xbetadist}, the resulting bound on the mean squared error is
  \begin{equation*}
    \mse \ge \frac{\pmin}{4} \beta^{-2}
    Q\left (\sqrt{\frac{\snr}{\ssq + \delta}} \frac{\beta^{-1}}{2} \right)
    (B-A-\beta^{-1}
    ).
  \end{equation*}
  Replacing $\beta^2 = \snr^{1-\e}$ and using the fact that $Q(x)$ converges to
  $\exp\{-x^2/2\}/\sqrt{2\pi}x$ for $x \goesto \infty$
  (cf.~\cite{AbramowitzS1964}) we obtain
  \begin{equation*}
    \mse \ge c \snr^{-1 + \e/2} \exp\{-\snr^\e/k\}
  \end{equation*}
  for sufficiently large~$\snr$, with $c$ and $k$ positive constants that do
  not depend on~$\snr$, thus proving the lemma.
\end{proof}

The following lemma will be used to prove Theorem~\ref{thm:scalinglb}, the main
result of this section.

\begin{lemma}
  \label{lem:epssolution}
  Define $W(x)$ to be the function that satisfies $W(x)e^{W(x)} = x$ for $x >
  0$.  This function is well defined and is sometimes called the \emph{Lambert
  $W$-function}~\textnormal{\cite{CorlessGHJK1996}}. Then for $\snr > 1$ and
  arbitrary real constants $a$, $b>0$, and $k > 0$, 
  \begin{equation}
    \label{eq:epsequation}
    \snr^{a+b\e} = \exp\{-\snr^\e/k\},
  \end{equation}
  if and only if
  \begin{equation}
    \label{eq:epssolution}
    \snr^\e = bk W(\snr^{-a/b} / bk).
  \end{equation}
\end{lemma}

\begin{proof}
  Let $\snr>1$. Since $\snr^{a+b\e}$ is strictly increasing and
  $\exp\{-\snr^\e/k\}$ is strictly decreasing in~$\e$, there is at most one
  solution to~\eqref{eq:epsequation}.  Assume now $\snr^\e$ is as
  in~\eqref{eq:epssolution}. Then
  \begin{equation*}
    \exp\{-\snr^\e/k\} = \exp\{-b W(\snr^{-a/b}/bk)\}.
  \end{equation*}
  On the other hand,
  \begin{align*}
    \snr^{a+b\e} &= \snr^a \left( bk W(\snr^{-a/b}/bk) \right)^b \\
    &= \left( W(\snr^{-a/b}/bk) / (\snr^{-a/b}/bk) \right)^b.
  \end{align*}
  By definition, $W(x)/x = e^{-W(x)}$, so the above is equal to
  \begin{equation*}
    \snr^{a+b\e} = \exp\{-bW(\snr^{-a/b}/bk)\},
  \end{equation*}
  which proves the claim.
\end{proof}

The following is the main result of this section.
\begin{theorem}
  \label{thm:scalinglb}
  For any parameter~$\beta$ and for any decoder, the mean squared error of the
  transmission strategy described in Section~\ref{sec:commscheme} satisfies
  \begin{equation*}
    \mse \in \Omega(\snr^{-n}(\log\snr)^{n-1}).
  \end{equation*}
\end{theorem}

\emph{Discussion:} The asymptotic lower bound on the mean squared error given by
the theorem coincides with the asymptotic performance achieved by the suboptimal
decoder in Section~\ref{sec:achievable}; the bound is therefore asymptotically
tight. 

%
\begin{proof}[Proof of Theorem~\ref{thm:scalinglb}]
  For notational symplicity define $l_1(\snr, \e) = \snr^{-n+(n-1)\e}$ and $l_2(\snr,\e) = \snr^{-1+\e/2}
  \exp\{-\snr^\e/k\}$. By Lemmata~\ref{lem:lowerbound1}
  and~\ref{lem:lowerbound2},
  \begin{equation*}
    \mse \in \Omega\big(\max \left( l_1(\snr,\e), l_2(\snr,\e) \right) \big).
  \end{equation*}
  The optimal parameter $\e(\snr)$ is therefore such that for
  any~$\snr$
  \begin{equation}
    \label{eq:epsmax}
    \max\left( l_1(\snr,\e), l_2(\snr,\e) \right)
  \end{equation}
  is minimized. Now for any fixed~$\snr$, $l_1(\snr,\e)$ is increasing in~$\e$,
  and $l_2(\snr,\e)$ is increasing in~$\e$ for $0 \le \e < \xi =
  \log(k/2)/\log\snr$ and decreasing in~$\e$ for $\e \ge \xi$.
  The maximum
  in~\eqref{eq:epsmax} is therefore minimized either for $\e = 0$ or for $\e \ge
  \xi$
  such that $l_1(\e) = l_2(\e)$. As we have remarked earlier, $\e = 0$ leads to
  a worse performance than that achieved in Section~\ref{sec:achievable}, and so
  this cannot be the optimal parameter. We therefore have to choose
  $\e(\snr)$ such that $l_1(\snr,\e) = l_2(\snr, \e)$.  Inserting the
  definitions of $l_1$ and $l_2$ and rearranging the terms yields
  \begin{equation*}
    \snr^{-(n-1) + (n-3/2)\e} = \exp\{-\snr^\e/k\},
  \end{equation*}
  which is of the form~\eqref{eq:epsequation} with $a = -(n-1)$ and $b = n-3/2$.
  By Lemma~\ref{lem:epssolution}, for $\snr > 1$,
  \begin{equation*}
    \snr^\e = (n-3/2)k W(\snr^{\frac{2(n-1)}{2n-3}} / ((n-3/2)k)).
  \end{equation*}
  We now use the fact that $W(x)/\log x$ converges to~$1$ for $x \ra \infty$;
  this can be shown using L'H\^opital's rule and because the derivative of
  $W(x)$ is $W(x)/[x(1 + W(x))]$ (cf.~\cite{CorlessGHJK1996}).

  For sufficiently large $\snr$, therefore, there exists a constant $c > 0$ such
  that
  \begin{equation*}
    \snr^\e \ge c(n-3/2)k \left[ \frac{2(n-1)}{2n-3}\log\snr - \log((n-3/2)k)
    \right],
  \end{equation*}
  and so $\snr^\e \in \Omega(\log\snr)$. Plugging this into the bound of
  Lemma~\ref{lem:lowerbound1} we finally obtain\footnote{If $a(x) \in
  \Omega(f(x))$ and $b(x) \in \Omega(g(x))$, then $a(x)b(x)^m \in
  \Omega(f(x)g(x)^m)$.}
  \begin{equation*}
    \mse \in \Omega(\snr^{-n}(\log\snr)^{n-1}),
  \end{equation*}
  and no choice of $\e(\snr)$ can improve this bound.
\end{proof}


\subsection{Asymptotical Achievability of Lower Bound}
\label{sec:achievable}

\subsubsection{A Suboptimal Decoder}

The $X_i$ are transmitted across the channel, producing at the channel output
the symbols
\begin{equation*}
  Y_i = X_i + Z_i, \quad i = 1, \dots, n,
\end{equation*}
where the $Z_i$ are iid Gaussian random variables of variance~$\szq$. 
To estimate $S$ from  $Y_1$, \dots, $Y_n$, the decoder first
computes separate estimates $\Qh_1$, \dots, $\Qh_{n-1}$ and $\Eh_{n-1}$, and
then combines them to obtain the final estimate~$\Sh$.  While this strategy is
suboptimal in terms of achieving a small MSE, we will see that it is good enough
to achieve the desired \sdr\ scaling.

To estimate the $Q_i$ we use a maximum likelihood (ML) decoder, which yields the
minimum distance estimate
\begin{equation}
  \label{eq:mldecoder}
  \Qh_i = \frac{1}{\beta} \arg \min_{j\in \Z} \left| \sqrt{\frac{P}{\ssq
  + \delta} } \frac{j}{\beta} - Y_i \right|.
\end{equation}
To estimate $E_{n-1}$, we use a linear minimum mean-square error (LMMSE)
estimator (see \eg~\cite[Section~8.3]{Scharf1990}), which computes
\begin{equation}
  \label{eq:lmmse}
  \Eh_{n-1} = \frac{\E[E_{n-1} Y_n]}{\E[Y_n^2]} Y_n.
\end{equation}
Finally we use the relationship~\eqref{eq:unwraprec} to obtain
\begin{equation}
  \label{eq:unwrapestim}
  \Sh = \sum_{i=1}^{n-1} \frac{1}{\beta^{i-1}} \Qh_i + \frac{1}{\beta^{n-1}}
  \Eh_{n-1}.
\end{equation}


\subsubsection{Error Analysis}

The overall MSE $\E[(S-\Sh)^2]$ can be broken up into contributions due to the
errors in decoding $Q_i$ and $E_{n-1}$ as follows. From~\eqref{eq:unwraprec}
and~\eqref{eq:unwrapestim}, the difference between $S$ and $\Sh$ is
\begin{equation*}
  S - \Sh = \sum_{i=1}^{n-1} \frac1{\beta^{i-1}} (Q_i - \Qh_i) + \frac1{\beta^{n-1}}
  (E_{n-1} - \Eh_{n-1}).
\end{equation*}
The error terms $Q_i - \Qh_i$ depend only on the noise of the respective channel
uses and are therefore independent of each other and of $E_{n-1} - \Eh_{n-1}$,
so we can write the error variance componentwise as
\begin{equation}
  \label{eq:totalerror}
  \E[(S-\Sh)^2] = \sum_{i=1}^{n-1} \frac{1}{\beta^{2(i-1)}} \Eqi +
  \frac{1}{\beta^{2(n-1)}} \Ee, 
\end{equation}
where $\Eqi \deq \E[(Q_i - \Qh_i)^2]$ and $\Ee \deq \E[(E_{n-1} -
\Eh_{n-1})^2]$.

\begin{lemma}
  \label{lem:eqbound}
  For each $i = 1$, \dots, $n-1$, 
  \begin{equation}
    \label{eq:eqidecay}
    \Eqi \in O\left(\exp\{-k \snr/\beta^2\}\right),
  \end{equation}
  where $\snr = P/\szq$ and $k > 0$~is a constant.
\end{lemma}
(The $O$-notation is defined in Appendix~\ref{app:asymptotic}.)

\begin{proof}
  Define the interval
  \begin{equation*}
    \I_j = \left[ \frac{(j - \frac12) \sqrt{P}}{\beta \sqrt{\ssq + \delta}},
    \frac{(j + \frac12) \sqrt{P}}{\beta \sqrt{\ssq + \delta}} \right).
  \end{equation*}
  According to the minimum distance decoder~\eqref{eq:mldecoder}, $\Qh_i - Q_i
  = j/\beta$ whenever $Z_i \in \I_j$.  The error $\Eqi$ satisfies thus
  \begin{align}
    \E[(Q_i - \Qh_i)^2] &= \frac{1}{\beta^2} \sum_{j \in \Z} j^2 \Pr[Z_i \in
    \I_j]  \nonumber \\
    &= \frac{2}{\beta^2} \sum_{j = 1}^\infty j^2 \Pr[Z_i \in \I_j],
    \label{eq:eqexact}
  \end{align}
  where the second equality follows from the symmetry of the distribution
  of~$Z_i$. Now,
  \begin{equation*}
    \Pr[Z_i \in \I_j] = Q\left( \frac{(j - \frac12) \sqrt \snr}{\beta \sqrt{\ssq
    + \delta }} \right) - Q\left( \frac{(j + \frac12) \sqrt{\snr}}{\beta
    \sqrt{\ssq + \delta}} \right),
  \end{equation*}
  where
  \begin{equation*}
    Q(x) = \frac{1}{\sqrt{2\pi}} \int_x^\infty e^{-\xi^2/2} d\xi,
  \end{equation*}
  which can be bounded from above for $x \ge 0$ as
  \begin{equation*}
    Q(x) \le \frac12 e^{-x^2/2}.
  \end{equation*}
  For $\beta \ge 1$ we can now bound~\eqref{eq:eqexact} as
  \begin{equation*}
    \Eqi \le \sum_{j=1}^\infty j^2 \exp\left\{ - \frac{(j - 1/2)^2
    \snr}{2\beta^2(\ssq + \delta)} \right\}.
  \end{equation*}
  Note that for $j \ge 2$, $(j - 1/2)^2 > j$.  Thus
  \begin{eqnarray}
    \Eqi &\le & \exp \left \{ - \frac{\snr}{8 \beta^2 (\ssq +\delta)} \right\}
    \nonumber \\
    & & \mbox{} + 
    \sum_{j = 2}^\infty j^2 \exp \left \{ - \frac{j \snr}{2\beta^2(\ssq
    +\delta)}
    \right\}. \label{eq:eqibound}
  \end{eqnarray}
  To bound the infinite sum we use 
  \begin{equation}
    \label{eq:geomsum}
    \sum_{j=2}^\infty j^2 p^j \le \sum_{j=1}^\infty j^2 p^j = 
    \frac{p^2+p}{(1-p)^3}
  \end{equation}
  with $p = \exp\{-\snr/2 \beta^2 (\ssq+\delta)\}$. The first term
  of~\eqref{eq:eqibound} thus dominates for large values of
  $\snr/\beta^2$ and
  \begin{equation*}
    \Eqi \le c\exp\left\{ - \frac{\snr}{8 \beta^2 (\ssq + \delta)} \right\}
  \end{equation*}
  for some~$c > 0$, which completes the proof. 
\end{proof}

\begin{lemma}
  \label{lem:eedecay}
  $\Ee \in O(\snr^{-1})$. 
\end{lemma}
\begin{proof}
  The mean-squared error that results from the LMMSE estimation~\eqref{eq:lmmse}
  is
  \begin{equation}
    \label{eq:lmmse-error}
    \Ee = \seq - \frac{(\E[E_{n-1}
    Y_n])^2}{\E[Y_n^2]}. 
  \end{equation}
  Since
  \begin{equation*}
    Y_n = X_n + Z_n = \sqrt{\frac{P}{\seq}} E_{n-1} + Z_n,
  \end{equation*}
  we have $\E[E_{n-1}Y_n] = \sqrt{P\seq}$. Moreover, $\E[Y_n^2] = \E[X^2] +
  \E[Z^2] = P+\szq$.  Inserting this into~\eqref{eq:lmmse-error} we obtain
  \begin{align*}
    \Ee &= \seq - \frac{P \seq}{P + \szq} \\
    &= \seq \left( 1 - \frac{P}{P + \szq} \right) \\
    &= \frac{\seq}{1 + \snr} \\
    & < \frac{\seq}{\snr}.
  \end{align*}
  Since $\seq$ is bounded (cf.\ Proposition~\ref{prop:qeproperties}), 
  $\Ee \in O(\snr^{-1})$ as claimed.

\end{proof}



\section{Lattice Quantizers}\label{sec:latticequant}

The scalar quantizer scheme described in the previous section can be extended
quite easily to treat blocks of $m$ source symbols and to use lattice
quantizers. 


\subsection{Lattice Basics}

For an extremely comprehensive treatment of lattices, the reader is referred to
the book by Conway and Sloane~\cite{ConwayS1988}. Hereafter, only the
definitions relevant to the problem at hand are given.



\subsection{Transmission Strategy}

The procedure to encode a vector~$\S$ of $m$~source symbols into $mn$~channel
input vectors $\X_1$, \ldots, $\X_n$ using the lattice~$\Lambda$ for
quantization is analog to that in \secref{scalarquant}, except that now all
involved quantities are $m$-dimensional vectors. First define $\Evc_0 = \S$.
For $i = 1$, \ldots, $n-1$ define
\begin{align}
  \Q_i &= \frac{1}{\beta} \QL(\beta \Evc_{i-1})  \nonumber \\
  \Evc_i &= \beta ( \Evc_{i-1} - \Q_i). \label{eq:latticeQE}
\end{align}
For the reasons given in \secref{scalarquant}, the variance of a quantized
random vector converges to the variance of the random vector as the quantization
resolution~$\beta$ grows. Moreover, as $\beta \goesto \infty$, the distribution
of the quantization errors~$\Evc_i$ approaches a uniform distribution over the
Voronoi region of the point~$\vz$~\cite{ZamirF1996}. There exists therefore
$\beta_0$ and a constant~$\gamma$ such that $\Var \Q_i \le \gamma$ and
$\Var \Evc_i \le \gamma$  for $\beta > \beta_0$ and for all $i = 1$, \ldots,
$n-1$. 

The channel inputs are now given by
\begin{align*}
  \X_i &= \frac{\sqrt{mP}}{\gamma} \Q_i \quad \text{for $i = 1$, \dots, $n-1$ and}
  \\
  \X_n &= \frac{\sqrt{mP}}{\gamma} \Evc_{n-1}.
\end{align*}
By the above argument, this ensures that $\E[\X_i^2]/m \le P$ for all~$i$.



\subsection{Error Lower Bound}

Ziv's lower bound on the mean squared error, used in \secref{scalarquant},
allows a straightforward extension to vector sources. With this, essentially the
same argument sequence as in \secref{scalarquant} can be used to lower bound the
squared error, independent of the particular decoder used.

In the following we again write the quantization resolution in terms of~$\snr$
as $\beta^2 = \snr^{1-\e(\snr)}$, without loss of generality.

\begin{lemma}
  \label{lem:zivboundvec}
  Consider a communication system where a continuous-valued source vector~$\S$
  is encoded into a vector~$\X(S)$, sent across independent parallel scalar AWGN
  channels with noise variance~$\szq$, and decoded at the receiver to produce
  an estimate~$\Shv$. If the density $f_{\S}(\s)$ of the source is such that
  there exists a set~$\Xi$ and a number $\pmin > 0$ such that $f_{\S}(\s) \ge
  \pmin$ whenever $s \in \Xi$, then for any vector~$\Dv$ the mean squared
  error incurred by the communication system satisfies
  \begin{equation*}
    \msev \ge \pmin  \left( \frac{\|\Dv\|}{2}\right)^2 \int_{\Xi \cap (\Xi -
    \Dv)} Q(d(\s, \Dv)/2 \sz) d\s,
  \end{equation*}
  where $d(\s,\Dv) = \|\X(\s) - \X(\s + \Dv)\|$. The addition of a set $A$ and a
  vector~$\x$ is defined as $A + \x = \{\vect{a} + \x : \vect{a} \in A\}$.
\end{lemma}

\begin{remark}
  \label{rem:zivboundvec}
  The set $\Xi \cap (\Xi - \Dv)$ is the equivalent of the interval $[A,
  B-\Delta]$ in the scalar case. It has the property that for every $\s \in \Xi
  \cap (\Xi - \Dv)$, $\s + \Dv \in \Xi$. To get a meaningful lower bound,
  $\Dv$~should of course be chosen such that $\Xi \cap (\Xi - \Dv)$ is nonempty.
  Note also that if $\|\Dv\| \goesto 0$, then the volume (or area) of $\Xi \cap
  (\Xi - \Dv)$ converges to the volume of~$\Xi$.
\end{remark}

\begin{proof}[Proof of \lemref{zivboundvec}]
  The proof is essentially the same as that for the scalar case
  (\lemref{zivbound}). The only difference is that the
  integrals~$\int_A^{B-\Delta}$ and~$\int_{A+\Delta}^B$ are replaced,
  respectively, with~$\int_{\Xi \cap (\Xi - \Dv)}$ and~$\int_{\Xi \cap (\Xi +
  \Dv)}$.
\end{proof}

Using \lemref{zivboundvec}, Lemmata~\ref{lem:lowerbound1}
and~\ref{lem:lowerbound2} can be rederived for the case of lattices; the
statements are in fact identical to the scalar case.

\begin{lemma}
  \label{lem:lowerbound1vec}
  For an arbitrary function~$\e(\snr)$, the mean squared error of the lattice
  quantizer transmission scheme of \secref{latticequant} satisfies
  \begin{equation*}
    \msev \in \Omega(\snr^{-n + (n-1)\e}).
  \end{equation*}
\end{lemma}

\begin{lemma}
  \label{lem:lowerbound2vec}
  For an arbitrary function $\e(\snr) \ge 0$ satisfying $\limtoinf{\snr}
  \snr^{\e(\snr)} = \infty$, the mean squared error of the strategy of
  \secref{latticequant} satisfies
  \begin{equation*}
    \msev \in \Omega(\snr^{-1 + \e/2} \exp\{- \snr^\e/k \},
  \end{equation*}
  where $k > 0$ does not depend on~$\snr$.
\end{lemma}

Since Lemmata~\ref{lem:lowerbound1vec} and~\ref{lem:lowerbound2vec} are
identical to Lemmata~\ref{lem:lowerbound1} and~\ref{lem:lowerbound2}, it is a
direct consequence that \thmref{scalinglb} also applies to lattice quantizers.
It is restated here for completeness.

\begin{theorem}
  \label{thm:scalinglbvec}
  For any parameter~$\beta$ and for any decoder, the mean squared error of the
  lattice quantizer transmission strategy of \secref{latticequant} satisfies
  \begin{equation*}
    \msev \in \Omega(\snr^{-n} (\log\snr)^{n-1}).
  \end{equation*}
\end{theorem}

\begin{proof}
  The proof is identical to that of \thmref{scalinglb}.
\end{proof}


\section{General Bandwidth Expansion}\label{sec:genbwexp}

So far, Section~\ref{sec:scalarquant} and~\ref{sec:latticequant} have treated
the encoding of one source symbol into $n$~channel input symbols and of
$m$~source symbols into $mn$~channel input symbols, respectively. This section
extends the coding scheme of \secref{scalarquant} to encode $k$~source symbols
into $n$~channel inputs, for any $k \le n$. It is shown that using a suboptimal
decoder similar to that of \secref{achievable}, the \sdr\ scales as
\textbf{[TO BE SPECIFIED]}.

For simplicity it is first assumed that the support of the source is limited to
the interval~$[-1/2, 1/2]$; the general case will be treated afterwards. 


\subsection{Transmission Strategy}

The strategy used to encode the source symbols $S_1$, \dots, $S_k$ into the
channel inputs $X_1$, \dots, $X_n$ is displayed schematically on
\figref{knencoding} for~$k=3$ and~$n=5$. 
\begin{figure}[tbp]
  \begin{center}
    \input{figures/knencoding.tex_t}
  \end{center}
  \caption{Schematic display of the transmission strategy of \secref{genbwexp}
  for~$k = 3$ and~$n=5$. The boxes labeled~\textsf{Q} are uniform scalar
  quantizers with resolution~$\beta$ as described in~\eqref{eq:QEdefkn}.}
  \label{fig:knencoding}
\end{figure}
The encoder consists of $k$~parallel scalar encoders that encode each~$S_i$ into
$n-k$~quantizer outputs $Q_{i,1}$, \dots, $Q_{i,n-k}$ and a quantization
error~$E_{j,n-k}$, just like for the case~$k=1$. The quantizer outputs at each
level are then combined into a single value $Q_{\tot,j}$
(cf.~\figref{knencoding}). More precisely, letting $i \in \{1, \dots, k\}$ be
the index of the source symbols, the encoder defines $E_{i,0} = S_i$ for
each~$i$ and computes
\begin{align}
  Q_{i,j} &= \frac{1}{\beta} \Int(\beta E_{i,j-1}) \quad\text{and} \nonumber \\
  E_{i,j} &= \beta(E_{i,j-1} - Q_{i,j}). \label{eq:QEdefkn}
\end{align}
(This is exactly the same as~\eqref{eq:QEdef}, except for the addition of the
subscript~$i$.) For each quantization level~$j = 1$, \dots, $n-k$, the
$i$~parallel quantizer outputs are combined into
\begin{equation*}
  Q_{\tot,j} = \sum_{i=1}^k \frac{1}{\beta^{i-1}} Q_{i,j}.
\end{equation*}
Since each $Q_{i,j}$ is assumed to lie in $[-1/2, 1/2]$, the mapping from the
$Q_{i,j}$ into~$Q_{\tot,j}$ is invertible, and the inverse can be recursively
specified as

\subsection{Decoder}



\subsection{Encoder and Decoder}


\subsection{Error Analysis}


\begin{subappendices}
  % [Marius] This is no longer needed, since we use the more general lemma that
  % gives the upper bound on the quantization error for all lattices.
%  \section{Proof of Lemma~\ref{lem:qvarconvergence}}
%  \label{app:lemma1proof}
%
%  Since all involved distributions are symmetric, $\E[Q_i] = 0$.  Writing $Q_i$
%  as a function of $E_{i-1}$, we have
%  \begin{equation}
%    \var(Q_i) = \E[Q_i^2] = \int_{-\infty}^\infty Q_i(\xi)^2 f(\xi) d\xi,
%    \label{eq:varqintegral}
%  \end{equation}
%  %\begin{align}
%  %  \var(Q_i) &= \E[Q_i^2] \nonumber\\
%  %  &= \int_{-\infty}^\infty Q_i(\xi)^2 f(\xi) d\xi,
%  %  \label{eq:varqintegral}
%  %\end{align}
%  where $f(\xi)$ is the pdf\footnote{probability density function} of $E_{i-1}$.
%  Now, $Q_i(\xi) = j/\beta$ whenever
%  \begin{equation*}
%    \xi \in \left[ \frac{j - 1/2}{\beta}, \frac{j + 1/2}{\beta} \right).
%  \end{equation*}
%  With this, the integral~\eqref{eq:varqintegral} becomes
%  \begin{align*}
%    \var(Q_i) &= \frac{1}{\beta^2} \sum_{j \in \Z} j^2 
%    \int_{\frac{j - 1/2}{\beta}}^{\frac{j + 1/2}{\beta}} f(\xi) d\xi \\
%    &= \sum_{j\in \Z} \left( \frac{j}{\beta} \right)^2 \left[ F\left( 
%    { \textstyle
%    \frac{j + 1/2}{\beta} }\right) - F \left( { \textstyle \frac{j - 1/2}{\beta}
%    } \right) \right],
%  \end{align*}
%  where $F(\xi)$ is the cdf\footnote{cumulative distribution function} of
%  $E_{i-1}$. As $\beta$ goes to infinity, this sum converges to a
%  Riemann-Stieltjes integral:
%  \begin{equation*}
%    \var(Q_i) \longrightarrow \int \xi^2 dF(\xi) = \var(E_{i-1}) \quad
%    \text{as $\beta \goesto \infty$.}
%  \end{equation*}
%  \qed

  
  \section{Proof of Ziv's Lower Bound (Lemma~\ref{lem:zivbound})}
  \label{app:zivboundproof}

  If we condition the mean squared error on~$S$ and use the assumption on~$p_S$
  we obtain
  \begin{equation*}
    \mse \ge \pmin \int_A^B \msecond ds.
  \end{equation*}
  For $\Delta \in [0, B-A]$ we can further bound this in two ways:
  \begin{align*}
    \mse &\ge \pmin \intabd \msecond ds \\
    \mse &\ge \pmin \int_{A+\Delta}^B \msecond ds \\
    &= \pmin \intabd \msecondd ds.
  \end{align*}
  Averaging the two lower bounds yields
  \begin{equation}
    \label{eq:avglbd}
    \mse \ge \frac{\pmin}{2} \intabd \bigg( \msecond + \\
    \msecondd \bigg) ds,
  \end{equation}
  and applying Markov's inequality to the expectation terms leads to
  \begin{equation}
    \label{eq:markov1}
    \msecond \ge \dhsq \Pr[|\Sh - S| \ge \Delta/2 \mid s]
  \end{equation}
  and
  \begin{equation}
    \label{eq:markov2}
    \msecondd \\
    \ge \dhsq \Pr[|\Sh - S - \Delta| \ge \Delta/2 \mid s+\Delta].
  \end{equation}

  Now suppose that we use the communication system in question for binary
  signaling. We want to send either $s$ or $s+\Delta$; at the decoder we use the
  estimate~$\Sh$ to decide for~$s$ or $s + \Delta$ depending on which one $\Sh$
  is closer to. When $s$ is sent, the decoder makes an error only if $|\Sh - s|
  \ge \Delta/2$; when $s + \Delta$ is sent, it makes an error only if $|\Sh - s
  - \Delta| \ge \Delta/2$. The conditional error probabilities therefore satisfy
  $\Pr[\text{error} | s] \le \Pr[|\Sh - S| \ge \Delta/2 \mid s]$ and
  $\Pr[\text{error} | s + \Delta] \le \Pr[|\Sh - S - \Delta| \ge \Delta/2 \mid s +
  \Delta]$. Applying this to~\eqref{eq:markov1} and~\eqref{eq:markov2} and
  inserting the result in~\eqref{eq:avglbd} yields
  \begin{equation}
    \label{eq:zivalmostproved}
    \mse \ge \pmin \dhsq \intabd \pe(s, \Delta) ds,
  \end{equation}
  where $\pe(s, \Delta) = \left(\Pr[\text{error}|s] + \Pr[\text{error}|s +
  \Delta] \right)/2$ is the average error probability.

  If $s$ and $s+\Delta$ are picked with equal probability and transmitted across
  $n$~parallel Gaussian channels as $\X(s)$ and $\X(s+\Delta)$, and if $d(s,
  \Delta) = \| \X(s) - \X(s + \Delta)\|$, then the error probability of the MAP
  decoder is $Q(d(s,\Delta) / 2 \sz)$, a standard result of communication theory
  (see \eg~\cite[Section~4.5]{WozencraftJ1965}). Because the MAP decoder minimizes
  the error probability, $Q(d(s,\Delta)/2\sz) \le \pe(s,\Delta)$, which, when
  inserted into~\eqref{eq:zivalmostproved}, completes the proof. \hfill\qed


  \section{Proof of Lemmata~\ref{lem:lowerbound1vec}
  and~\ref{lem:lowerbound2vec}}

  \begin{proof}[Proof of \lemref{lowerbound1vec}]
    Consider the lattice $\Ln \deq \Lambda/\beta^{n-1}$. For $\p \in
    \Ln$, let $V(\p)$ be the Voronoi region of~$\p$.

    \begin{notebox}
      Need to show that for each $\p \in \Ln$ there is a set $V'(\p) \subset
      V(\p)$ with the property that~$V'(\p)$ is not divided by any of the
      Voronoi regions of the higher level lattices.
    \end{notebox}

    For some $\vxi \in V'(\vz)$ define $\Dv = \Delta \vxi$ for some $\Delta \le
    1$ and define $\Vd'(\p) = V'(\p) \cap (V'(\p) - \Dv)$. This set has the
    property that $\y + \Dv \in V'(\p)$ whenever $\y \in \Vd'(\p)$.\footnote{For
    reference, the sets $\Vd'(\p)$ for $\p \in \Ln$ are the equivalent of the
    intervals $\I_j^{\Delta}$ in the proof of \lemref{lowerbound1}.}

    The sets $\Vd'(\p)$ defined this way have the property that if $\s \in
    \Vd'(\p)$ for some $\p \in \Ln$,
    \begin{align*}
      d(\s,\Dv) &= \frac{\sqrt{mP}}{\gamma} \| \Evc_{n-1}(\s) - \Evc_{n-1}(\s +
      \Dv) \| \\
      &= \frac{\sqrt{mP}}{\gamma} \| \vxi \| \beta^{n-1} \Delta.
    \end{align*}
    We can now apply \lemref{zivboundvec} and restrict the integral to the set
    \[ \Psi(\Dv) \deq \Xi \cap (\Xi - \Dv) \cap \bigcup_{\p \in \Ln} \Vd'(\p).
    \]
    The lower bound is then relaxed to give
    \begin{equation*}
      \msev \ge c_1 \Delta^2 Q(c_2 \sqrt{\snr} \beta^{n-1} \Delta) 
      \int_{\Psi(\Dv)} d\s
    \end{equation*}
    for some positive constants~$c_1$ and~$c_2$. Letting $\Delta =
    1/(\sqrt{\snr} \beta^{n-1})$ and $\beta^2 = \snr^{1-\e}$ yields
    \begin{equation*}
      \msev \ge c_3 \snr^{-n+(n-1)\e} \int_{\Psi(\Dv)} d\s,
    \end{equation*}
    with $c_3 > 0$ a constant independent of~\snr.

    It remains to prove the convergence of $\int_{\Psi(\Dv)} d\s$ to a constant.
    Since $\Delta$ goes to zero as $\snr \goesto \infty$, the set $\Psi(\Dv)$
    converges to the set $\Xi \cap \bigcup_{\p \in \Ln} V'(\p)$. Since both
    $\Xi$ and all~$V'(\p)$ have positive volume, the overall set also has
    positive volume.
  \end{proof}


  \begin{proof}[Proof of \lemref{lowerbound2vec}]
    Let $\Lambda$~be the normalized lattice used for quantization. Let $\vxi \in
    \Lambda$ be arbitrary but fixed. Let $\Dv = \beta^{-1} \vxi$. Using the same
    reasoning as in the proof of \lemref{lowerbound2}, it follows that $\Q_1(\s
    + \Dv) = \Q_1(\s) + \Dv$, and also $\Q_i(\s + \Dv) = \Q_i(\s)$ for $i = 2$,
    \ldots, $n-1$, and $\Evc_{n-1}(\s+\Dv) = \Evc_{n-1}(\s)$. Consequently,
    \begin{equation*}
      \| \X(\s) - \X(\s +\Dv) \| = \frac{\sqrt{mP}}{\gamma}\|\vxi\| \beta^{-1}.
    \end{equation*}
    According to \lemref{zivboundvec}, therefore,
    \begin{equation*}
      \msev \ge c_1 \beta^{-2} Q(c_2 \sqrt{\snr} \beta^{-1}) \int_{\Xi \cap (\Xi
      - \Dv)} d\s,
    \end{equation*}
    where $c_1$ and~$c_2$ are positive constants independent of~\snr.  The rest
    of the proof is essentially identical to that of
    \lemref{lowerbound2}. 
  \end{proof}

\end{subappendices}

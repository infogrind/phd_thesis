\chapter*{Zusammenfassung}

Shannons Separationstheorem liefert eine exakte Charakterisierung der Region
der erreichbaren Qualität/Kosten-Paare für die Übertragung einer
Informationsquelle über einen rauschenden Kanal. Dies aber nur für den Fall,
dass eine beliebig grosse Verzögerung in Kauf genommen wird. Ist die
tolerierbare Verzögerung begrenzt, so erhält man durch das Separationstheorem
lediglich eine äussere Schranke dieser Region; ihre genaue Form ist im
Allgemeinen nicht bekannt. 

Der erste Teil dieser Dissertation befasst sich mit der Übertragung einer
stetigwertigen Quelle über einen Kanal mit additivem weissem gaussschem
Rauschen, wenn jedes Quellensymbol separat kodiert werden muss. Es geht im
Speziellen um den Fall, dass für jedes Quellensymbol mehrere Übertragungen
durchgeführt werden können. Dazu wird ein einfaches Verfahren vorgestellt,
welches von einem bekannten Rückkopplungsverfahren inspiriert ist und
asymptotisch eine mittlere quadratische Abweichung erreicht, die so gut wie die
besten bekannten Verfahren ist. Andererseits reicht die Leistung dieses
Verfahrens nicht an die theoretischen Schranken, welche ohne Beschränkung der
Verzögerung erreicht werden können, heran. Weiter wird gezeigt, dass eine
Erweiterungen des Verfahrens auf grössere Blocklängen derselben
Leistungsbeschränkung unterliegt.

Der zweite Teil der Dissertation befasst sich mit der Übertragung einer Quelle
über einen Kanal, wenn es gilt, das \emph{Verhältnis} der Wiedergabequalität zu
den Übertragungskosten zu maximieren. Es wird eine Verbindung zwischen dem
höchstmöglichen solchen Verhältnis und existierenden informationstheoretischen
Grössen hergestellt. Zudem liefert dieser Teil exakte Bedingungen, unter welchen
ein Punkt-zu-Punkt-Kommunikationssystem das Maximum erreicht. 

Der dritte und letzte Teil greift ein etwas handfesteres Thema auf, namentlich
die Computersimulation von Übertragungsverfahren. Die These, dass sich
objektorientierte Programmierung besonders für Simulationen eignet wird
aufgestellt und durch die Präsentation eines kompletten Simulators für
Punkt-zu-Punkt-Übertragung untermauert. Dieser Simulator erlaubt es, beliebige
Kommunikationsverfahren besonders schnell zu implementieren und zu analysieren. 

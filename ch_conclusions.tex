\chapter{Conclusions and Outlook}\label{ch:conclusions}

The characterization of the achievable cost and distortion region of
point-to-point communication systems under a delay constraint is an important
unsolved problem in information theory. In this thesis we have looked at the
particular case of minimal-delay transmission with bandwidth expansion across
Gaussian channels. We have analyzed a hybrid transmission strategy based on
quantization and uncoded transmission. This strategy is by no means new; it has
appeared previously in various shapes (see the historical notes section in
\chapref{mindelbwex}). Here we have established a justification for this
strategy, inspired by the case with feedback, arguing that \emph{any}
minimal-delay bandwidth expansion scheme with uncoded components should use
uncoded transmission only in the \emph{last} channel use. Furthermore, we have
exactly characterized the scaling behavior of the signal-to-distortion ratio
(SDR) achieved when the signal-to-noise ratio (SNR) goes to infinity. 

To date there is no known minimal-delay communication scheme for the Gaussian
channel with bandwidth expansion that achieves an SDR that scales better than
the hybrid strategy presented here, namely $\snr^n / (\log\snr)^{n-1}$. Where
does the $\log\snr$ factor come from? Can it be explained by the particular
nature of our hybrid scheme, or is there something more fundamental to it? Can
the optimal SDR scaling of $\snr^n$ be achieved at all with minimal delay for $n
> 1$? These are open questions that should be investigated in the future, the
ultimate goal being to completely characterize the achievable cost and
distortion region. 

\medbreak

A communication system achieves an optimal fidelity--cost tradeoff if the
elements making up the system are properly matched, which requires that the cost
and distortion measures are related in a certain way to the statistics of the
communication system. This has been known before~\cite{GastparRV2003}.  We have
shown here that the set of cost and distortion measures for which a given
communication system is thus matched has a \emph{subset} of measures for which
the \emph{ratio} of fidelity per cost is maximized, in the sense that no
alternative encoder or decoder can increase this ratio. The set of communication
systems operating at maximal fidelity per cost is thus a subset of those
communication systems that achieve an optimal fidelity--cost tradeoff.

Whether achieving the maximal ratio of fidelity per cost is of practical value
is not conclusively clear. The challenge lies in finding a fidelity measure for
which one can quantitatively compare the performance of encoding many source
symbols at low fidelity with encoding few source symbols at high fidelity. The
conclusion to \chapref{fidelity} mentions one potential application; it might
be interesting to search for others. 


\medbreak

The chapter on simulation was written with two goals in mind. The first was to
show how object-oriented techniques are particularly helpful for simulations.
The second goal was to make the \jscsim\ simulator freely available to the
public. The author hopes that it may save at least some students from the
frustrating experience of simulator code gone chaotic. 

\medbreak

It may yet take a long time until the fundamental results of information theory
are extended to completely take into account delay constraints. Meanwhile, it is
the author's hope that the results contained in this thesis indicate some of the
avenues to follow and some of the questions to investigate. If one day a
complete answer has been found, notify him -- without delay!

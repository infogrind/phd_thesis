\chapter{Conclusions and Outlook}\label{ch:conclusions}

The characterization of the achievable cost and distortion region of
point-to-point communication systems under a delay constraint is an important
unsolved problem in information theory. In this thesis we have looked at the
particular case of minimal-delay transmission with bandwidth expansion across
Gaussian channels. We have analyzed a hybrid transmission strategy based on
quantization and uncoded transmission. This strategy is by no means new; it has
appeared previously in various shapes (see the historical notes section in
\chapref{mindelbwex}). Here we have established a justification for this
strategy, inspired by the case with feedback, arguing that \emph{any}
minimal-delay bandwidth expansion scheme with uncoded components should use
uncoded transmission only in the \emph{last} channel use. Furthermore, we have
exactly characterized the scaling behavior of the signal-to-distortion ratio
(SDR) achieved when the signal-to-noise ratio (SNR) goes to infinity. 

To date there is no known minimal-delay communication scheme for the Gaussian
channel with bandwidth expansion that achieves an SDR that scales better than
the hybrid strategy presented here, namely $\snr^n / (\log\snr)^{n-1}$. Where
does the $\log\snr$ factor come from? Can it be explained by the particular
nature of our hybrid scheme, or is there something more fundamental to it? Can
the optimal SDR scaling of $\snr^n$ be achieved at all with minimal delay for $n
> 1$? These are open questions that should be investigated in the future, the
ultimate goal being to completely characterize the achievable cost and
distortion region. 

\medbreak

A communication system achieves an optimal fidelity--cost tradeoff if the
elements making up the system are properly matched, which requires that the cost
and distortion measures are related in a certain way to the statistics of the
communication system. This has been known before.  We have shown here that the
set of cost and distortion measures for which a given communication system is
thus matched has a \emph{subset} of measures for which the \emph{ratio} of
fidelity per cost is maximized, in the sense that no alternative encoder or
decoder can increase this ratio. The set of communication systems operating at
maximal fidelity per cost is thus a subset of those communication systems that
achieve an optimal fidelity--cost tradeoff.

Why would one want to maximize the \emph{ratio} of performance per cost, rather
than to obtain the greatest fidelity for a given cost constraint or incur the
least cost for a required fidelity? It is not trivial to find an application for
which this is indeed the case. There seems to be an important practical
difference between fidelity on one hand, and rate and cost on the other hand.
Rate and cost are naturally additive quantities. Suppose a channel is used
$n$~times per second at cost $P/n$ per channel use. Then the total cost per
second is~$P$, regardless of~$n$, and the total number of bits transmitted per
second is the sum of the bits transmitted in each of the $n$~channel uses. This
gives important practical significance to capacity per unit cost: it determines
the maximum number of bits per second one can transmit for a cost constraint
\emph{per second} if the number of channel uses is a free parameter (such as in
wideband communication). 

Fidelity is not obviously additive. Our results say that for a given cost per
second we have the choice to either transmit a single source symbol at high
fidelity, or several source symbols at a lower fidelity.  Which one is
better? In many situations the two cases may not be comparable. One possible
application is oversampling with low-resolution quantization.  For example, if a
bandlimited analog source is oversampled at a sufficiently high rate, the
quantization resolution can be as low as $1$~bit per sample. Our
results may yield a tradeoff between quantization resolution and sampling
frequency; we have not investigated this issue further though.  Nevertheless, it
certainly presents an interesting opportunity for future research. 

\medbreak

The chapter on simulation was written with two goals in mind. The first was to
show how object-oriented techniques are particularly helpful for simulations.
The second goal was to make the \jscsim\ simulator freely available to the
public. It is our hope that it may save at least some students from the
frustrating experience of simulator code gone chaotic. 

\medbreak

It may yet take a long time until the fundamental results of information theory
are extended to completely take into account delay constraints. Meanwhile, it is
the author's hope that the results contained in this thesis indicate some of the
avenues to follow and some of the questions to investigate. If one day a
complete answer has been found, let him be notified -- without delay!

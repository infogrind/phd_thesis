\addcontentsline{toc}{chapter}{Acknowledgments} 
\chapter*{Acknowledgments}

The one person who has seen me through the not always smooth path from the
beginning to the end of this thesis was my advisor, Bixio Rimoldi. With my
strong tendency for distraction I certainly haven't made it any easier for him.
Nevertheless he kept me going, always coming up with suggestions about what to
do next, and applying ever so slight a pressure to get me to finally produce
some results. For all of this, and for becoming a good friend, I am extremely
grateful to him. 

% This paragraph is too convoluted - make it simpler.
Our collaboration when it came to writing papers will remain unforgotten. As
much as anyone, Bixio understood that the quality of a paper equals content
times form. It often happened that after several hours of reflection I found
what I thought was clearly the rhetorically perfect to express an
idea.  When Bixio then told me that what I wrote probably wouldn't be clear to a
reader, I was on the verge of declaring that I would withdraw my name from the
paper if this sentence didn't stay in. Of course in the end he turned out to be
right, and I have only learned from it.

This thesis has also greatly profited from the help of Emre Telatar. He gave few
hints here and there, but these hints turned out to be of great value and
directly lead to some of the main ideas set forth here. Moreover, it was always
a great pleasure to discuss with Emre such diverse subjects as books,
restaurants, religion, or typography (the list would go on). Thanks Emre!

Rüdiger Urbanke, fellow early riser, was the driving force behind many of our
more creative endeavors, in particular our home-brewed IPG movies. I am thankful
to him for this, for his unrivaled sense of humor, and for introducing me to
unicycling. 

Many thanks to the members of my thesis committee: Emre, Rüdiger, and especially
Michael Gastpar and Tor Ramstad, who took the time and effort to travel from far
away. Michael also deserves thanks for his own thesis, which was a great source
of inspiration to me from the start.

The great atmosphere during the last five years in the Information Processing
Group is due to its present and past members. Besides the ones already
mentioned, I must thank Suhas, Christina, Olivier, Nicolas, Emmanuel; our
secretaries Françoise, Muriel, and Yvonne; our sysadmin Damir, the kindest
person in the lab, for his enthusiasm to help out with even the smallest
computer problem (thanks again for your indulgence when I had my new MacBook Pro
stolen); Giovanni; former LCM members Peter, Nicu, and Tarik; the gang of
Indians: Satish, Dinkar, Sanket, Vish, and my brother from another mother
Shrinivas; furthermore Abdel, Alberto, Amin, Aslan, Christine, Cyril, Etienne,
Hamed, Mahdi, Marc, Shirin, Sibi, Soheil, Stéphane B., and Stéphane M.

Quite literally the closest people are those one shares an office with. I had
the great luck and pleasure to have Ayfer as my officemate in INR036 (baptized
by Olivier ``the office of silly talks'') for four years. Days if not weeks of
lost productivity will no doubt be attributed to our arguments about politics,
religion (a constantly recurring subject, as others can testify), various
medical conditions, as well as the French and the Turkish language (of the
latter Ayfer taught me more than just the basics). I am thankful to Ayfer for
remaining a close personal friend even though I've strained her patience enough
over the years (``Can I ask you just a quick question\dots'' comes to mind).
During my last year, Mohammad proved to be an equally pleasant officemate and a
like-minded individual (though our office hours did not coincide often). 

At the risk of forgetting someone, a long list of other people who have made my
time in Lausanne such an enjoyable one (in no particular order): Jérémie, for
the greatest movie collaborations EPFL has ever seen; Harm, co-inventor of the
early lunch and a great friend throughout; Eren, for an endless supply of movie
quotes recited with perfect accuracy; Maria and Denisa -- I can't put it in
words!; Nino, for teaching me Sicilian mafia hand gestures; Bertrand M., for
lessons in Alsatian language and culture (one day I'll have that
\emph{Choucroute Formidable}); Florence and Vojislav for inviting me to their
wedding after all (sorry I couldn't make it in the end); Klaske, for introducing
me to kayaking and for the afterwork beers at Sat; Luigi, my original
friend-friend, for the Calabrian delicacies; Claudia, for the German-Italian
tandem collaboration; Karin, for inviting us to the fanciest ball in Vienna and
to the Heurigen; Jasper, an excellent colleague for too short a time, for being
a great host in Enschede; Maya, for challenging my skepticism; Simon, for
teaching me you can't eat Weisswurst after the midday bell has rung; Gizil, for
more Turkish lessons; Daniel, for sharing his profound knowledge of all things
programming; Zafer African, for organizing the best Turkish parties; Sébastien
and Julien, for co-organizing the beer tasting; Andi, Gion, Manu, Nico, Piotr,
and Stefan for the first five years at EPFL and for remaining very good friends
to this day; also Albrecht, Ali, Bertrand N. N., Brammert, Dominique, Elena,
Emily, Emre A., Evelina, German, Hossein, Irina A., Irina B., Klaas, Lorenzo,
Mahdi, Mahdi, Masoud, Nathan, Nicolas L.-B., Olivier B., Pooya, Ruud, Shahram,
Shirin, Soheil, Sophie, Tobi, Veronica, Willem-Jan; the climbing group of the
Club Montagne; the Kayak Club Lausanne; the cities of Budapest, Firenze,
Barcelona, Gaborone, Swakopmund, Capetown, Roma, München for memorable holidays
as well as The Great Escape, Holy Cow, Lausanne Moudon, and Denis Martin for my
gastronomic and alcoholic welfare in Lausanne. 

I would never even have made it to Lausanne were it not for my family, on whose
generous support I could count since I can remember. Thanking them here will
never be enough for all they've done for me. 

Last but absolutly, definitely, categorically not least, my deepest gratitude
goes to Krem and to Ghid, for sharing some of the greatest moments during these
last five years.

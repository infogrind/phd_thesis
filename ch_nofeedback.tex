\chapter{Source-Channel Coding without Feedback}
\label{ch:nofeedback}

While Chapter~\ref{ch:qescheme} is about a very particular encoding scheme, this
chapter takes a more general look at possible strategies for bandwidth
expansion. It focuses on the AWGN channel and on the squared-error distortion. 

\section{Limitations of Linear Codes}

This section contains the proof why linear codes don't work for bandwidth
expansion. It also shows that to achieve the optimal distortion scaling
$\snr^{-n}$, the channel input distribution must be non-degenerate, which is not
possible with a linear code. 

Conclusion: to take advantage of the additional channel uses, a code must split
the source into independent parts. 


\section{Coded and Uncoded}

In this section I show that the optimum distortion can be achieved if we first
encode the source and transmit it approximately error-free in the first $n-k$
channel uses and then send the error uncoded in the last $k$ channel uses.

However, if the error from the first (coding) phase is taken into account, the
achievable mean-squared error decays at best as $\snr^{k/n-1}$. 


\section{Taking Into Account Coding Errors}

Here we look at using lattices to transmit the source across the first $n-k$
channel uses. We study how the lattice properties influence the distortion
scaling, in particular how close the Voronoi regions are to spheres. 

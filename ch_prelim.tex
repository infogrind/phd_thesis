\chapter{Preliminaries}
\label{ch:prelim}

The problem considered in this and the next chapter is the transmission of
analog sources across Gaussian channels. The present chapter provides the
necessary definitions and notation, and recalls previous results.

\section{Basic Definitions and Notation}
\label{sec:defs}

\begin{definition}[Source]
  \label{def:analogsource}
  A discrete-time analog source is specified by a probability density function
  (\pdf) $\pS(s)$ on~$\R$.
\end{definition}

\begin{definition}[Channel]
  \label{def:dtmlc}
  A discrete-time memoryless channel is specified by a conditional probability
  distribution $\pyx(y|x)$, assigning for each channel input~$x$ an output
  distribution $p_{Y|X=x}$. 
\end{definition}

\begin{definition}
  \label{def:awgn}
  An additive white Gaussian noise channel (AWGN channel) with noise
  variance~$\sq$ is a discrete-time memoryless channel whose output, conditioned
  on the input $X=x$, is a Gaussian random variable $Y$ with mean~$x$ and
  variance~$\sq$. An AWGN channel is sometimes also specified by relating the
  output $Y$ to the input $X$ through $Y = X + Z$, where $Z$ is zero-mean
  Gaussian with variance~$\sq$. 
\end{definition}

\begin{definition}[Joint source/channel code]
  \label{def:knsccode}
  A $(k,n)$~\emph{joint source/channel code} $(f,g)$ consists of an
  \emph{encoder} function $f : \R^k \ra \R^n$ and a \emph{decoder}
  function $g:\R^n \ra \R^k$. The \emph{rate} of a $(k,n)$~joint
  source/channel code is $\k = k/n$. If $\k < 1$, the code is called
  a \emph{bandwidth expansion} code; if $k > 1$, the code is called a
  \emph{bandwidth compression} code. If $\k = 1$, the code is called a
  \emph{bandwidth matched} code.
\end{definition}

\begin{definition}[Minimal-delay codes]
  \label{def:mindelcode}
  A $(k,n)$~\emph{minimal-delay} joint source\slash channel code is a
  $(k,n)$~joint source/channel code with $\gcd(k,n) = 1$. A minimal-delay joint
  source/channel code of rate~$1$ is also called a \emph{single letter} code.
\end{definition}

\begin{definition}
  \label{def:jointsccommsys}
  A $(k,n)$~\emph{source/channel communication system} consists of
  a discrete-time memoryless source~$\pS$, a discrete memoryless channel~$\pyx$,
  and a $(k,n)$~joint source/channel code~$(f,g)$ of rate~$\k = k/n$. The
  encoder maps $k$~source symbols $S^k$ into $n$~channel input symbols~$X^n =
  f(S^k)$, and the decoder maps $n$~channel output symbols $Y^n$ into $k$~source
  estimates $\Sh^k = g(Y^n)$.
\end{definition}

For an illustration of a general source/channel communication system, see
\figref{gensccommsys}.
\begin{figure}[tbp]
  \begin{center}
    \figbox{gensccommsys}
  \end{center}
  \caption{A general source/channel communication system of rate $\k = k/n$
  according to \defref{jointsccommsys}.}
  \label{fig:gensccommsys}
\end{figure}

\begin{definition}
  \label{def:PDgen}
  For a given \emph{channel input cost measure}~$\rho(x)$ and a given
  \emph{distortion measure}~$d(s,\sh)$, the \emph{expected cost}~$P$ and
  \emph{expected distortion}~$D$ incurred by a $(k,n)$ source/channel
  communication system are given by
  \begin{equation*}
    P = \frac1n \sn \E[\rho(X_i)] \quad \text{and} \quad
    D = \frac1k \sk \E[d(S_i - \Sh_i)],
  \end{equation*}
  respectively.
\end{definition}

\begin{definition}
  \label{def:sdrsnr}
  Consider a source/channel communication system whose source has
  variance~$\ssq$ and whose channel is Gaussian with noise variance~$\szq$. If
  the cost measure is the input power, \ie, $\rho(x) = x^2$, and the distortion
  measure is the squared error, \ie, $d(s,\sh) = (s - \sh)^2$, then the
  \emph{source-to-distortion ratio} \sdr\ and the \emph{signal-to-noise ratio}
  \snr\ are defined respectively as
  \begin{equation*}
    \sdr = \frac{\ssq}{D} \quad \text{and} \quad
    \snr = \frac{P}{\szq}.
  \end{equation*}
\end{definition}


\begin{definition}
  \label{def:codingscheme}
  A $(k,n)$ coding scheme consists of a function $\tilde{f} : \R^k \times \R_+
  \ra \R^n$ and a function $\tilde{g}: \R^n \times \R_+ \ra \R^k$, such that for
  every value of $\snr$, $(f(\cdot, \snr), g(\cdot, \snr))$ is a $(k,n)$~joint
  source/channel code that when used in a Gaussian communication system leads to
  $P/\szq = \snr$.
\end{definition}


\section{Fundamental Limits of Performance}
\label{sec:limits}

The most general bound on the performance of a joint source/channel coding
system is the converse to Shannon's separation theorem.

\begin{theorem}
  \label{thm:sepconverse}
  The expected distortion~$D$ and expected cost~$P$ of a source/channel
  communication system of rate~$\k$ are related by
  \begin{equation}
    \label{eq:sepconverse}
    \k R(D) \le C(P), 
  \end{equation}
  where $R(D)$ is the rate-distortion function of the source and $C(P)$ is the
  capacity-cost function of the channel.
\end{theorem}

\begin{proof}
  For a proof, see~\cite[Theorem~9.6.1]{Gallager1968}.
\end{proof}

\begin{definition}
  \label{def:optimalcode}
  A joint source/channel communication system that satisfies the bound of
  \thmref{sepconverse} with equality is said to use an \emph{optimal code}.
\end{definition}

For Gaussian channels, \thmref{sepconverse} leads to the following upper bound
on the~\sdr.
\begin{theorem}
  \label{thm:sdrub}
  If a discrete-time analog source $\pS$ of zero mean and variance~$\ssq$ is
  transmitted across an AWGN channel with noise variance~$\szq$ using a code of
  rate~$\k$, the \sdr\ is bounded by
  \begin{equation}
    \label{eq:sdrub}
    \sdr \le 2^{2 D(\pS \| \phi_{\ssq})} (1 + \snr)^{1/\k},
  \end{equation}
  where $D(\cdot \| \cdot)$ is the relative entropy or Kullback-Leibler distance
  between two distributions (see \eg~\cite{CoverT1991}) and $\phi_{\ssq}$ is a
  centered Gaussian distribution of variance~$\ssq$.
\end{theorem}

\begin{remark}
  \label{rem:perflimitgaussiansource}
  For a Gaussian source the bound of \thmref{sdrub} simplifies to $\sdr \le (1 +
  \snr)^{1/\k}$ and is simply that of \thmref{sepconverse} applied to a Gaussian
  source and channel. This bound is therefore tight, in the sense that there
  exist codes that can come arbitrarily close to equality. 
  For other sources the bound is generally not tight, but it becomes tight as
  $\snr \ra \infty$~\cite{LinderZ1994}.
\end{remark}

\begin{proof}[Proof of \thmref{sdrub}]
  The rate distortion function of an analog source of finite entropy~$h(S)$
  with squared error distortion satisfies~\cite{Shannon1959}
  \begin{equation*}
    R(D) \ge h(S) - \frac12 \log_2(2\pi e D).
  \end{equation*}
  Inserting the capacity of the Gaussian channel into~\eqref{eq:sepconverse}
  results in
  \begin{equation*}
    \k R(D) \le \frac12 \log_2(1 + \snr).
  \end{equation*}
  Combining the two above inequalities yields
  \begin{equation*}
    \frac1D \le 2^{-2h(S)} 2\pi e (1 + \snr)^{1/\k}.
  \end{equation*}
  Multiplying both sides with~$\ssq$ and noting that $0.5 \log_2(2\pi e \ssq) =
  h(\phi_{\ssq})$ leads to
  \begin{equation*}
    \sdr \le 2^{2(h(\phi_{\ssq}) - h(S))} (1 + \snr)^{1/\k}.
  \end{equation*}
  Applying the property $h(\phi_{\ssq}) - h(S) = D(\pS \|
  \phi_{\ssq})$~\cite[Theorem~8.6.5]{CoverT1991} completes the proof.
\end{proof}

\begin{remark}
  \label{rem:asympbound}
  In asymptotic terms, \thmref{sdrub} says that the \sdr\ scales at best as
  $\snr^{1/\k}$, or $\sdr \in O(\snr^{1/\k})$. (See \appref{asymptotic} for the
  definition of the $O$-notation.)
\end{remark}

\begin{definition}
  A coding scheme is called \emph{asymptotically optimal} if it satisfies $\sdr
  \in \Omega(\snr^{1/\k})$.
\end{definition}


According to the source coding and channel coding theorems, there exist
asymptotically optimal coding schemes. The codes whose existence is proved by
these theorems, however, are of unbounded blocklength (\ie, unbounded delay) and
of unbounded complexity. For minimal delay codes, bounds based on
\thmref{sepconverse} are in general not tight; the remainder of this chapter
studies certain exceptions.


\section{Optimal Minimal-Delay Codes}
\label{sec:optmindel}

\subsection{Bandwith Matched Codes}

By definition, minimal-delay bandwidth matched codes are single letter codes. A
particular example of an optimal single letter code is that of a Gaussian source
connected to a Gaussian channel, as given in the following example.
\begin{example}
  \label{ex:gausssingle}
  Let the source be zero-mean Gaussian with variance~$\ssq$ and let the channel
  be AWGN with noise variance~$\szq$. The encoder is given by $X = f(S) =
  \sqrt{P/\ssq} S$ and the decoder is given by $\Sh = g(Y) = \sqrt{P} \ssq Y /
  (P + \szq)$. Using $Y = X + Z$ it is quickly verified that 
  \begin{equation*}
    D = \E[(S - \Sh)^2] = \ssq / (1 + P/\szq),
  \end{equation*}
  which is indeed the optimal distortion according to
  \remref{perflimitgaussiansource}.
\end{example}

\exref{gausssingle} is a particular instance of a more general paradigm called
\emph{measure matching}. Measure matching provides conditions under which a code
of finite blocklength achieves the bound of \thmref{sepconverse} with equality.
For an excellent treatment of measure matching the reader is referred to
Gastpar~\cite{GastparRV2003,GastparThesis}; in the sequel only the results
relevant to this thesis are quoted.

The first relevant result from~\cite{GastparRV2003} says that any bandwidth
matched code is optimal for \emph{some} way of measuring the cost and
distortion.

\begin{theorem}
  \label{thm:tcntcbwmatch}
  A single letter code is optimal for a source/channel communication system of
  rate one, \ie, $R(D) = C(P)$, if and only if the following conditions are
  met.
  \begin{enumerate}[(i)]
    \item The cost measure~$\rho(x)$ satisfies
      \begin{equation}
        \label{eq:optcost}
        \rho(x)
        \begin{cases}
          = c_1 D(p_{Y|X=x} \| p_Y) + \beta & \text{if $p_X(x) > 0$} \\
          \ge c_1 D(p_{Y|X=x} \| p_Y) + \beta & \text{if $p_X(x) = 0$},
        \end{cases}
      \end{equation}
      where $c_1 > 0$ and $\beta$ are arbitrary real constants.

    \item The distortion measure~$d(s,\sh)$ satisfies
      \begin{equation}
        \label{eq:optdist}
        d(s,\sh) = - c_2 \log_2 \frac{p(\sh|s)}{p(\sh)} + d_0(s),
      \end{equation}
      where $c_2 > 0$ and $d_0(\cdot)$ is an arbitrary function.

    \item The code is information lossless in the sense that~$I(S;\Sh) =
      I(X;Y)$.
  \end{enumerate}
\end{theorem}

\begin{proof}
  The proof hinges on the inequality chain $R(D) \le I(S;\Sh) \le I(X;Y) \le
  C(P)$. The complete proof, found in~\cite{GastparRV2003}, shows that $I(X;Y) =
  C(P)$ if and only if condition~(i) is satisfied and that $R(D) = I(S;\Sh)$ if
  and only if condition~(ii) is satisfied. It is then clear that condition~(iii)
  is the third condition needed for $R(D) = C(P)$.
\end{proof}

\begin{remark}
  \label{rem:tcntcrug}
  The statement of \thmref{tcntcbwmatch} ignores a few special cases,
  notably the case $I(S;\Sh) = 0$ and the case $I(X;Y) = \max_P C(P)$; these are
  treated in detail in~\cite{GastparRV2003} and are not relevant here.
\end{remark}

The single letter code of \exref{gausssingle} is optimal precisely because
$\rho(x) = x^2$ satisfies condition~(i) and $d(s,\sh) = (s - \sh)^2$ satisfies
condition~(ii) for the communication system at hand. Moreover, since the encoder
and decoder are both one-to-one maps, condition~(iii) is trivially satisfied.

According to \thmref{tcntcbwmatch} there exists an infinity of optimal
single letter codes. For systems with bandwith expansion or compression,
however, optimal minimal delay codes may not exist, as the following section
argues.


\subsection{Bandwidth Expansion/Compression Codes}

If $k \ne n$ and $\gcd(k,n) = 1$ it is not known whether minimal delay codes
exist.  For the particular case of a Gaussian source and a Gaussian channel with
input power constraint and squared error distortion, Ingber et
al.~\cite{IngberLZF2008} used a result by Ziv and Zakai~\cite{ZivZ1973} to prove
that no optimal $(1,n)$ or $(k,1)$~code exists.

For codes that are not bandwidth matched, optimality is no longer only a matter
of measure matching in the sense of \thmref{tcntcbwmatch}; additional conditions
need to be fulfilled. The following theorem lists these conditions. It is
essentially Theorem~3.9 from~\cite{GastparThesis}, adapted to the case without
feedback and with a slightly more extended discussion on the ``information
lossless'' property of the code. 

\begin{theorem}
  \label{thm:tcntc1n}
  For $k \ne n$, a $(k, n)$~source/channel communication system is optimal if
  and only if the following conditions are met.
  \begin{enumerate}[(i)]
    \item
      \begin{enumerate}[(a)]
        \item The conditional distribution of the source symbols given the
          estimates can be factored as $p(s^k|\sh^k) = \pk p(s_i|\sh_i)$, and
        \item each $p(s_i|\sh_i)$ achieves the rate distortion function at the
          same average distortion.
      \end{enumerate}

    \item The estimates $\Sh^k$ form a sufficient statistic for $S^k$ given the
      outputs~$Y^n$.

    \item The encoder is information lossless in the sense that $I(S^k; Y^n) =
      I(X^n; Y^n)$. 

    \item The channel outputs $Y_1$, \ldots, $Y_n$ are mutually independent.

    \item The marginal distributions $p(x_i)$ of the channel inputs all achieve
      the capacity at the same average cost.
  \end{enumerate}
\end{theorem}

\begin{discussion}
  If $k = 1$, condition~(i)(a) of the theorem is trivially satisfied. If $n =
  1$, condition~(iv) is trivially satisfied. Furthermore, if condition~(i)(a) is
  satisfied then condition~(i)(b) is satisfied if the distortion measure is
  suitably matched. Similarly, if condition~(iv) is satisfied then achieving
  condition~(v) is a matter of matching the cost measure to the output
  distribution. 
\end{discussion}

\begin{proof}
  The proof is based on the following chain of inequalities.
  \begin{align}
    \label{eq:14step1}
    k R(D) &\le I(S^k; \Sh^k) \\
    \label{eq:14step2}
    &\le I(S^k; Y^n) \\
    \label{eq:14step3}
    &\le I(X^n; Y^n) \\
    \label{eq:14step4}
    &\le \sn I(X_i; Y_i)  \\
    \label{eq:14step5}
    &\le n C(P).
  \end{align}
  \eqref{eq:14step1} is by definition of~$R(D)$; it becomes an equality if and
  only if condition~(i) is satisfied. \eqref{eq:14step2} is the data processing
  inequality, it becomes an inequality if and only if condition~(ii) is
  satisfied. Next, \eqref{eq:14step3} is again the data processing inequality
  and it is satisfied with equality if and only if condition~(iii) is satisfied.
  Inequality~\eqref{eq:14step4} follows because the channel is memoryless and
  because conditioning can only decrease the entropy. It becomes an inequality
  if and only if condition~(iv) is satisfied. Inequality~\eqref{eq:14step5},
  finally, follows from the definition of~$C(P)$ and becomes an equality if and
  only if condition~(v) is satisfied.
\end{proof}

The following lemma gives a condition equivalent to conditions~(i)(a) and~(ii).
\begin{lemma}
  \label{lem:ssil}
  Adapt here the lemma from the feedback case to the feedbackfree case.
\end{lemma}
\begin{proof}
  \todo.
\end{proof}

\begin{remark}
  \label{rem:inflosslessenc}
  A deterministic encoder is sufficient but not necessary for condition~(iii).
  Nevertheless, any nondeterministic encoder satisfying condition~(iii) can be
  turned into a deterministic encoder that also satisfies the condition as the
  following argument shows. 

  \todo: Complete this argument.
\end{remark}

The consequence of \thmref{tcntc1n}, and the discussion following it, is this:
when $k \ne n$, finding an optimal code is no longer only a matter of matching
the cost and distortion measure. More precisely, only conditions~(i)(b) and~(v)
can be fulfilled by choosing a matching cost and distortion measure, and only
provided that the remaining conditions are satisfied. These remaining conditions
are absolute; they depend only on the statistical properties of the relevant
quantities. These findings are summarized in the following lemma.

\begin{lemma}
  \label{lem:optcodenkexist}
  For the communication system of \figref{gensccommsys} with $k \ne n$, an
  optimal minimal delay $(k,n)$~code exists if and only if there exists a
  $(k,n)$~code satisfying conditions~(i)(a) and (ii)--(iv) of \thmref{tcntc1n}.
\end{lemma}


\section{Optimal Minimal-Delay Codes with Feedback}
\label{sec:optmindelfb}

\begin{definition}
  \label{def:fbcode}
  A $(k,n)$ \emph{joint source/channel feedback code} $(f,g)$ consists of
  $n$~\emph{encoding functions} $f_i : \R^k \times \R^{i-1} \ra \R$, $i = 1$,
  \dots, $n$, and a decoding function $g: \R^n \ra \R^k$. 
\end{definition}

\begin{definition}
  \label{def:fbcommsys}
  A $(k,n)$ \emph{source/channel communication system with feedback} is a
  $(k,n)$ communication system where the encoder has noiseless feedback from the
  channel output. A $(k,n)$ feedback code computes the $i^{\text{th}}$ channel
  input as $X_i = f_i(S^k, Y^{i-1})$ and the source estimate as $\Sh^k =
  g(Y^n)$. 
\end{definition}

\figref{fbcommsys} has a schematic depiction of a feedback communication system.
\begin{figure}[tbp]
  \begin{center}
    \figbox{fbcommsys}
  \end{center}
  \caption{A general source/channel communication system with feedback.}
  \label{fig:fbcommsys}
\end{figure}

Since feedback does not increase the capacity of memoryless channels, the bound
of \thmref{sepconverse} also applies to systems with feedback. It is therefore
clear that an optimal single letter code for a system without feedback is also
optimal for a system with feedback. While there are no known optimal codes for
the feedbackfree case when $k \ne n$, there is at least one known optimal
feedback system with $k = 1$ and $n > 1$, as the following example shows.

\begin{example}
  \label{ex:gaussfb}
  In this example, a Gaussian source is transmitted across an AWGN channel, and
  the channel is used $n$~times per source symbol.  Define $E_0 = S$. In the
  $i^{\text{th}}$ channel use, the encoder produces
  \begin{equation}
    \label{eq:gaussfbxi}
    X_i = \sqrt{\frac{P}{\Var E_{i-1}}} E_{i-1}.
  \end{equation}
  Both the receiver and the sender now compute the minimum mean-squared
  error (MMSE) estimator $\Eh_{i-1}$ of $E_{i-1}$ given $Y_i$. The sender
  computes $E_i = \Eh_{i-1} - E_{i-1}$ and proceeds to the next round.

  After $n$~rounds of transmission, the receiver has $n$~estimates $\Eh_0$
  to~$\Eh_{n-1}$. Using these, it computes the final estimate~$\Sh$ as
  \begin{equation}
    \label{eq:shdecom1}
    \Sh = \Eh_0 - \Eh_1 + \Eh_2 - \cdots \pm \Eh_{n-1}.
  \end{equation}
  (The sign of the last term is $+$ if $n$~is even and $-$ if $n$~is odd.)

  To compute the overall distortion $\mse$, note that by definition $\Eh_{i-1} =
  E_{i-1} + E_i$, so \eqref{eq:shdecom1}~can be written as
  \begin{align*}
    \Sh &= E_0 + E_1 - E_1 - E_2 + E_2 + E_3 - \cdots \pm E_{n-1} \pm E_n \\
    &= E_0 \pm E_n,
  \end{align*}
  and since we have defined $E_0 = S$, we have $\mse = \E[E_n^2]$, where $E_n$
  is the remaining error after the last round of transmission.

  To compute $\E[E_n^2]$, note that since $\Eh_i$ is the MMSE estimator of
  $E_i$, the estimation error variance is given by (see
  \eg~\cite[Section~8.3]{Scharf1990})
  \begin{equation}
    \label{eq:gaussvardec}
    \E[(\Eh_i - E_i)^2] = \frac{\E[E_i^2]}{1 + P/\szq}.
  \end{equation}
  Using $\E[E_0^2] = \E[S^2] = \ssq$ and recursively applying the above, we find
  that
  \begin{equation*}
    \frac{\ssq}{\E[E_n^2]} = \frac{\ssq}{\E[(\Eh_{n-1} - E_{n-1})^2]} =
    (1 + P/\szq)^n,
  \end{equation*}
  which is indeed the largest possible \sdr\ according
  to~\remref{perflimitgaussiansource}.
\end{example}

To see why this example works, let us restate \thmref{tcntc1n} for the case with
feedback. 
\begin{theorem}
  \label{thm:tcntcfb}
  For $k \ne n$, a $(k, n)$~source/channel communication system with feedback is
  optimal if and only if the following conditions are met.
  \begin{enumerate}[(i)]
    \item
      \begin{enumerate}[(a)]
        \item The conditional distribution of the source symbols given the
          estimates can be factored as $p(s^k|\sh^k) = \pk p(s_i|\sh_i)$, and
        \item each $p(s_i|\sh_i)$ achieves the rate distortion function at the
          same average distortion.
      \end{enumerate}

    \item The estimates $\Sh^k$ form a sufficient statistic for $S^k$ given the
      outputs~$Y^n$.

    \item The encoder is information lossless in the sense that $I(S^k; Y^n) =
      I(X^n \ra Y^n)$. 

    \item The channel outputs $Y_1$, \ldots, $Y_n$ are mutually independent.

    \item The marginal distributions $p(x_i)$ of the channel inputs all achieve
      the capacity at the same average cost.
  \end{enumerate}
\end{theorem}

\begin{proof}
  The proof is essentially the same as that of \thmref{tcntc1n}, except that the
  inequality $I(X^n;Y^n) \le \sn I(X_i; Y_i)$ is replaced by $I(X^n \ra Y^n) \le
  \sn I(X_i; Y_i)$, where $I(X^n \ra Y^n)$ is the \emph{directed information}
  from~$X^n$ to~$Y^n$ (see~\cite{Massey1990,Kramer1998}). A necessary and
  sufficient condition for $I(X^n \ra Y^n) = \sn I(X_i; Y_i)$ is condition~(iv)
  of the theorem; this completes the proof.
\end{proof}

Let us now revisit Example~\ref{ex:gaussfb} and see why it achieves the optimal
distortion. We will go through the conditions of \thmref{tcntcfb}
in reverse order, starting at condition~(v). Since the source and the noise are
jointly Gaussian and the encoder and decoder are linear, all channel inputs
$X_i$ are Gaussian; since they are scaled to have variance~$P$ they all achieve
the capacity at the same average cost. Next, because the estimation error of an
MMSE estimator is uncorrelated with the observation and because in the Gaussian
case uncorrelated implies independence, $E_i$ is independent of $Y_{i-1}$ and
thus so are~$X_i$ and~$Y_i$, satisfying condition~(iv). Condition~(iii) is
satisfied because the encoder is deterministic. For conditions~(i) and~(ii),
observe that the final estimate $\Sh$ is such that $S = \Sh + E_n$, where $E_n$
is independent of $\Sh$ and of $Y^n$. Given $\Sh$, $S$ is therefore independent
of~$Y^n$, which makes $\Sh$ a sufficient statistic, satisfying condition~(ii).
Moreover, the relationship between $S$ and $\Sh$ is exactly the one leading to
the distribution that achieves the rate distortion function (see
\eg~\cite{CoverT1991}), fulfilling condition~(i)(b). Condition~(i)(a) is
trivially satisfied since~$k=1$.

It appears as though this example works only because of the particular
properties of the Gaussian distribution: preservation of distribution under
linear transformation, linear MMSE decoder, equivalence of uncorrelatedness and
independence, and so on. One would assume, therefore, that this example does not
give any indication about how to use the feedback for general sources and
channels. As the next section shows, though, conditions~(iii)--(v) of
\thmref{tcntcfb} can be achieved with minimal complexity for any
channel if perfect feedback is available.


\subsection{Posterior Matching}

We have seen in Example~\ref{ex:gaussfb} how a simple transmission scheme can
achieve the optimal distortion using noiseless feedback. The discussion in the
previous section showed how the particular properties of the Gaussian
distribution helped in achieving this. In the sequel we show that
conditions~(iii) to~(v) of \thmref{tcntcfb} can be satisfied for arbitrary
channels. The underlying idea, called \emph{posterior matching}, was used by
Shayevitz and Feder in~2007~\cite{ShayevitzF2007,ShayevitzF2008} to generalize
the capacity achieving channel coding schemes of Schalkwijk and
Kailath~\cite{SchalkwijkK1966} and Horstein~\cite{Horstein1963} to arbitrary
channels with feedback.

Before continuing we prove some properties of cumulative distribution functions
(\cdf s).

\begin{lemma}
  \label{lem:cdfunif}
  Let $X$ be a continuous random variable with density $f(x)$ and \cdf\ $F_X$,
  \ie,
  \begin{equation*}
    F_X(x) = \Pr[X \le x].
  \end{equation*}
  Then the random variable $F_X(X)$ is uniformly distributed on~$[0,1]$.
\end{lemma}

\begin{proof}
  Let $Y = F_X(X)$. Then
  \begin{align*}
    \Pr[Y \le y] &= \Pr[F_X(X) \le y] \\
    &= \Pr[X \le F_X^{-1}(y)] \\
    &= \int_{-\infty}^{F_X^{-1}(y)} f(x) dx \\
    &= F_X(F_X^{-1}(y)) = y,
  \end{align*}
  which is the \cdf\ of a uniform random variable on $[0,1]$.
\end{proof}


\begin{lemma}
  \label{lem:invcdf}
  Let $U$ be a uniform random variable on~$[0,1]$ and let $F$ be the \cdf\ of an
  arbitrary random variable~$X$. Let $F^{-1}$ be the ``inverse'' of~$F$, in the
  sense
  \begin{equation}
    \label{eq:invcdf}
    F^{-1}(y) = \sup \{x : F(x) \le y\}.
  \end{equation}
  Then the random variable $F^{-1}(U)$ has the same distribution as~$X$.
\end{lemma}

\begin{proof}
  The definition of $F^{-1}$ according to~\eqref{eq:invcdf} has the property
  that $\Pr[F^{-1}(U) \le x] = \Pr[U \le F(x)]$. Thus,
  \begin{align*}
    \Pr[F^{-1}(U) \le x] &= \Pr[U \le F(x)] \\
    &= \int_0^{F(x)} d\xi = F(x).
  \end{align*}
\end{proof}

Consider now a channel $p(y|x)$ and let $\pi(x)$ be the capacity achieving
distribution at average cost~$P$, \ie, 
\begin{equation*}
  \pi(x) = \arg\max_{p(x): \E[\rho(X)] \le P} I(X;Y).
\end{equation*}
As before, we assume perfect feedback. The problem is to encode one source
symbol of an \emph{analog} source into~$n$ channel inputs. 

Let $\Fpi$ be the cumulative distribution function (\cdf) of the distribution
$\pi(x)$, and let $F_S$ be the \cdf\ of the source. In the first channel
use, the encoder produces
\begin{equation}
  \label{eq:posteriorx1}
  X_1 = \Fpi^{-1}(F_S(S)),
\end{equation}
where $\Fpi^{-1}$ is the inverse of $\Fpi$ according to~\eqref{eq:invcdf}. By
Lemma~\ref{lem:cdfunif}, $F_S(S)$ has uniform distribution on $[0,1]$, and so by
Lemma~\ref{lem:invcdf}, $\Fpi^{-1}(F_S(S))$ to a uniform random variable
produces a random variable with \cdf\ $\Fpi$. Thus, $X_1$ has the capacity
achieving distribution~$\pi(x)$. 

After $i-1$ rounds of transmission, the encoder knows $y_1$, \ldots, $y_{i-1}$,
the values of the $i-1$ past received symbols, and can compute the conditional
\cdf\ $F_{S|y_1, \ldots, y_{i-1}}$. It then sends in the $i^{\text{th}}$ channel
use
\begin{equation}
  \label{eq:posteriorxi}
  X_i = \Fpi^{-1}(F_{S|y_1,\dots,y_{i-1}}(S)).
\end{equation}
For any particular received values $y_1$, \ldots, $y_{i-1}$, therefore,
\begin{equation*}
  p(x_i|s, y_1, \dots, y_{i-1}) = \pi(x)
\end{equation*}
and so $X_i$ is independent of $Y_1$, \ldots, $Y_{i-1}$. 

Using this strategy the encoder produces an \iid\ sequence of inputs $X_i$
with the capacity achieving distribution~$\pi(x)$, satisfying conditions~(iv)
and~(v) of \thmref{tcntcfb}; condition~(iii) of the theorem is
trivially satisfied because the encoder is deterministic.

Let us look at Example~\ref{ex:gaussfb} once more, now from the perspective of
posterior matching.

\begin{example}
  \label{ex:gaussfbpost}
  Let $F_{\N(\mu, \sq)}$ be the \cdf\ of a Gaussian random variable of
  mean~$\mu$ and variance~$\sq$ and let $F_\N \deq F_{\N(0,1)}$. Then
  $F_{\N(\mu,\sq)}(x) = F_\N((x-\mu)/\sigma)$. Furthermore, the inverse \cdf\ is
  given by
  \begin{equation*}
    F_{\N(\mu,\sq)}^{-1}(y) = \sigma F_\N^{-1}(y) + \mu.
  \end{equation*}

  Since all involved random variables are jointly Gaussian, the source~$S$ can
  be written as
  \begin{equation*}
    S = \gamma_1 Y_1 + \dots + \gamma_{i-1} Y_{i-1} + W,
  \end{equation*}
  where $W$ is a Gaussian random variable of zero mean and variance $\swq$,
  independent of $Y_1$, \ldots, $Y_{i-1}$. Hence the distribution of~$S$ given
  $Y_1$, \ldots, $Y_{i-1}$ is Gaussian with mean $\E[S|Y_1, \dots, Y_{i-1}] =
  \gamma_1 Y_1 + \dots + \gamma_{i-1} Y_{i-1}$ and variance~$\swq$.  The
  posterior matching encoder~\eqref{eq:posteriorxi} therefore evaluates to
  \begin{align}
    X_i &= \sqrt{P} F_\N^{-1}\left( F_\N\left( \frac{S - \E[S|Y_1^{i-1}]}
    {\sqrt{\Var(S - \E[S|Y_1^{i-1}])}} \right) \right) \nonumber \\
    \label{eq:gausspmenc}
    &= \sqrt{P} \frac{S - \E[S|Y_1^{i-1}]}
    {\sqrt{\Var(S - \E[S|Y_1^{i-1}])}} .
  \end{align}
  Write now
  \begin{align*}
    S &= E_0 + (E_1 - E_1) - (E_2 - E_2) + \dots \pm (E_{i-2} -
    E_{i-2}) \\
    &= (E_0 + E_1) - (E_1 + E_2) + (E_2 + E_3) - \dots - E_{i-2} \\
    &= \Eh_0 - \Eh_1 + \Eh_2 - \dots - E_{i-2}.
  \end{align*}
  Since $\E[\Eh_j | Y_1^{i-1}] = \Eh_j$ for $j = 1$, \dots,~$i-2$, and
  $\E[E_{i-2}|Y_1^{i-1}] = \Eh_{i-2}$, 
  \begin{equation*}
    \E[S|Y_1^{i-1}] = \Eh_0 - \Eh_1 + \dots  - \Eh_{i-2}
  \end{equation*}
  and so $S - \E[S|Y_1^{i-1}] = \Eh_{i-2} - E_{i-2} = E_{i-1}$. Plugging this
  into~\eqref{eq:gausspmenc} yields exactly the encoder~\eqref{eq:gaussfbxi} of
  Example~\ref{ex:gaussfb}.
\end{example}


\subsection{Achieving the Rate Distortion Function?}

Above it was shown how, having knowledge of the past channel outputs,
the decoder could deterministically produce a new channel input that was
independent of the previous outputs and had the capacity achieving distribution.
According to \thmref{tcntcfb}, a further necessary
condition for a system to be optimal is that the conditional distribution of
$\Sh$ given $S$ achieves the rate distortion function of the source. While we
can use measure matching to find an the distortion function for which an
arbitrary Is this possible using a similar distribution matching strategy?
Unfortunately it turns out that the answer is no in general; the Gaussian
distribution is a notable exception.

It is appealing to apply a similar transformation at the channel output as done
at the channel input for the capacity case, in order to make the conditional
distribution of $\Sh$ given $S=s$ equal to the distribution $\Phi_s(\sh)$ that
achieves the rate distortion function. Assume the source symbol~$S$ is
transmitted across a single channel use. Let the decoder be given by
\begin{equation}
  \label{eq:distmatchdec}
  \Sh = g(Y) = F_{\Phi_s}^{-1}(F_{Y|S=s}(Y)).
\end{equation}
For any~$s$, therefore, $\Sh$ given~$s$ is distributed according to~$\Phi_s$,
and so the rate distortion function is achieved.

It is immediately clear, however, that this approach cannot work -- both \cdf s
needed to implement this decoder depend on the actual value of~$s$, which is
obviously not known at the decoder (there would not really be a communication
problem otherise). Interestingly, though, in the Gaussian case the dependence
on~$s$ of $F_{\Phi_s}$ and of $F_{Y|S=s}$ cancel each other out, and the
decoder~\eqref{eq:distmatchdec} yields again the MMSE decoder, as the following
example shows.

\begin{example}
  Let the source $S$ be distributed as $\N(0,1)$ and let the channel be AWGN
  with noise variance~$1$ and input constraint $\E[X^2] \le P$.
  The distortion is the squared error. The smallest achievable distortion is
  \begin{equation}
    \label{eq:exmindist}
    D_{\min} = \frac{1}{1 + P}.
  \end{equation}
  The capacity-achieving input distribution is $\N(0,P)$ and the conditional
  distribution of $\Sh$ given $S=s$ that achieves the rate distortion function
  at distortion~$D$ is $\N((1-D)s, D(1-D))$.
  
  The encoder that makes $X$ capacity achieving is therefore~$X = \sqrt{P}S$.
  The decoder from~\eqref{eq:distmatchdec} is
  \begin{align*}
    g(y) &= F_{\Phi_s}^{-1} \circ F_{Y|S=s}(y) \\
    &= \sqrt{D(1-D)} F_{\N}^{-1} \circ F_{\N}\left( y-\sqrt{P}s
    \right) + (1-D)s \\
    &= \sqrt{D(1-D)} \left( y-\sqrt{P}s \right) + (1-D)s,
  \end{align*}
  where we used $F_{\N}$ to denote the \cdf\ of a $\N(0,1)$ distribution.
  This expression still depends on~$s$; if we plug in the optimal distortion
  $D_{\min}$ from~\eqref{eq:exmindist}, however, the decoder becomes
  \begin{align*}
    g(y) &= \frac{\sqrt{P}}{P+1} (y - \sqrt{P}s) + \frac{P}{P +
    1}s \\ 
    &= \frac{\sqrt{P}y}{P + 1},
  \end{align*}
  which no longer depends on~$s$. Furthermore, this decoder is the MMSE decoder.
\end{example}

\subsection{Statisticically Sufficient Decoder}


\subsection{Geometric Interpretation}

If the channel is an AWGN channel, the distortion measure is the squared error,
and the decoder computes the LMMSE estimator, there is a nice geometrical
interpretation for the feedback transmission problem. (The derivation of the
inner product space follows Cramér and
Leadbetter~\cite[Section~5.6]{CramerL1967}.)

\begin{definition}
  An \emph{inner product space} $H$ is a set of elements (points, vectors) $x$,
  $y$, \dots satisfying the following properties.
  \begin{enumerate}
    \item There is an operation of \emph{addition}, assigning to each two
      elements $x, y \in H$ a unique element $z \in H$, denoted $x + y$.
      The unique element $y$ satisfying $x + y = z$ is denoted $y = z - x$. The
      element $0 = x-x$ is unique. 
    \item For every $c \in \R$, the operation of \emph{scalar multiplication}
      maps each $x \in H$ to $cx \in H$. 
    \item To every two elements $x,y \in H$ corresponds a unique scalar $\sp x
      y$ called the \emph{inner product} of $x$ and $y$ with the properties that
      for all $c \in \R$ and $x,y, z\in H$, $\sp{cx+y}{z} = c\sp xz + \sp yz$,
      and $\sp xx \ge 0$, with equality if and only if $x = 0$. The
      \emph{norm} of an element $x \in H$ is defined as $\|x\| = \sqrt{\sp xx}$.
      If $\sp xy = 0$ for some $x$ and $y$, then $x$ and $y$ are called
      \emph{orthogonal}.
  \end{enumerate}
\end{definition}

\begin{lemma}
  \label{lem:rvinprodsp}
  The set of zero-mean random variables with finite variance forms an inner
  product space under ordinary addition and with $\sp XY = \E[X Y]$. 
\end{lemma}

\begin{lemma}
  \label{lem:inprodspcond}
  Given a set of random variables in an inner product space~$H$ as defined
  above, the space of the random variables conditioned on $X \in H$ is
  equivalent to the projection of $H$ on the subspace orthogonal to~$X$.
\end{lemma}
\begin{proof}
  Conditioned on~$X$, we have $\sp YZ = \E[YZ|X]$. The covariance between $Y$
  and~$Z$ given~$X$ is
  \begin{align*}
    \Cov(Y, Z \mid X) &= \E[YZ|X] - \E[Y|X]\E[Z|X] \\
    &= \E[YZ|X] - \frac{\E[XY|X]\E[XZ|X]}{\E[X^2|X]} \\
    &= \sp YZ - \frac{\sp XY \sp XZ}{\|X\|^2} \\
    &= \left\bra Y - \frac{\sp XY}{\|X\|^2} X, Z - \frac{\sp XZ}{\|X\|^2} X
    \right\ket \\
    &= \sp{Y_{\perp X}}{Z_{\perp X}},
  \end{align*}
  where the subscript $\perp X$ denotes the projection onto the subspace
  orthogonal to~$X$.
\end{proof}


From the point of view of this inner product space, the first two rounds of
communication in Example~\ref{ex:gaussfb} can be illustrated as in
Figure~\ref{fig:gaussfb1}.
\begin{figure}[tbph]
  \begin{center}
    \figbox{gaussfb1}
  \end{center}
  \caption{Geometrical interpretation of the first two rounds of
  Example~\ref{ex:gaussfb}. The estimate $\Eh_0$ of the first transmission is
  the \emph{projection} of $S$ onto $Y_1$, hence the estimation error $E_1$ is
  orthogonal to~$Y_1$.}
  \label{fig:gaussfb1}
\end{figure}
The figure shows that while $X_1$ and $X_2$ are not independent, $X_2$ is
independent of $Y_1$. Since all future rounds of communication involve only
linear combinations of~$E_1$ and the noise components $Z_2$, $Z_3$, \dots, all
future communication takes place in a subspace orthogonal to~$Y_1$. For
reference, \figref{gaussfb2} shows the next round of communication.
\begin{figure}[tbph]
  \begin{center}
    \figbox{gaussfb2}
  \end{center}
  \caption{Second and third round of communication of \exref{gaussfb}.}
  \label{fig:gaussfb2}
\end{figure}

Since the estimate $\Eh_i$ is the projection of $E_i$ onto $Y_{i+1}$, 
\begin{equation*}
  \Eh_i = \frac{\sp{E_i}{Y_{i+1}}}{\|Y_{i+1}\|^2} Y_{i+1}.
\end{equation*}
Writing $Y_{i+1} = \alpha E_i + Z_{i+1}$, with $\alpha = \sqrt{P}/\|E_i\|$,
yields
\begin{equation*}
  \|E_{i+1}\|^2 = \|\Eh_i - E_i\|^2 = \frac{\|E_i\|^2}{1 + P/\|Z_{i+1}\|^2},
\end{equation*}
which is nothing else than Equation~\ref{eq:gaussvardec} expressed in the inner
product space formalism.



\subsection{Other Results}

\subsubsection{Noisy Feedback}

\subsubsection{Minimal-Delay Codes for Arbitrary Rates}


\begin{subappendices}
  \section{Asymptotic Notation}
  \label{app:asymptotic}

  \begin{definition}
    \label{def:bigo}
    Let $f(x)$ and $g(x)$ be two functions defined on~$\R$. The set $O(g(x))$ is
    defined as
    \begin{equation*}
      f(x) \in O(g(x))
    \end{equation*}
    if and only if there exists an $x_0$ and a constant~$c$ such that
    \begin{equation*}
      f(x) \le c g(x)
    \end{equation*}
    for all $x > x_0$. 
    Similarly, $f(x) \in \Omega(g(x))$ if $\le$ is replaced by $\ge$ in
    the above definition. Finally, $\Theta(g(x)) \deq O(g(x)) \cap
    \Omega(g(x))$.
  \end{definition}

\end{subappendices}

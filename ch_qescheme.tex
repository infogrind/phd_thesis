\chapter{A Hybrid Transmission Scheme}
\label{ch:qescheme}

Introductory paragraphs go here.

\section{Transmission Strategy}
\label{sec:commscheme}

To encode a single source letter $S$ into $n$~channel input symbols $X_1$,
\dots, $X_n$, we proceed as follows. Define $E_0 = S$ and recursively compute
the pairs $(Q_i, E_i)$ as
\begin{align}
  Q_i &= \frac{1}{\beta} \Int(\beta E_{i-1}) \nonumber \\
  E_i &= \beta (E_{i-1} - Q_i) \label{eq:QEdef}
\end{align}
for $i = 1$, \dots, $n-1$ where $\Int(x)$ is the unique integer~$i$ such that
\begin{equation*}
%  \label{eq:intdef}
  x \in \left[ i - \frac12 , i + \frac12 \right)
\end{equation*}
and $\beta$ is a scaling factor that \emph{grows with the power $P$}
in a way to be determined later.  The following result will be useful in the
sequel.
\begin{lemma}
  \label{lem:qvarconvergence}
  As $\beta$ goes to infinity, the variance of $Q_i$ converges to that of
  $E_{i-1}$ for all $i = 1$, \dots, $n-1$. 
\end{lemma}

\begin{proof}
  Intuitively this is so since $Q_i$ is $E_{i-1}$ quantized, and the
  quantization step becomes smaller as $\beta$ goes to infinity.  A rigorous
  proof is given in Appendix~\ref{app:lemma1proof}.
\end{proof}

\begin{proposition}
  \label{prop:qeproperties}
  The $Q_i$ and $E_i$ satisfy the following properties:
\begin{enumerate}
  \item The map $S \mapsto (Q_1, \dots, Q_{n-1}, E_{n-1})$ is one-to-one and
    \begin{equation}
      \label{eq:unwraprec}
      S = \sum_{i=1}^{n-1} \frac{1}{\beta^{i-1}} Q_i + \frac{1}{\beta^{n-1}}
      E_{n-1}.
    \end{equation}

  \item The variance of $E_0$ is $\ssq$ and for all $i = 1$, \dots, $n-1$, $E_i
    \in [-1/2, 1/2)$ and $\var(E_i) \le 1/4$.
  \item For any $\delta > 0$ there exists $\beta_0$ such that for $\beta
    > \beta_0$,
    \begin{equation}
      \label{eq:qvarbound}
      \var(Q_i) \le
      \begin{cases}
        \ssq + \delta & \text{for $i = 1$} \\
        1/4 + \delta & \text{for $i = 2$, \dots, $n-1$.}
      \end{cases}
    \end{equation}
\end{enumerate}
\end{proposition}

\goodbreak
\begin{proof}
  \begin{enumerate}
    \item From the definition~\eqref{eq:QEdef} we have
    \begin{equation}
      \label{eq:reverserec}
      E_{i-1} = \frac{1}{\beta} E_i + Q_i.
    \end{equation}
    Repeated use of this relationship leads to the given expression for~$S$. 
  \item First, $\var(E_0) = \var(S) = \ssq$. Next, $E_i \in [-1/2, 1/2)$ follows
    trivially from the definition of~$E_i$.  Furthermore, the variance of any
    random variable with support in an interval of length~$1$ is bounded from
    above by~$1/4$. 
  \item The result follows directly from Lemma~\ref{lem:qvarconvergence} and
    from the bound on the variance of the $E_i$ in point~2 above.
  \end{enumerate}
\end{proof}

Without loss of generality we assume hereafter that $\ssq > 1/4$ so that the
first bound of~\eqref{eq:qvarbound} applies to all~$Q_i$.


We determine the channel input symbols $X_i$ from the $Q_i$ and from $E_{n-1}$
according to 
\begin{align}
  X_i &= \sqrt{\frac{P}{\ssq + \delta}} Q_i \quad
  \text{for $i = 1$, \dots, $n-1$ and} \nonumber\\
  X_n &= \sqrt{\frac{P}{\seq}} E_{n-1},
  \label{eq:Xdef}
\end{align}
where $\seq = \var(E_{n-1})$.  Following Proposition~\ref{prop:qeproperties},
this ensures that $\E[X_i^2] \le P$ for all~$i$ and for $\beta >
\beta_0(\delta)$.  Since we are interested in the large SNR regime and since we
have defined $\beta$ to grow with~$P$, we can thus assume for the remainder that
the power constraint is satisfied. 


\section{Geometrical Aspects}

\section{Lower Bound on the Squared Error}

The goal of this section is to lower bound the scaling of the mean squared error
of the transmission strategy described in Section~\ref{sec:commscheme}.

Throughout this section we assume $\beta^2 = \snr^{1-\e}$, where $\e = \e(\snr)$
is a positive function of~$\snr$. This results in no loss of generality, since
for an arbitrary positive function~$f$ we can set $\e(\snr) =
1-\log(f(\snr))/\log\snr$ to get $\beta^2(\snr) = f(\snr)$.

Note that by~\eqref{eq:QEdef}, the $Q_i$ are completely determined by~$S$.
In this section, with a slight abuse of notation, we therefore write $Q_i(s)$ to
denote the value of~$Q_i$ when $S = s$. We use $E_i(s)$ and $X_i(s)$ in a
similar manner. Furthermore, we define $\X(s) = (X_1(s), \dots, X_n(s))$.

The following result, adapted from Ziv~\cite{Ziv1970}, is a key ingredient in
the proofs of the lemmas that follow.

\begin{lemma}
  \label{lem:zivbound}
  \label{LEM:ZIVBOUND}
  Consider a communication system where a con\-tin\-u\-ous-valued source~$S$ is
  encoded into an $n$-dimensional vector $\X(S)$, sent across $n$~independent
  parallel AWGN channels with noise variance~$\szq$, and decoded at the receiver
  to produce an estimate~$\Sh$.  If the density $p_S$ of the source is such that
  there exists an interval $[A,B]$ and a number $p_{\min} > 0$ such that $p_S(s)
  \ge p_{\min}$ whenever $s \in [A,B]$, then for any $\Delta \in [0,B-A)$ the
  mean squared error incurred by the communication system satisfies
  \begin{equation}
    \label{eq:zivbound}
    \E[(\Sh - S)^2] \ge p_{\min} \left(\frac{\Delta}{2} \right)^2 
    \int_A^{B-\Delta} Q(d(s, \Delta) / 2 \sz) ds,
  \end{equation}
  where $d(s, \Delta) \deq \|\vect{X}(s) - \vect{X}(s+\Delta)\|$ and 
  \[Q(x) = \int_x^{\infty} (1/\sqrt{2\pi}) \exp\{-\xi^2/2\} d\xi.\]
\end{lemma}

\begin{proof}
  See Appendix~\ref{app:zivboundproof}.
\end{proof}

The next two lemmata provide two different asymptotic lower bounds on the
mean squared error of our transmission strategy, each of which is tighter for a
different class of~$\e$. They hold regardless of the decoder used.  (The
$\Omega$-notation is defined in Appendix~\ref{app:Onotation}.)

\begin{lemma}
  \label{lem:lowerbound1}
  For an arbitrary function $\e(\snr) \ge 0$, the mean squared error satisfies
  \begin{equation*}
    \mse \in \Omega(\snr^{-n + (n-1)\e}).
  \end{equation*}
\end{lemma}

\begin{lemma}
  \label{lem:lowerbound2}
  For an arbitrary function $\e(\snr) \ge 0$, the mean squared error satisfies
  \begin{equation*}
    \mse \in \Omega(\snr^{-1+\e/2} \exp\{-\snr^\e/k\})
  \end{equation*}
  where $k > 0$ does not depend on~$\snr$.
\end{lemma}

\emph{Discussion:} An immediate consequence of the lemmata is that the
theoretically optimal scaling $\snr^{-n}$ is not achievable with the given
encoding strategy: by Lemma~\ref{lem:lowerbound1} this would require $\e = 0$,
but following Lemma~\ref{lem:lowerbound2} the scaling is at best $\snr^{-1}$ if
$\e = 0 $.  More generally, which one of the two lower bounds decays more slowly
and is therefore tighter depends on the scaling of~$\e(\snr)$. How to
choose~$\e(\snr)$ optimally will be the subject of Theorem~\ref{thm:scalinglb}.

\begin{proof}[Proof of Lemma~\ref{lem:lowerbound1}]
  Assume $\Delta \in [0, \beta^{-(n-1)})$ and define for $j \in \Z$
  \[ \I_j^\Delta = \left[ (j - \frac12 )\beta^{-(n-1)}, 
    (j + \frac12 ) \beta^{-(n-1)} - \Delta \right).\]
  It can be verified from~\eqref{eq:QEdef} that if $s \in \I_j^\Delta$ for
  some~$j$, the following properties hold: 1) $Q_i(s) = Q_i(s+\Delta)$ for
  $i=1$, \dots, $n-1$, and 2) $E_{n-1}(s+\Delta) - E_{n-1}(s) =
  \beta^{n-1}\Delta$.  From~\eqref{eq:Xdef} it follows that
  $s \in \I_j^\Delta$ implies $d(s, \Delta) = \sqrt{P/\seq} \beta^{n-1}\Delta$.

  We now apply Lemma~\ref{lem:zivbound} and restrict the integral to the
  set~$\psi(\Delta) \deq [A,B-\Delta) \cap \bigcup_{j\in\Z} \I_j^\Delta$. The
  lower bound is then relaxed to give
  \begin{equation*}
    \mse \ge \frac{\pmin}{4} \Delta^2 Q(\sqrt{\snr/\seq} \beta^{n-1} \Delta/2)
    \int_{\psi(\Delta)} ds.
  \end{equation*}
  Letting $\Delta = 1/(\sqrt{\snr}\beta^{n-1})$ and $\beta^2 = \snr^{1-\e}$
  yields
  \begin{equation*}
    \mse \ge \frac{\pmin}{4} \snr^{-n+(n-1)\e} Q\left(\frac{1}{2\se}\right)
    \int_{\psi(\Delta)} ds.
  \end{equation*}

  The proof is almost complete, but we still have to show that
  $\int_{\psi(\Delta)}ds$ can be bounded below by a constant for large SNR. The
  length of a single interval~$\I_j^\Delta$ is $\beta^{-(n-1)} - \Delta$. Within
  $[A,B-\Delta)$ there are $(B-A-\Delta)\beta^{n-1}$ such
  intervals. The total length of all intervals~$\I_j^\Delta$ in $[A, B-\Delta)$
  is therefore
  \[ \int_{\psi(\Delta)} ds = (B-A-\Delta)
  (1 - \beta^{n-1}\Delta), \]
  which, for the given values of~$\beta$ and~$\Delta$, 
  converges to $B-A$ for $\snr \ra \infty$ and thus can be lower bounded by a
  constant for $\snr$ greater than some $\snr_0$. With this, the proof is
  complete.
\end{proof}

\begin{proof}[Proof of Lemma~\ref{lem:lowerbound2}]
  Observe first that~\eqref{eq:QEdef} implies $Q_1(s + \beta^{-1}) = Q_1(s) +
  \beta^{-1}$ and $E_1(s + \beta^{-1}) = E_1(s)$. Since all $Q_i$ and $E_i$ for
  $i \ge 2$ are by recursion a function of $E_1$ only, $Q_i(s) = Q_i(s +
  \beta^{-1})$ for $i = 2$, \dots, $n-1$, and $E_{n-1}(s) = E_{n-1}(s +
  \beta^{-1})$. Consequently,  $X_i(s) = X_i(s + \beta^{-1})$ for all $i =
  2$, \dots, $n$. By~\eqref{eq:Xdef} and the above, the Euclidean distance
  between $\X(s)$ and~$\X(s+\beta^{-1})$ is therefore
  \begin{equation}
    \label{eq:xbetadist}
    \sqrt{\frac{P}{\ssq + \delta}} |Q_1(s) - Q_1(s+\beta^{-1})| 
    = \sqrt{\frac{P}{\ssq + \delta}} \beta^{-1}.
  \end{equation}

  We now apply Lemma~\ref{lem:zivbound} with $\Delta = \beta^{-1}$. The
  parameter $\beta$ will be chosen to increase with~$\snr$, therefore $\Delta
  \in [0, B-A)$ holds for sufficiently large~$\snr$.

  Using~\eqref{eq:xbetadist}, the resulting bound on the mean squared error is
  \begin{equation*}
    \mse \ge \frac{\pmin}{4} \beta^{-2}
    Q\left (\sqrt{\frac{\snr}{\ssq + \delta}} \frac{\beta^{-1}}{2} \right)
    (B-A-\beta^{-1}
    ).
  \end{equation*}
  Replacing $\beta^2 = \snr^{1-\e}$ and using the fact that $Q(x)$ converges to
  $\exp\{-x^2/2\}/\sqrt{2\pi}x$ for $x \goesto \infty$
  (cf.~\cite{AbramowitzS1964}) we obtain
  \begin{equation*}
    \mse \ge c \snr^{-1 + \e/2} \exp\{-\snr^\e/k\}
  \end{equation*}
  for sufficiently large~$\snr$, with $c$ and $k$ positive constants that do
  not depend on~$\snr$, thus proving the lemma.
\end{proof}

The following lemma will be used to prove Theorem~\ref{thm:scalinglb}, the main
result of this section.

\begin{lemma}
  \label{lem:epssolution}
  Define $W(x)$ to be the function that satisfies $W(x)e^{W(x)} = x$ for $x >
  0$.  This function is well defined and is sometimes called the \emph{Lambert
  $W$-function}~\textnormal{\cite{CorlessGHJK1996}}. Then for $\snr > 1$ and
  arbitrary real constants $a$, $b>0$, and $k > 0$, 
  \begin{equation}
    \label{eq:epsequation}
    \snr^{a+b\e} = \exp\{-\snr^\e/k\},
  \end{equation}
  if and only if
  \begin{equation}
    \label{eq:epssolution}
    \snr^\e = bk W(\snr^{-a/b} / bk).
  \end{equation}
\end{lemma}

\begin{proof}
  Let $\snr>1$. Since $\snr^{a+b\e}$ is strictly increasing and
  $\exp\{-\snr^\e/k\}$ is strictly decreasing in~$\e$, there is at most one
  solution to~\eqref{eq:epsequation}.  Assume now $\snr^\e$ is as
  in~\eqref{eq:epssolution}. Then
  \begin{equation*}
    \exp\{-\snr^\e/k\} = \exp\{-b W(\snr^{-a/b}/bk)\}.
  \end{equation*}
  On the other hand,
  \begin{align*}
    \snr^{a+b\e} &= \snr^a \left( bk W(\snr^{-a/b}/bk) \right)^b \\
    &= \left( W(\snr^{-a/b}/bk) / (\snr^{-a/b}/bk) \right)^b.
  \end{align*}
  By definition, $W(x)/x = e^{-W(x)}$, so the above is equal to
  \begin{equation*}
    \snr^{a+b\e} = \exp\{-bW(\snr^{-a/b}/bk)\},
  \end{equation*}
  which proves the claim.
\end{proof}

The following is the main result of this section.
\begin{theorem}
  \label{thm:scalinglb}
  For any parameter~$\beta$ and for any decoder, the mean squared error of the
  transmission strategy described in Section~\ref{sec:commscheme} satisfies
  \begin{equation*}
    \mse \in \Omega(\snr^{-n}(\log\snr)^{n-1}).
  \end{equation*}
\end{theorem}

\emph{Discussion:} The asymptotic lower bound on the mean squared error given by
the theorem coincides with the asymptotic performance achieved by the suboptimal
decoder in Section~\ref{sec:achievable}; the bound is therefore asymptotically
tight. 

%
\begin{proof}[Proof of Theorem~\ref{thm:scalinglb}]
  Define $l_1(\snr, \e) = \snr^{-n+(n-1)\e}$ and $l_2(\snr,\e) = \snr^{-1+\e/2}
  \exp\{-\snr^\e/k\}$. By Lemmata~\ref{lem:lowerbound1}
  and~\ref{lem:lowerbound2},
  \begin{equation*}
    \mse \in \Omega\big(\max \left( l_1(\snr,\e), l_2(\snr,\e) \right) \big).
  \end{equation*}
  The optimal parameter $\e(\snr)$ is therefore such that for
  any~$\snr$
  \begin{equation}
    \label{eq:epsmax}
    \max\left( l_1(\snr,\e), l_2(\snr,\e) \right)
  \end{equation}
  is minimized. Now for any fixed~$\snr$, $l_1(\snr,\e)$ is increasing in~$\e$,
  and $l_2(\snr,\e)$ is increasing in~$\e$ for $0 \le \e < \xi =
  \log(k/2)/\log\snr$ and decreasing in~$\e$ for $\e \ge \xi$.
  The maximum
  in~\eqref{eq:epsmax} is therefore minimized either for $\e = 0$ or for $\e \ge
  \xi$
  such that $l_1(\e) = l_2(\e)$. As we have remarked earlier, $\e = 0$ leads to
  a worse performance than that achieved in Section~\ref{sec:achievable}, and so
  this cannot be the optimal parameter. We therefore have to choose
  $\e(\snr)$ such that $l_1(\snr,\e) = l_2(\snr, \e)$.  Inserting the
  definitions of $l_1$ and $l_2$ and rearranging the terms yields
  \begin{equation*}
    \snr^{-(n-1) + (n-3/2)\e} = \exp\{-\snr^\e/k\},
  \end{equation*}
  which is of the form~\eqref{eq:epsequation} with $a = -(n-1)$ and $b = n-3/2$.
  By Lemma~\ref{lem:epssolution}, for $\snr > 1$,
  \begin{equation*}
    \snr^\e = (n-3/2)k W(\snr^{\frac{2(n-1)}{2n-3}} / ((n-3/2)k)).
  \end{equation*}
  We now use the fact that $W(x)/\log x$ converges to~$1$ for $x \ra \infty$;
  this can be shown using L'H\^opital's rule and because the derivative of
  $W(x)$ is $W(x)/[x(1 + W(x))]$ (cf.~\cite{CorlessGHJK1996}).

  For sufficiently large $\snr$, therefore, there exists a constant $c > 0$ such
  that
  \begin{equation*}
    \snr^\e \ge c(n-3/2)k \left[ \frac{2(n-1)}{2n-3}\log\snr - \log((n-3/2)k)
    \right],
  \end{equation*}
  and so $\snr^\e \in \Omega(\log\snr)$. Plugging this into the bound of
  Lemma~\ref{lem:lowerbound1} we finally obtain\footnote{If $a(x) \in
  \Omega(f(x))$ and $b(x) \in \Omega(g(x))$, then $a(x)b(x)^m \in
  \Omega(f(x)g(x)^m)$.}
  \begin{equation*}
    \mse \in \Omega(\snr^{-n}(\log\snr)^{n-1}),
  \end{equation*}
  and no choice of $\e(\snr)$ can improve this bound.
\end{proof}


\section{Asymptotical Achievability of Lower Bound}
\label{sec:achievable}

\subsection{A Suboptimal Decoder}

The $X_i$ are transmitted across the channel, producing at the channel output
the symbols
\begin{equation*}
  Y_i = X_i + Z_i, \quad i = 1, \dots, n,
\end{equation*}
where the $Z_i$ are iid Gaussian random variables of variance~$\szq$. 
To estimate $S$ from  $Y_1$, \dots, $Y_n$, the decoder first
computes separate estimates $\Qh_1$, \dots, $\Qh_{n-1}$ and $\Eh_{n-1}$, and
then combines them to obtain the final estimate~$\Sh$.  While this strategy is
suboptimal in terms of achieving a small MSE, we will see that it is good enough
to achieve optimal scaling.

To estimate the $Q_i$ we use a maximum likelihood (ML) decoder, which yields the
minimum distance estimate
\begin{equation}
  \label{eq:mldecoder}
  \Qh_i = \frac{1}{\beta} \arg \min_{j\in \Z} \left| \sqrt{\frac{P}{\ssq
  + \delta} } \frac{j}{\beta} - Y_i \right|.
\end{equation}
To estimate $E_{n-1}$, we use a linear minimum mean-square error (LMMSE)
estimator (see \eg~\cite[Section~8.3]{Scharf1990}), which computes
\begin{equation}
  \label{eq:lmmse}
  \Eh_{n-1} = \frac{\E[E_{n-1} Y_n]}{\E[Y_n^2]} Y_n.
\end{equation}
Finally we use the relationship~\eqref{eq:unwraprec} to obtain
\begin{equation}
  \label{eq:unwrapestim}
  \Sh = \sum_{i=1}^{n-1} \frac{1}{\beta^{i-1}} \Qh_i + \frac{1}{\beta^{n-1}}
  \Eh_{n-1}.
\end{equation}


\subsection{Error Analysis}

The overall MSE $\E[(S-\Sh)^2]$ can be broken up into contributions due to the
errors in decoding $Q_i$ and $E_{n-1}$ as follows. From~\eqref{eq:unwraprec}
and~\eqref{eq:unwrapestim}, the difference between $S$ and $\Sh$ is
\begin{equation*}
  S - \Sh = \sum_{i=1}^{n-1} \frac1{\beta^{i-1}} (Q_i - \Qh_i) + \frac1{\beta^{n-1}}
  (E_{n-1} - \Eh_{n-1}).
\end{equation*}
The error terms $Q_i - \Qh_i$ depend only on the noise of the respective channel
uses and are therefore independent of each other and of $E_{n-1} - \Eh_{n-1}$,
so we can write the error variance componentwise as
\begin{equation}
  \label{eq:totalerror}
  \E[(S-\Sh)^2] = \sum_{i=1}^{n-1} \frac{1}{\beta^{2(i-1)}} \Eqi +
  \frac{1}{\beta^{2(n-1)}} \Ee, 
\end{equation}
where $\Eqi \deq \E[(Q_i - \Qh_i)^2]$ and $\Ee \deq \E[(E_{n-1} -
\Eh_{n-1})^2]$.

\begin{lemma}
  \label{lem:eqbound}
  For each $i = 1$, \dots, $n-1$, 
  \begin{equation}
    \label{eq:eqidecay}
    \Eqi \in O\left(\exp\{-k \snr/\beta^2\}\right),
  \end{equation}
  where $\snr = P/\szq$ and $k > 0$~is a constant.
\end{lemma}
(The $O$-notation is defined in Appendix~\ref{app:Onotation}.)

\begin{proof}
  Define the interval
  \begin{equation*}
    \I_j = \left[ \frac{(j - \frac12) \sqrt{P}}{\beta \sqrt{\ssq + \delta}},
    \frac{(j + \frac12) \sqrt{P}}{\beta \sqrt{\ssq + \delta}} \right).
  \end{equation*}
  According to the minimum distance decoder~\eqref{eq:mldecoder}, $\Qh_i - Q_i
  = j/\beta$ whenever $Z_i \in \I_j$.  The error $\Eqi$ satisfies thus
  \begin{align}
    \E[(Q_i - \Qh_i)^2] &= \frac{1}{\beta^2} \sum_{j \in \Z} j^2 \Pr[Z_i \in
    \I_j]  \nonumber \\
    &= \frac{2}{\beta^2} \sum_{j = 1}^\infty j^2 \Pr[Z_i \in \I_j],
    \label{eq:eqexact}
  \end{align}
  where the second equality follows from the symmetry of the distribution
  of~$Z_i$. Now,
  \begin{equation*}
    \Pr[Z_i \in \I_j] = Q\left( \frac{(j - \frac12) \sqrt \snr}{\beta \sqrt{\ssq
    + \delta }} \right) - Q\left( \frac{(j + \frac12) \sqrt{\snr}}{\beta
    \sqrt{\ssq + \delta}} \right),
  \end{equation*}
  where
  \begin{equation*}
    Q(x) = \frac{1}{\sqrt{2\pi}} \int_x^\infty e^{-\xi^2/2} d\xi,
  \end{equation*}
  which can be bounded from above for $x \ge 0$ as
  \begin{equation*}
    Q(x) \le \frac12 e^{-x^2/2}.
  \end{equation*}
  For $\beta \ge 1$ we can now bound~\eqref{eq:eqexact} as
  \begin{equation*}
    \Eqi \le \sum_{j=1}^\infty j^2 \exp\left\{ - \frac{(j - 1/2)^2
    \snr}{2\beta^2(\ssq + \delta)} \right\}.
  \end{equation*}
  Note that for $j \ge 2$, $(j - 1/2)^2 > j$.  Thus
  \begin{eqnarray}
    \Eqi &\le & \exp \left \{ - \frac{\snr}{8 \beta^2 (\ssq +\delta)} \right\}
    \nonumber \\
    & & \mbox{} + 
    \sum_{j = 2}^\infty j^2 \exp \left \{ - \frac{j \snr}{2\beta^2(\ssq
    +\delta)}
    \right\}. \label{eq:eqibound}
  \end{eqnarray}
  To bound the infinite sum we use 
  \begin{equation}
    \label{eq:geomsum}
    \sum_{j=2}^\infty j^2 p^j \le \sum_{j=1}^\infty j^2 p^j = 
    \frac{p^2+p}{(1-p)^3}
  \end{equation}
  with $p = \exp\{-\snr/2 \beta^2 (\ssq+\delta)\}$. The first term
  of~\eqref{eq:eqibound} thus dominates for large values of
  $\snr/\beta^2$ and
  \begin{equation*}
    \Eqi \le c\exp\left\{ - \frac{\snr}{8 \beta^2 (\ssq + \delta)} \right\}
  \end{equation*}
  for some~$c > 0$, which completes the proof. 
\end{proof}

\begin{lemma}
  \label{lem:eedecay}
  $\Ee \in O(\snr^{-1})$. 
\end{lemma}
\begin{proof}
  The mean-squared error that results from the LMMSE estimation~\eqref{eq:lmmse}
  is
  \begin{equation}
    \label{eq:lmmse-error}
    \Ee = \seq - \frac{(\E[E_{n-1}
    Y_n])^2}{\E[Y_n^2]}. 
  \end{equation}
  Since
  \begin{equation*}
    Y_n = X_n + Z_n = \sqrt{\frac{P}{\seq}} E_{n-1} + Z_n,
  \end{equation*}
  we have $\E[E_{n-1}Y_n] = \sqrt{P\seq}$. Moreover, $\E[Y_n^2] = \E[X^2] +
  \E[Z^2] = P+\szq$.  Inserting this into~\eqref{eq:lmmse-error} we obtain
  \begin{align*}
    \Ee &= \seq - \frac{P \seq}{P + \szq} \\
    &= \seq \left( 1 - \frac{P}{P + \szq} \right) \\
    &= \frac{\seq}{1 + \snr} \\
    & < \frac{\seq}{\snr}.
  \end{align*}
  Since $\seq$ is bounded (cf.\ Proposition~\ref{prop:qeproperties}), 
  $\Ee \in O(\snr^{-1})$ as claimed.

\end{proof}


\section{Towards Lower Bounds for General Schemes}

\subsection{Apparent Limitations of Ziv's Bound}

Lemma~\ref{lem:zivbound} allows to lower bound the mean squared error by
exploiting particular geometric properties of the constellation. More precisely,
if there exists a $\Delta$ such that $d(s, \Delta)/2\sz$ can be upper bounded
(at least for a set of~$s$ with minimum positive probability), we can lower
bound the error probability of binary signaling and through the lemma obtain a
bound on the mean squared error.

We could use this lemma to lower bound the error of the scheme presented in
Section~\ref{sec:achievable} because the regular structure of the constellation
gave a lower bound on $d(s,\Delta)$ for $\Delta = \beta^{-1}$. 

For a more general class of encodings we not have a bound on $d(s,\Delta)$ for
some fixed~$\Delta$. Rather, there might be a~$\Delta$ for each~$s$, but
these~$\Delta$ might not be the same for all~$s$.

Consider the following example. Divide the unit interval into $m^2$ intervals.
Map each interval to a point to the unit square~${-\frac12, \frac12}^2$.
Transmit the point through a $2$-dimensional AWGN channel. At the channel
output, the decoder estimates the transmitted point and outputs the midpoint of
the corresponding quantization interval. If the correct point is estimated then
the squared error is of the order $1/m^2$. 

Let now $m = \sqrt{\snr}$. The assumption is that the following is true: There
is a positive (lower bounded) probability that the transmitted point, say $x_i$,
has within distance $1/m = k/\sqrt{\snr}$, with $k$ a constant independent
of~$\snr$, another point $x_j$ such that the midpoints of the source intervals
corresponding to $x_i$ and $x_j$ are more than $m/\sqrt{\snr}$ apart.

\subsection{Extension of Ziv's Bound to Average Distance}

Recall the bound from Lemma~\ref{lem:zivbound}:
\begin{equation}
  \label{eq:zivagain}
  \mse \ge \pmin \dhsq \intabd Q(d(s,\Delta) / 2 \sz) ds.
\end{equation}
Now define the average distance between $f(s)$ and $f(s+\Delta)$ when $s$ is
uniformly distributed over $[A, B-\Delta]$ as
\begin{equation*}
  \dbar(\Delta) \deq \frac{1}{B-A-\Delta} \intabd d(s, \Delta) ds.
\end{equation*}
Let $\phi(\Delta)$ be the set of $s \in [A, B-\Delta]$ for which $d(s, \Delta)
\le 2 \dbar(\Delta)$. By Markov's inequality, this set is such that
\begin{equation*}
  \frac{1}{B-A-\Delta} \int_{\phi(\Delta)} ds \ge \frac{1}{2}.
\end{equation*}
Thus, restricting the integral in~\eqref{eq:zivagain} to the set~$\phi(\Delta)$,
we obtain
\begin{equation}
  \label{eq:zivdbar}
  \mse \ge \frac{\pmin}{2} \dhsq Q(\dbar(\Delta)/\sz) (B-A-\Delta).
\end{equation}
Lastly define
\begin{equation*}
  d(\Delta) \deq \frac{1}{B-A} \int_A^B d(s,\Delta) ds
\end{equation*}
and note that
\begin{align*}
  d(\Delta) &\ge \frac{B-A-\Delta}{B-A} \frac{1}{B-A-\Delta} \intabd d(s,
  \Delta) ds \\
  &= \left(1 - \frac{\Delta}{B-A}\right) \dbar(\Delta).
\end{align*}
Inserting in~\eqref{eq:zivdbar} yields
\begin{equation*}
  \mse \ge \frac{\pmin}{2} \dhsq Q\left(\frac{d(\Delta)}{\left(1 -
  \frac{\Delta}{B-A}\right) \sz} \right) (B-A-\Delta).
\end{equation*}
If $\Delta \goesto 0$ as $\snr \goesto \infty$, this
is asymptotically equivalent to
\begin{equation*}
  \mse \ge c \Delta^2 Q(d(\Delta)/\sz)
\end{equation*}
with $c$ appropriately defined.

The advantage of defining $d(\Delta)$ as the average distance $d(s,\Delta)$ over
an \emph{uniform} distribution of~$s$ is that it allows to compute $d(\Delta)$
as a function of the encoding strategy alone, without regard to the actual
source distribution.

\section{Summary}

This section will summarize the chapter's results.



%%%%%%%%%%%%%%%%%%%%%%%%%%%% APPENDICES %%%%%%%%%%%%%%%%%%%%%%%%%%%%%%%%%%%%%%%%

\begin{subappendices}
  \section{Proof of Lemma~\ref{lem:qvarconvergence}}
  \label{app:lemma1proof}

  Since all involved distributions are symmetric, $\E[Q_i] = 0$.  Writing $Q_i$
  as a function of $E_{i-1}$, we have
  \begin{equation}
    \var(Q_i) = \E[Q_i^2] = \int_{-\infty}^\infty Q_i(\xi)^2 f(\xi) d\xi,
    \label{eq:varqintegral}
  \end{equation}
  %\begin{align}
  %  \var(Q_i) &= \E[Q_i^2] \nonumber\\
  %  &= \int_{-\infty}^\infty Q_i(\xi)^2 f(\xi) d\xi,
  %  \label{eq:varqintegral}
  %\end{align}
  where $f(\xi)$ is the pdf\footnote{probability density function} of $E_{i-1}$.
  Now, $Q_i(\xi) = j/\beta$ whenever
  \begin{equation*}
    \xi \in \left[ \frac{j - 1/2}{\beta}, \frac{j + 1/2}{\beta} \right).
  \end{equation*}
  With this, the integral~\eqref{eq:varqintegral} becomes
  \begin{align*}
    \var(Q_i) &= \frac{1}{\beta^2} \sum_{j \in \Z} j^2 
    \int_{\frac{j - 1/2}{\beta}}^{\frac{j + 1/2}{\beta}} f(\xi) d\xi \\
    &= \sum_{j\in \Z} \left( \frac{j}{\beta} \right)^2 \left[ F\left( 
    { \textstyle
    \frac{j + 1/2}{\beta} }\right) - F \left( { \textstyle \frac{j - 1/2}{\beta}
    } \right) \right],
  \end{align*}
  where $F(\xi)$ is the cdf\footnote{cumulative distribution function} of
  $E_{i-1}$. As $\beta$ goes to infinity, this sum converges to a
  Riemann-Stieltjes integral:
  \begin{equation*}
    \var(Q_i) \longrightarrow \int \xi^2 dF(\xi) = \var(E_{i-1}) \quad
    \text{as $\beta \goesto \infty$.}
  \end{equation*}
  \qed


  \section{Proof of Ziv's Lower Bound (Lemma~\ref{lem:zivbound})}
  \label{app:zivboundproof}

  If we condition the mean squared error on~$S$ and use the assumption on~$p_S$
  we obtain
  \begin{equation*}
    \mse \ge \pmin \int_A^B \msecond ds.
  \end{equation*}
  For $\Delta \in [0, B-A]$ we can further bound this in two ways:
  \begin{align*}
    \mse &\ge \pmin \intabd \msecond ds \\
    \mse &\ge \pmin \int_{A+\Delta}^B \msecond ds \\
    &= \pmin \intabd \msecondd ds.
  \end{align*}
  Averaging the two lower bounds yields
  \begin{equation}
    \label{eq:avglbd}
    \mse \ge \frac{\pmin}{2} \intabd \bigg( \msecond + \\
    \msecondd \bigg) ds,
  \end{equation}
  and applying Markov's inequality to the expectation terms leads to
  \begin{equation}
    \label{eq:markov1}
    \msecond \ge \dhsq \Pr[|\Sh - S| \ge \Delta/2 \mid s]
  \end{equation}
  and
  \begin{equation}
    \label{eq:markov2}
    \msecondd \\
    \ge \dhsq \Pr[|\Sh - S - \Delta| \ge \Delta/2 \mid s+\Delta].
  \end{equation}

  Now suppose that we use the communication system in question for binary
  signaling. We want to send either $s$ or $s+\Delta$; at the decoder we use the
  estimate~$\Sh$ to decide for~$s$ or $s + \Delta$ depending on which one $\Sh$
  is closer to. When $s$ is sent, the decoder makes an error only if $|\Sh - s|
  \ge \Delta/2$; when $s + \Delta$ is sent, it makes an error only if $|\Sh - s
  - \Delta| \ge \Delta/2$. The conditional error probabilities therefore satisfy
  $\Pr[\text{error} | s] \le \Pr[|\Sh - S| \ge \Delta/2 \mid s]$ and
  $\Pr[\text{error} | s + \Delta] \le \Pr[|\Sh - S - \Delta| \ge \Delta/2 \mid s +
  \Delta]$. Applying this to~\eqref{eq:markov1} and~\eqref{eq:markov2} and
  inserting the result in~\eqref{eq:avglbd} yields
  \begin{equation}
    \label{eq:zivalmostproved}
    \mse \ge \pmin \dhsq \intabd \pe(s, \Delta) ds,
  \end{equation}
  where $\pe(s, \Delta) = \left(\Pr[\text{error}|s] + \Pr[\text{error}|s +
  \Delta] \right)/2$ is the average error probability.

  If $s$ and $s+\Delta$ are picked with equal probability and transmitted across
  $n$~parallel Gaussian channels as $\X(s)$ and $\X(s+\Delta)$, and if $d(s,
  \Delta) = \| \X(s) - \X(s + \Delta)\|$, then the error probability of the MAP
  decoder is $Q(d(s,\Delta) / 2 \sz)$, a standard result of communication theory
  (see \eg~\cite[Section~4.5]{WozencraftJ1965}). Because the MAP decoder minimizes
  the error probability, $Q(d(s,\Delta)/2\sz) \le \pe(s,\Delta)$, which, when
  inserted into~\eqref{eq:zivalmostproved}, completes the proof. \hfill\qed


  \section{Asymptotic Notation}
  \label{app:Onotation}
  The ``$O$-'' and ``$\Omega$-'' asymptotic notation used at various points in
  the paper is defined as follows.  Let $f(x)$ and $g(x)$ be two functions
  defined on~$\R$.  We write
  \begin{equation*}
    f(x) \in O(g(x))
  \end{equation*}
  if and only if there exists an $x_0$ and a constant~$c$ such that
  \begin{equation*}
    f(x) \le c g(x)
  \end{equation*}
  for all $x > x_0$. 

  Similarly, we write $f(x) \in \Omega(x)$ $\le$ is replaced by $\ge$ in the
  above definition.

\end{subappendices}


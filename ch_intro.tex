\chapter*{Introduction}

We cite~\cite{KleinerR2009} so that we don't get the warning about the empty
bibliography.

\subsubsection{Themes}

\begin{itemize}

  \item \textbf{Minimal-Delay Source-Channel Coding.} If coding delay and
    complexity is unconstrained, the region of achievable cost and distortion
    pairs for a given source, channel, cost measure and distortion measure is
    completely characterized by Shannon's separation theorem. On one hand, this
    theorem provides an outer bound to the achievability region; on the other
    hand it shows that any point in this region can be achieved using separate
    source and channel coding. Under a delay constraint, however, only the outer
    bound applies, and the exact achievability region is no longer known. 

    This thesis considers the strictest form of a delay constraint, where a
    single source symbol is to be encoded at a time. The focus is
    on the situation where the channel accepts inputs for transmission at a
    higher rate than the source produces them (this is sometimes called
    \emph{bandwidth expansion}). The analysis is concentrated on the Gaussian
    channel with an input power constraint and the squared error distortion
    measure. 

  \item \textbf{Optimal Cost--Fidelity Ratio.} The source-channel communication
    problem represents a tradeoff between cost and fidelity. Usually, the goal
    is either to maximize the fidelity obtained for a fixed transmission cost
    constraint, or to minimize the cost required to achieve a given fidelity of
    reproduction. The bigger the cost one is willing to pay, the bigger a
    fidelity one can obtain. 

    An alternative point of view is to look at the \emph{ratio} between cost and
    fidelity. This is the approach taken in \chapref{fidelity} of this thesis,
    where the question of interest is how to characterize the largest fidelity
    per cost.

  \item \textbf{Simulation.} For the communication engineer, simulations are a
    valuable tool. For all their limitations, they not only help to quickly test
    new ideas and to determine which ones deserve a more thorough theoretical
    analysis, they also allow one to perform a reality check on theoretical
    derivations.

    The ideas to be simulated are often simple enough that an engineer can write
    a complete simulator from the ground up. However, simulation code often
    becomes cluttered up from countless modifications in countless places, which
    gradually decreases clarity and productivity. Sometimes, a simulator changes
    so much over time that it is no longer possible to reproduce previous
    results.

    In \chapref{simulator}, this thesis explores how the object-oriented
    programming paradigm helps to manage the complexity of simulators while at
    the same time allowing rapid, undistracted implementation of new
    communication strategies. 

\end{itemize}


\subsubsection{Contributions}

\begin{enumerate}

  \item \textbf{Asymptotic Performance of a Minimal-Delay Communication
    Strategy.} A simple hybrid transmission strategy to transmit a single
    continuous-valued source symbol across $n$~uses of a Gaussian channel is
    analyzed. It consists of repeatedly quantizing the source symbol,
    transmitting the quantizer outputs in the first $n-1$~channel uses, and
    sending the remaining error uncoded in the $n\th$~channel use. This thesis
    provides the first exact characterization of the best asymptotic behavior of
    this strategy at large signal-to-noise ratios. It is shown that the
    asymptotic performance is strictly bounded away from the optimal behavior
    (if separate source and channel coding could be used), and that the gap (in
    dB) to the optimum increases with increasing SNR. 

  \item \textbf{Extensions of Minimal-Delay Communication Scheme.} Two
    extensions to the hybrid scheme are provided. The first extension is to
    encode several source symbols at a time using vector quantization with
    lattices. The second extension is to encode not one, but $k$ source symbols
    into $n$~channel uses, where $1 < k < n$. It is found that these extensions
    are subject to the same asymptotic performance limits as when a single
    source symbol is encoded at a time.

  \item \textbf{Connection to Feedback.} It is well known that a Gaussian source
    can be communicated optimally with minimal delay across $n$~uses of a
    Gaussian channel if the encoder has access to perfect feedback from the
    receiver. In this work,  the use of the hybrid communication scheme sketched
    above is justified by drawing on insights from the case with feedback.
    Furthermore, it is shown that there are clear parallels between this point
    of view and the more traditional way of analyzing minimal-delay
    source-channel codes geometrically. 

  \item \textbf{Characterization of Optimal Fidelity--Cost Ratio.} The three
    problems of source coding with a fidelity criterion, channel coding with an
    input cost constraint, and reproducing a source across a noisy channel all
    face a similar tradeoff between resource and performance. Of particular
    interest is the operating point with the highest performance per resource.
    In the case of channel coding, channel input cost is traded for rate, and
    the optimal tradeoff corresponds to the \emph{capacity per unit cost}, a
    subject that has been well studied in the past. This thesis defines the
    equivalent notions of \emph{fidelity per unit rate} for the source coding
    problem and \emph{fidelity per unit cost} for the end-to-end source-channel
    communication problem and shows how they relate. 

  \item \textbf{Matching Conditions for Fidelity Per Unit Cost.} It is known
    that when the channel transition distribution, the input distribution, and
    the input cost measure are matched in a particular way then the channel
    operates at capacity. Here a refined condition is provided under which the
    channel operates at capacity per unit cost. The condition under which a
    source operates at the rate-distortion function is refined in a similar way.
    When combined, these refined conditions yield a necessary and sufficient
    condition for a source-channel communication system to operate at fidelity
    per unit cost. 

  \item \textbf{An Object-Oriented Source-Channel Coding Simulator.} A full,
    workable implementation of a joint source-channel coding simulator in
    \matlab\ is presented. It allows rapid testing of ideas, while its structure
    is such that code remains clean even when many different configurations
    exist. 

\end{enumerate}


\subsubsection{Outline}



\chapter*{Introduction}
\addcontentsline{toc}{chapter}{Introduction}

The task of the communication engineer is to process information and to
transform it such that it can be transmitted reliably across an unreliable
medium. The information can be bits from a computer file, an audio signal
recorded from a microphone, images captured by a TV camera, or any other
physical signal. The unreliable medium is for example a wire undergoing
corruption by thermal noise, a wireless link subject to interference from other
transmitters, or a hard disk with occasional read errors.

The processing of information at the transmitter is called \emph{encoding}. It
is usually done on \emph{blocks} of data. For example, one kilobyte of a file is
encoded at a time, or one second of recorded voice. The more information is
encoded at once, the better the reconstruction quality one can expect at the
receiver. This is because a longer sequence of data is more likely to contain a
\emph{structure} that can be efficiently exploited. 

However, encoding long blocks of data at a time causes delay, as the encoder has
to wait until the source has produced enough data. Suppose your cell phone
encoded ten seconds of audio at once. This may increase the quality of your
voice at the receiver, but your conversation partner will hardly care about this
if she hears your voice only ten seconds after you have started talking.  

The subject of this thesis is communication under a strict delay constraint. In
particular, the following themes are addressed. 

\subsubsection{Themes}

\begin{itemize}

  \item \textbf{Minimal-Delay Source-Channel Coding.} If coding delay and
    complexity is unconstrained, the region of achievable cost and distortion
    pairs for a given source, channel, cost measure and distortion measure is
    completely characterized by Shannon's separation theorem. On one hand this
    theorem provides an outer bound to the achievability region; on the other
    hand it shows that any point in this region can be achieved using separate
    source and channel coding. Under a delay constraint, however, only the outer
    bound applies, and the exact achievability region is no longer known. 

    This thesis considers the strictest form of a delay constraint, where a
    single source symbol is to be encoded at a time. The focus is
    on the situation where the channel accepts inputs for transmission at a
    higher rate than the source produces them (this is sometimes called
    \emph{bandwidth expansion}). The analysis is concentrated on the Gaussian
    channel with an input power constraint and the squared error distortion
    measure. 

  \item \textbf{Optimal Cost--Fidelity Ratio.} The source-channel communication
    problem represents a tradeoff between cost and fidelity. Usually, the goal
    is either to maximize the fidelity obtained for a fixed transmission cost
    constraint, or to minimize the cost required to achieve a given fidelity of
    reproduction. The bigger the cost one is willing to pay, the bigger a
    fidelity one can obtain. 

    An alternative point of view is to look at the \emph{ratio} between cost and
    fidelity. This is the approach taken in \chapref{fidelity} of this thesis,
    where the question of interest is how to characterize the largest fidelity
    per cost.

  \item \textbf{Simulation.} For the communication engineer, simulations are a
    valuable tool. For all their limitations, they not only help to quickly test
    new ideas and to determine which ones deserve a more thorough theoretical
    analysis, but they also allow one to perform a reality check on theoretical
    derivations.

    The ideas to be simulated are in most cases simple enough that an engineer
    can write a complete simulator from the ground up. Over time, however,
    simulation code often becomes cluttered up from countless modifications in
    countless places, which gradually decreases clarity and productivity.
    Sometimes the original code changes so much that it is no longer possible to
    reproduce previous results.

    In \chapref{simulator}, this thesis explores how the object-oriented
    programming paradigm helps to manage the complexity of simulators while at
    the same time allowing rapid, undistracted implementation of new
    communication strategies. 

\end{itemize}


\subsubsection{Contributions}

\begin{enumerate}

  \item \textbf{Asymptotic Performance of a Minimal-Delay Communication
    Strategy.} A simple hybrid transmission strategy to transmit a single
    continuous-valued source symbol across $n$~uses of a Gaussian channel is
    analyzed. It consists of repeatedly quantizing the source symbol,
    transmitting the quantizer outputs in the first $n-1$~channel uses, and
    sending the remaining error uncoded in the $n\th$~channel use. This thesis
    provides the first exact characterization of the best asymptotic behavior of
    this strategy at large signal-to-noise ratios. It is shown that the
    asymptotic performance (signal-to-distortion ratio) is strictly bounded away
    from that achievable without a delay constraint, and that the gap (in dB) to
    the optimum increases with increasing signal-to-noise ratio.

  \item \textbf{Extensions of Minimal-Delay Communication Scheme.} Two
    extensions to the hybrid scheme are provided. The first extension is to
    encode several source symbols at a time using vector quantization with
    lattices. The second extension is to encode not one, but $k$ source symbols
    into $n$~channel uses, where $1 < k < n$. It is found that these extensions
    are subject to the same asymptotic performance limits as when a single
    source symbol is encoded at a time.

  \item \textbf{Connection to Feedback.} It is well known that a Gaussian source
    can be communicated optimally with minimal delay across $n$~uses of a
    Gaussian channel if the encoder has access to perfect feedback from the
    receiver. In this work,  the use of the hybrid communication scheme sketched
    above is justified by drawing on insights from the case with feedback.
    Furthermore, it is shown that there are clear parallels between this point
    of view and the more traditional way of analyzing minimal-delay
    source-channel codes geometrically. 

  \item \textbf{Characterization of Optimal Fidelity--Cost Ratio.} The three
    problems of source coding with a fidelity criterion, channel coding with an
    input cost constraint, and reproducing a source across a noisy channel all
    face a similar tradeoff between resource and performance. Of particular
    interest is the operating point with the highest performance per resource.
    In the case of channel coding, channel input cost is traded for rate, and
    the optimal tradeoff corresponds to the \emph{capacity per unit cost}, a
    subject that has been well studied in the past. This thesis defines the
    equivalent notions of \emph{fidelity per unit rate} for the source coding
    problem and \emph{fidelity per unit cost} for the end-to-end source-channel
    communication problem and shows how they relate. 

  \item \textbf{Matching Conditions for Fidelity Per Unit Cost.} It is known
    that when the channel transition distribution, the input distribution, and
    the input cost measure are matched in a particular way then the channel
    operates at capacity. Here a refined condition is provided under which the
    channel operates at capacity per unit cost. The condition under which a
    source operates at the rate-distortion function is refined in a similar way.
    When combined, these refined conditions yield a necessary and sufficient
    condition for a source-channel communication system to operate at fidelity
    per unit cost. 

  \item \textbf{An Object-Oriented Source-Channel Coding Simulator.} A full,
    workable implementation of a joint source-channel coding simulator in
    \matlab\ is presented. It allows rapid testing of ideas, while its structure
    is such that code remains clean even when many different configurations
    exist. 

\end{enumerate}


\subsubsection{Outline}

\bchapref{fundamentals} introduces the fundamentals of the source-channel
communication problem and defines the relevant performance measures as well as
their theoretical limits.  This is followed by a review of the general
conditions required for a system to operate at these limits, \ie, on the border
of the region of achievable cost and distortion pairs. Next, codes with a delay
constraint are discussed and related to the wider setting. The last part of the
chapter focuses on the particular case where a memoryless Gaussian source is to
be transmitted across a memoryless Gaussian channel with minimal delay, \ie,
when a single source symbol is to be encoded at a time. If one channel use is
available per source symbol, it is well known that uncoded transmission (with
minimal delay) achieves the theoretical limits. If more than one value can be
transmitted across the channel per source symbol, a similarly simple optimal
scheme exists only if the encoder has perfect feedback. These optimal schemes
are revisited, and some conclusions are drawn from the case with feedback on how
to design a good minimal-delay code for the situation without feedback if
several channel uses are available per source symbol.

\bchapref{mindelbwex} is concerned purely with transmission across Gaussian
channels. It first looks at the problem of minimal-delay bandwidth-expansion
coding from two points of view: one inspired by the example of optimal coding
with feedback; the other is a more traditional geometric analysis of the signal
constellation. Motivated by these reflections, the chapter then introduces a
concrete hybrid transmission scheme to transmit a single analog source symbol
across several channel uses.  Asymptotic upper bounds as well as asymptotic
lower bounds on the achieved mean squared error as a function of the SNR are
given, and they are shown to coincide. The hybrid communication scheme is
extended to two more general cases, and the respective asymptotic performance is
given. The last section of the chapter looks ahead and suggests possible
directions to obtain a more general characterization of the performance of
minimal-delay schemes. 

The thesis returns to arbitrary sources and channels in \bchapref{fidelity}.
Instead of finding the best cost--performance tradeoff, the goal of this chapter
is to characterize the best \emph{ratio} between performance and cost. For this,
the notion of distortion is replaced with a (mathematically equivalent) notion
of \emph{fidelity}. By refining previous results, necessary and sufficient
conditions are found for a memoryless point-to-point communication system to
operate at the maximal fidelity per cost ratio. 

\bchapref{simulator} is of a more practical nature. It argues that
object-oriented programming is particularly suitable for implementing
simulations and presents a complete simulator for joint source-channel coding
that allows for fast and easy testing of arbitrary point-to-point communication
schemes. 

Lastly, \bchapref{conclusions} presents conclusions and possible directions for
future research.


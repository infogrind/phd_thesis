\chapter{A Simulator for Joint Source-Channel Codes}%

\section{Introduction}

This section introduces the simulator and explains why the object-oriented
programming paradigm, especially inheritance, is particularly suited for the
task of simulating many similar communication strategies.

There are two main advantages to the object-oriented approach. The first is the
ability to quickly produce an alternative version of a scheme with only small
changes, without the need to copy a complete file or modify the original. This
is demonstrated by the many variations of a quantize/uncoded scheme. The second
advantage is that the code to run and to evaluate the results of the many
schemes need only be written one for a shared interface. 


\section{A Step-By-Step Tutorial}


\section{Class Hierarchy and Reference}

\subsection{Communication Strategies}

\subsection{Processors}

\subsection{Output}

\subsection{Batch Processing and Makefile Inclusion}


\section{Implementation Notes}

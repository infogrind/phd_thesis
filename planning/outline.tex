%        File: outline2.tex
%      Author: Marius Kleiner <marius.kleiner@epfl.ch>
%     Created: Wed Oct 21 01:00 PM 2009 C
% Last Change: Wed Oct 21 01:00 PM 2009 C
%
% $Id$
%
\documentclass[a4paper]{article}
\bibliographystyle{IEEEtran}
\usepackage[font={small,sf},labelfont={bf,sf}]{caption}
\usepackage{graphicx}
\usepackage{subfig}
\usepackage{amsmath}
\usepackage{amsfonts}
\usepackage{amsthm}
\usepackage{tikz}


% Theorem definitions etc
\newtheorem{theorem}{Theorem}
\newtheorem{lemma}[theorem]{Lemma}

\theoremstyle{definition}
\newtheorem{definition}[theorem]{Definition}
\newtheorem{example}{Example}

% Useful macros
\newcommand{\ud}{\mathrm{d}}
\newcommand{\vect}[1]{\mathbf{#1}}
\newcommand{\mat}[1]{\mathbf{#1}}
\newcommand{\ie}{i.e.}
\newcommand{\eg}{e.g.}
\newcommand{\abbrev}[1]{\textsc{\MakeLowercase{#1}}}
\newcommand{\snr}{\abbrev{snr}}
\newcommand{\sdr}{\abbrev{sdr}}
\let\e\epsilon
\newcommand\cdf{\abbrev{CDF}}
\newcommand\N{\mathcal{N}}
\newcommand{\cE}{\mathcal{E}}
\newcommand{\y}{\vect{y}}
\newcommand{\Y}{\vect{Y}}
\newcommand{\x}{\vect{x}}
\newcommand{\X}{\vect{X}}
\newcommand{\sh}{\hat{s}}
\newcommand{\E}{\mathbb{E}}
\newcommand{\deq}{\stackrel{\Delta}{=}}
\DeclareMathOperator{\Erf}{Erf}
\newcommand{\Sh}{\hat{S}}
\newcommand{\Shh}{\hat{\hat{S}}}
\newcommand{\Z}{\mathbb{Z}}

\frenchspacing


\title{Weekly Notes}
\author{Marius Kleiner}
\date{October 21, 2009\\\small(last updated: \today)}

\begin{document}
\maketitle

\section{Introduction}

\section{On Low-Delay Joint Source/Channel Coding}
\subsection{Background}
\begin{itemize}
  \item Fundamental limits of source-channel communication: separation theorem
    (general case $n \ne k$), in particular the converse
  \item Achievability: using long block codes
  \item Necessary and sufficient conditions for achievability (these hold
    irrespective of whether separation is used or a joint code) 
  \item How in separation these conditions can be achieved using long codes
  \item If delay is constrained, these conditions cannot be all simultaneously
    satisfied (cf.~Ingber et al.\ result)
  \item Relaxed conditions if all we need is asymptotic order equality (i.e., $D
    \in O(\textsc{snr}^{-n})$). For example, it is enough for the input
    distribution to be non-degenerate at high SNR.
  \item Robust vs.\ non-robust. (in general, the encoder can also be a function
    of the SNR; in the robust case it isn't. Same for the decoder.)
  \item Ziv's result about robust codes, and how geometrical constraints
    influence the performance. Counter-example: fractal codes that perform
    better (Taherzadeh\slash Khandani)
  \item Intuition why (piecewise) continuous constellations limit the
    performance of robust codes. (If you look at a small enough region of the
    constellation, only one dimension is used.) -- Can we extend this to Ziv's
    general constraint?
\end{itemize}

\subsection{Definitions and Problem Setting}

\subsection{A Quantize-and-Scale Scheme}
\subsubsection{Definition}
\begin{itemize}
  \item encoder
  \item decoder
  \item comment about suboptimality
\end{itemize}
\subsubsection{Achievable Distortion}
\begin{itemize}
  \item Upper bound on distortion resp.\ lower bound on SDR
  \item Distinction between scaling \emph{exponent} and \emph{order} of scaling
    ($-n$ vs.\ $O(\snr^{-n}(\log\snr)^{n-1})$)
  \item Best choice of~$\e$
\end{itemize}
\subsubsection{Distortion Lower Bound for Quantize/Scale Schemes}
\begin{itemize}
  \item Ziv-based proof how the stretch implies a distortion lower bound, no
    matter how good the decoder is
  \item Intuition behind this?
\end{itemize}

\subsection{General Points on Minimum Delay Joint S-C Codes}
\begin{itemize}
  \item How to best use the additional available dimensions
  \item One way: Scale up by $\snr^n$, then cut and distribute in other $n-1$
    dimensions. However, we have to scale a bit less; some of the available SNR
    needs to ``go into'' increasing the space between constellation segments.
  \item There is a gap term in all the robust schemes by T\&K. Their last
    scheme, which varies with SNR, is claimed to scale as $\snr^{-n}$, but there
    are problems in the proof. It would be nice to sort these things out to
    definitely answer the question whether $\snr^{-n}$ scaling is possible with
    minimal delay. 
\end{itemize}


\section{On Fidelity per Unit Cost}
\begin{itemize}
  \item ISIT paper + more examples
  \item More thoughts about practical applications, especially of fidelity per
    unit rate. (In the capacity case, both cost and rate are additive
    quantities, and we can e.g.\ find the bandwidth at which to obtain the
    largest rate, as shown in the example in the paper. However, we haven't yet
    found any such example in the fidelity per unit rate case.)
\end{itemize}


\section{MATLAB Simulator}
\subsection{Introduction and First Examples}
\subsection{Functional Specification/Reference}
\subsection{Architecture and Implementation}

\section{Conclusion and Future Work}

% Uncomment once you have added citations
%\bibliography{mkbiblio}


\end{document}


